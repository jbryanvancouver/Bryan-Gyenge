\documentclass{amsart}

%\usepackage{authblk}


\title{$G$-fixed Hilbert schemes on $K3$ surfaces, modular forms, and
eta products}
\author{Jim Bryan and { \'{A}d\'{a}m~Gyenge}}
%\author{Adam Gyenge}
\date{\today}



\address{
Department of Mathematics\\
University of British Columbia \\
Room 121, 1984 Mathematics Road \\
Vancouver, B.C., Canada V6T 1Z2\\
}
\email{jbryan@math.ubc.ca}

\address{Mathematical Institute\\
University of Oxford, Andrew Wiles Building\\
Woodstock Road, Oxford, OX2 6GG, United Kingdom\\
}
\email{Adam.Gyenge@maths.ox.ac.uk}



%\usepackage{diagrams}

\usepackage{amsmath}
\usepackage{bm}
\usepackage{verbatim}
\usepackage{amsmath,amsthm,amsfonts}
\usepackage{amssymb}
\usepackage{times}
\usepackage{longtable}
\usepackage{tikz-cd}
%\usepackage{amstex}




\newtheorem{theorem}{Theorem}[section]
\newtheorem{proposition}[theorem]{Proposition}
\newtheorem{conjecture}[theorem]{Conjecture}
\newtheorem{lemma}[theorem]{Lemma}
\newtheorem{corollary}[theorem]{Corollary}
\newtheorem{convention}{Convention}[theorem]

\theoremstyle{definition}

\newtheorem{def-theorem}[theorem]{Definition-Theorem}
\newtheorem{remark}[theorem]{Remark}
\newtheorem{definition}[theorem]{Definition}
\newtheorem{example}[theorem]{Example}


\newcommand{\half}{\frac{1}{2}}
\newcommand{\CC} {{\mathbb C}}          % complex numbers
\newcommand{\NN} {{\mathbb N}}		% natural numbers
\newcommand{\RR} {{\mathbb R}}		% real numbers
\newcommand{\ZZ} {{\mathbb Z}}		% integers
\newcommand{\QQ} {{\mathbb Q}}		% rationals
\newcommand{\PP}{\mathbb{P}}
\newcommand{\HH}{\mathbb{H}}
\newcommand{\LL}{\mathbb{L}}
\newcommand{\X}{\mathfrak{X}}
\newcommand{\M}{\mathfrak{M}}
\newcommand{\bir}{\mathsf{bir}}
\newcommand{\reg}{\mathsf{reg}}
\newcommand{\Ell}{\mathsf{Ell}}
\newcommand{\Ellqy}{\operatorname{Ell}_{q,y}}
\newcommand{\chibar}{\overline{\chi}}
%\newcommand{\mdata}{\left\{m_{j}(i) \right\}}
\newcommand{\mdata}{\mathfrak{m}}
\newcommand{\FM}{\mathsf{FM}}

\renewcommand{\O}{\mathcal{O}}
%\renewcommand{\top}{\mathrm{t}}
\renewcommand{\top}{\,\mathsf{t}}
\newcommand{\varp}{\mathsf{p}}
\newcommand{\varq}{\mathsf{q}}
\newcommand{\varr}{\mathsf{r}}
\newcommand{\vara}{\mathsf{a}}


\newcommand{\Zcal}{\mathcal{Z}}
\newcommand{\mvec}{\bm{m}}
\newcommand{\zetavec}{\bm{\zeta }}
\newcommand{\rhovec}{\bm{\rho }}
\newcommand{\dvec}{\bm{d }}
\newcommand{\uvec}{\bm{u }}
\newcommand{\vvec}{\bm{v }}
\newcommand{\wvec}{\bm{w }}

%\newcommand{\mvec}{\vec{m}}
%\newcommand{\zetavec}{\vec{\zeta }}
%\newcommand{\dvec}{\vec{d }}
%\newcommand{\uvec}{\vec{u }}
%\newcommand{\vvec}{\vec{v }}


\newcommand{\Hom}{\operatorname{Hom}}
\newcommand{\Ker}{\operatorname{Ker}}
\newcommand{\End}{\operatorname{End}}
\newcommand{\Tr}{\operatorname{tr}}
\newcommand{\tr}{\operatorname{tr}}
\newcommand{\Sym}{\operatorname{Sym}}
\newcommand{\Coker}{\operatorname{Coker}}
\newcommand{\im}{\operatorname{Im}}
\newcommand{\Hilb}{\operatorname{Hilb}}

\sloppy
\allowdisplaybreaks

\begin{document}



%\tableofcontents

\begin{abstract}
Let $X$ be a complex $K3$ surface with an effective action of a group
$G$ which preserves the holomorphic symplectic form. Let 
\[
Z_{X,G}(q) = \sum_{n=0}^{\infty} e\left(\Hilb^{n}(X)^{G} \right)\, q^{n-1}
\]
be the generating function for the Euler characteristics of the
Hilbert schemes of $G$-invariant length $n$ subschemes. We show that
its reciprocal, $Z_{X,G}(q)^{-1}$ is the Fourier expansion of a
modular cusp form of weight $\half e(X/G)$ for the congruence subgroup
$\Gamma_{0}(|G|)$. We give an explicit formula for $Z_{X,G}$ in terms
of the Dedekind eta function for all 82 possible $(X,G)$. The key
intermediate result we prove is of independent interest: it
establishes an eta product identity for a certain shifted theta
function of the root lattice of a simply laced root system. We extend
our results to various refinements of the Euler characteristic, namely
the Elliptic genus, the Chi-$y$ genus, and the motivic class.
\end{abstract}


\maketitle
%\markboth{???}  {???}



%\tableofcontents
%\pagebreak


\section{Introduction}

Let $X$ be a complex $K3$ surface with an effective action of a group
$G$ which preserves the holomorphic symplectic form. Mukai showed that
such $G$ are precisely the subgroups of the Mathieu group
$M_{23}\subset M_{24}$ such that the induced action on the set
$\{1,\dots ,24 \}$ has at least five orbits
\cite{mukai1988finite}. Xiao classified all possible actions into
82 possible topological types of the quotient $X/G$ \cite{xiao1996galois}.

The \emph{$G$-fixed Hilbert scheme}\footnote{Some authors call this
the $G$-equivariant Hilbert scheme or the $G$-invariant Hilbert
scheme.} of $X$ parameterizes $G$-invariant length $n$ subschemes
$Z\subset X$. It can be identified with the $G$-fixed point locus in
the Hilbert scheme of points:
\[
\Hilb^{n}(X)^{G} \subset \Hilb^{n}(X)
\]

We define the corresponding \emph{$G$-fixed partition function} of
$X$ by
\[
Z_{X,G}(q) = \sum_{n=0}^{\infty} e\left(\Hilb^{n}(X)^{G} \right) q^{n-1} 
\]
where $e(-)$ is topological Euler characteristic.

Throughout this paper we set
\[
q=\exp\left(2\pi i \tau  \right)
\]
so that we may regard $Z_{X,G}$ as a function of $\tau \in \HH$ where
$\HH$ is the upper half-plane.

Our main result is the following:

\begin{theorem}
\label{thm:main} The function $Z_{X,G}(q)^{-1}$ is a modular cusp
form\footnote{By cusp form, we mean that the order of vanishing at
$q=0$ is at least 1. Modular forms of half integral weight transform with
respect to a multiplier system. We refer to \cite{kohler2011eta} for
definitions.} of weight $\half e(X/G)$ for the congruence subgroup
$\Gamma_{0}(|G|)$.
\end{theorem}
   

Our theorem specializes in the case where $G$ is the trivial group to
a famous result of G\"ottsche \cite{gottsche1990betti}. The case where
$G$ is a cyclic group was proved in \cite{bryan2018chl}.

% One can interpret our result as an instance of the Vafa-Witten
% S-duality conjecture for the orbifold $[X/G]$.

We give an explicit formula for $Z_{X,G}(q)$ in terms of the
Dedekind eta function
\[
\eta (\tau ) = q^{\frac{1}{24}}\prod_{n=1}^{\infty} (1-q^{n})
\]
as follows. Let $p_{1},\dots ,p_{r}$ be the singular points of $X/G$
and let $G_{1},\dots ,G_{r}$ be the corresponding stabilizer subgroups
of $G$. The singular points are necessarily of ADE type: they are
locally given by $\CC^{2}/G_{i}$ where $G_{i}\subset SU(2)$. Finite
subgroups of $SU(2)$ have an ADE classification and we let
$\Delta_{1},\dots ,\Delta_{r}$ denote the corresponding ADE root
systems.

For any finite subgroup $G_{\Delta}\subset SU(2)$ with associated root
system $\Delta$ we define the \emph{local $G_{\Delta }$-fixed
partition function} by
\[
Z_{\Delta} (q) = \sum_{n=0}^{\infty}
e\left(\Hilb^{n}(\CC^{2})^{G_{\Delta}} \right) \, q^{n-\frac{1}{24}} .
\]
The main geometric result we prove is the following. 
\begin{theorem}\label{thm: formula for local series (in intro)}
The local partition function for $\Delta$ of type $A_{n}$ is given by 
\[
Z_{A_{n}}(q) = \frac{1}{\eta (\tau )}
\]
and for type $D_{n}$ and $E_{n}$ by
\[
Z_{\Delta}(q) = \frac{\eta^{2}(2\tau )\eta (4E\tau )}{\eta (\tau )\eta (2E\tau )\eta (2F\tau )\eta (2V\tau )}
\]
where $(E,F,V)$ are given by\footnote{For $\Delta$ of type
$D_{n}$ or $E_{n}$, the group $H = G_{\Delta}/\{\pm 1 \}\subset SO(3)$
is the symmetry group of a polyhedral decomposition of $S^{2}$ into
isomorphic regular spherical polygons. Then $E$, $F$, and $V$ are the
number of edges, faces, and vertices of the polyhedron.}:
\[
(E,F,V) = \begin{cases}
(n-2,2,n-2), & \Delta =D_{n}\\
(6,4,4), & \Delta =E_{6}\\
(12,8,6), & \Delta =E_{7}\\
(30,20,12), & \Delta =E_{8}\\
\end{cases}
\]
\end{theorem}
Our proof of the above Theorem uses a trick exploiting the derived
McKay correspondence between $\X =[\CC^{2}/\{\pm 1 \}]$ and
$Y=\operatorname{Tot}(K_{\PP^{1}})$, (see sections \ref{sec: proof of
formula for local series} and \ref{sec: McKay correspondence}).




We will also prove in Lemma~\ref{lem: local series as theta/eta}  that
\[
Z_{\Delta}(q) =\frac{\theta_{\Delta}(\tau)}{\eta (k\tau )^{n+1}}
\]
where 
\[
\theta_{\Delta}(\tau ) = \sum_{\mvec \in M_{\Delta}}
q^{\frac{k}{2}\left(\mvec +\frac{1}{k}\zetavec |\mvec
+\frac{1}{k}\zetavec \right)}
\]
is a shifted theta function for $M_{\Delta} $, the root lattice of
$\Delta$. Here $n$ is the rank of the root system, $k=|G_{\Delta}|$, and
$\zetavec$ is dual to the longest root (see section~\ref{sec: local
partition functions} for details).

Theorem~\ref{thm: formula for local series (in intro)} then yields an
eta product identity for the theta function $\theta_{\Delta}(\tau )$
reminiscent of the MacDonald identities:
\begin{theorem}\label{thm: eta product for theta function}
The shifted theta function $\theta _{\Delta}(\tau )$ defined above
(c.f. \S~\ref{sec: local partition functions}) is given by an eta
product as follows:

For $\Delta$ of type $A_{n}$
\[
\theta_{A_{n}} (\tau ) = \frac{\eta^{n+1} ((n+1)\tau )}{\eta (\tau )}
\]
and for $\Delta$ of type $D_{n}$ or $E_{n}$ 
\[
\theta_{\Delta}(\tau ) = \frac{\eta^{2}(2\tau )\,\eta ^{n+2}(4E\tau
)}{\eta (\tau )\, \eta (2E\tau )\,\eta (2F\tau )\,\eta (2V\tau )}
\]
where $E,F,V$ are as in Theorem~\ref{thm: formula for local series (in intro)}.
\end{theorem}



The 82 possible collections of ADE root systems $\Delta_{1},\dots
,\Delta_{r}$ associated to $(X,G)$ a $K3$ surface with a symplectic
$G$ action, are given in Appendix~\ref{app:tableeta},
Table~\ref{table: list of eta products}. We let $k=|G|$,
$k_{i}=|G_{i}|$, and
\[
a = e(X/G) - r=\frac{24}{k}-\sum_{i=1}^{r} \frac{1}{k_{i}}.
\]

The global series $Z_{X,G}(q)$ can be expressed as a product of local
contributions (and thus via Theorem~\ref{thm: formula for local series
(in intro)} as an explicit eta product) by our next result:
\begin{theorem}\label{thm: eta product formula for Z}
With the above notation we have
\[
Z_{X,G}(q) = \eta^{-a}(k\tau )\prod_{i=1}^{r}
Z_{\Delta_{i}}\left(\frac{k\tau}{k_{i}} \right).
\]
\end{theorem}

Theorem~\ref{thm:main} then immediately follows from Theorem~\ref{thm:
Zorbifold formula} and Theorem~\ref{thm: eta product formula for Z}
using the formulas for the weight and level of an eta product given in
\cite[\S~2.1]{kohler2011eta}.



In Appendix~\ref{app:tableeta}, Table~\ref{table: list of eta
products} we have listed explictly the eta product of the modular form
$Z_{X,G}(q)^{-1}$ for all 82 possible cases of $(X,G)$.  Having
obtained such expressions allows us to make several observational
corollaries:

\begin{corollary}\label{cor: if G is a subgp of E then Zinv is a Hecke
eigenform} If $G$ is a finite subgroup of an elliptic curve $E$,
i.e. $G$ is isomorphic to a product of one or two cyclic groups, then
$Z_{X,G}(q)^{-1}$ is a Hecke eigenform. In Table~\ref{table: list of
eta products} these are the 13 cases having Xiao number in the set
$\{0,1,2,3,4,5,7,8,11,14,15,19,25 \}$. Moreover, in each of these
cases, the dimension of the Hecke eigenspace is one.
\end{corollary}

We remark that in these cases, we may form a Calabi-Yau threefold
called a CHL model by taking the free group quotient
\[
(X\times E)/G
\]
Then the partition function $Z_{X,G}(q)$ gives the Donaldson-Thomas
invariants of $(X\times E)/G$ in curve classes which are degree zero
over $X/G$ (c.f. \cite{bryan2018chl}).


For any eta product expression of a modular form, one may easily
compute the order of vanishing (or pole) at any of the cusps
\cite[Cor~2.2]{kohler2011eta}. Performing this computation on the 82
cases yields the following

\begin{corollary}\label{cor: vanishing at cusps}
The modular form $Z_{X,G}(q)^{-1}$ always vanishes with order 1 at the cusps
$i\infty$ and $0$. Moreover,
%\begin{itemize}
%\item $Z_{X,G}(q)^{-1}$ vanishes at all cusps except for the eleven
%cases with Xiao number in the set $\{13,20,27,29,37,38,45,53,54,60,69 \}$.
%\item
$Z_{X,G}(q)^{-1}$ is holomorphic at all cusps except for the two cases with
Xiao number 38 or 69, which have poles at the cusps $1/2$ and $1/8$
respectively. These are precisely the cases where $X/G$ has two
singularities of type $E_{6}$.
%\end{itemize}
\end{corollary}

\begin{remark}\label{rem: enumerative interpretation}
The integers $e\left(\Hilb^{n}(X)^{G} \right)$ should have enumerative
significance: they can be interpreted as virtual counts of
$G$-invariant curves, whose quotient is rational, in a complete linear
series of dimension $n$ on $X$. This generalizes the famous Yau-Zaslow
formula in the case where $G$ is the trivial group. The precise nature
between the virtual count and the actual count is expected to be
subtle for the case of general $G$. This has been recently explored in
\cite{Zhan-2019-counting-curves-on-K3}.
\end{remark}


We can extend our results to various refinements of the Euler
characteristic, namely the elliptic genus, the $\chi_{y}$ genus, and
the motivic class. These refinements all stem from the next
result which we prove in Section~\ref{sec: proof of thm about Zbir}. Let 
\[
Z^{\bir}_{X,G}(q) = \sum_{n=0}^{\infty} [\Hilb^{n}(X)^{G}]_{\bir} \,
\, q^{n-1}
\]
be a formal series whose coefficients we regard as birational
equivalence classes of compact hyperkahler manifolds. Such equivalence
classes form a semi-ring under disjoint union and Cartesian
product.

\begin{theorem}\label{thm: Formula for Zbir}
Let $Y$ be the minimal resolution of $X/G$, then
\[
Z^{\bir}_{X,G} (q) = Z^{\bir}_{Y}(q^{k})\cdot Z_{X,G}(q)\cdot \Delta (k\tau )
\]
where $k=|G|$, $\Delta (\tau ) = \eta (\tau )^{24}$, and we have
suppressed the trivial group from the notation in the series
$Z^{\bir}_{Y}(q^{k})$.
\end{theorem}

A famous theorem of Huybrechts \cite[Thm~4.6]{Huybrechts} asserts that
birational compact hyperkahler manifolds are deformation
equivalent. Moreover, combining Huybrecht's theorem with
\cite[Prop~3.21]{Nicaise-Shinder} it follows that birational compact
hyperkahler manifolds are equal in $K_{0}(\operatorname{Var}_{\CC})$,
the Grothendieck group of varieties.

Thus we may specialize the series $Z^{\bir}_{X,G}(q)$ to Elliptic
genus, motivic class, and Chi-$y$ genus since these are all well
defined on birational equivalence classes of compact hyperkahler
manifolds. These specializations are all well known for the series
$Z^{\bir}_{Y}$ and hence we easily get the following corollaries.

\begin{corollary}\label{cor: Zell formula}
Let $Q=\exp\left(2\pi i\sigma \right)$ and let
\[
Z^{\Ell}_{X,G}(Q,q,y) = \sum_{n=0}^{\infty} \Ellqy
\left(\Hilb^{n}(X)^{G} \right) Q^{n-1}
\]
where $\Ellqy (-)$ is elliptic genus. Then
\[
Z^{\Ell}_{X,G}(Q,q,y) = \frac{\phi_{10,1}(q ,y)}{\chi_{10}(k\sigma
,\tau ,z)} \cdot Z_{X,G}(q)\cdot \Delta (k\tau )
\]
where $\phi_{10,1}(q ,y)$ is the Fourier expansion of the unique Jacobi cusp form of weight
10 and index 1 and $\chi_{10}(\sigma ,\tau ,z)$ is Igusa's genus 2
Siegel cusp form of weight 10. 
\end{corollary}
We refer the reader to \cite{Pietromonaco_2018} (\S 5, \S 6, and
eqn~6.9.8) for definitions of $\Ellqy$, $\phi_{10,1}$, $\chi_{10}$,
and the formula for the elliptic genera of $\Hilb^{n}(Y)$.

A further specialization of particular interest is the (normalized)
$\chi_{y}$ genus. Let 
\[
\chibar_{-y}(M) = y^{-\half \dim
M}\chi_{-y}(M)
\]
and we note that $\chibar_{-y}(M) = \Ellqy
(M)|_{q=0}$.

\begin{corollary}\label{cor: Zchiy formula}
Let
\[
Z^{\chibar}_{X,G}(q,y) = \sum_{n=0}^{\infty}
\chibar_{y}\left(\Hilb^{n}(X)^{G} \right) q^{n-1} .
\]
Then
\[
Z^{\chibar}_{X,G}(q,y) = y^{-1}(1-y)^{2} \frac{Z_{X,G}(q)}{\phi_{-2,1}(q^{k},y)}
\]
where $\phi_{-2,1}$ is the unique weak Jacobi form of weight -2
and index 1. In particular,
\[
y^{-1}(1-y)^{2} Z^{\chibar}_{X,G}(q,y)^{-1} =\frac{
\phi_{-2,1}(q^{k},y)}{Z_{X,G}(q)} 
\]
is a Jacobi form of index 1 and weight 
\[
\half e(X/G)-2 = 10-\half \sum_{i=1}^{r} \operatorname{rank} \Delta_{i}
\]
for the congruence subgroup $\Gamma_{1}(k)$. 
\end{corollary}

We note that for $G$ cyclic, the series
$Z_{X,G}(q)/\phi_{-2,1}(q^{k},y) $ is the leading coefficient in the
expansion of the Donaldson-Thomas partition function of $(X\times
E)/G$ in the variable tracking the curve class in $X$ (see
\cite[Thm~0.1]{bryan2018chl}).

We also get a formula for the motivic classes of the $G$-fixed Hilbert schemes:


\begin{corollary}\label{cor: ZK0 formula}
Let 
\[
Z^{K_{0}}_{X,G}(q) = \sum_{n=0}^{\infty} [\Hilb^{n}(X)^{G}]_{K_{0}}\,\, q^{n-1}
\]
where $ [\Hilb^{n}(X)^{G}]_{K_{0}}\in K_{0}(\operatorname{Var}_{\CC})$
denotes the motivic class of the $G$-fixed Hilbert scheme. Then
\[
Z^{K_{0}}_{X,G}(q) = q^{-1}\cdot \prod_{m=1}^{\infty} \left(1-\LL^{m-1}
q^{km} \right)^{-[Y]}  \cdot Z_{X,G}(q)\cdot \Delta (k\tau )
\]
where $\LL  = [\mathbb{A}^{1}_{\CC}]\in
K_{0}(\operatorname{Var}_{\CC})$.
\end{corollary}
We refer the reader to \cite{GZ-L-MH-power} for the meaning
of $[Y]$ in the exponent and the formula for the motivic class of
$\Hilb^{n}(Y)$. The above series has further specializations giving formulas for the  Hodge polynomials and
Poincare polynomials of the $G$-fixed Hilbert schemes.



\subsection{Acknowledgements.} The authors warmly thank Jenny Bryan,
Federico Amadio Guidi, Georg Oberdieck, Ken Ono, Stephen Pietromonaco,
Bal\'azs Szendr\H{o}i, and Alex Weekes for helpful comments and/or
technical help.

\section{The local partition functions}\label{sec: local partition functions}

The classical McKay correspondence associates an ADE root system
$\Delta$ to any finite subgroup $ G_{\Delta}\subset SU(2)$. Using
the work of Nakajima \cite{nakajima2002geometric}, the partition function of the
Euler characteristics of the Hilbert scheme of points on the stack
quotient $[\CC^{2}/G_{\Delta}]$ was computed explicitly in
\cite{gyenge2015euler} in terms of the root data of $\Delta$.

The local partition functions $Z_{\Delta}(q)$ considered in this paper
are obtained from a specialization of the partition functions of the
stack $[\CC^{2} /G_{\Delta}]$ and in this section, we use this to express
$Z_{\Delta}(q)$ in terms of a shifted theta function for the root
lattice of $\Delta$.

A zero-dimensional substack $Z\subset [\CC^{2}/G_{\Delta}]$ may be
regarded as a $G_{\Delta}$ invariant, zero-dimensional subscheme of
$\CC^{2}$. Consequently, we may identify the Hilbert scheme of points
on the stack $[\CC^{2}/G_{\Delta}]$ with the $G_{\Delta}$ fixed locus
of the Hilbert scheme of points on $\CC^{2}$: 
\[
\Hilb \left([\CC^{2}/G_{\Delta}] \right) = \Hilb
(\CC^{2})^{G_{\Delta}} .
\]

This Hilbert scheme has components indexed by representations $\rho$
of $G_{\Delta}$ as follows
\begin{equation*}
\Hilb^{\rho} \left([\CC^{2}/G_{\Delta}] \right) = \left\{ Z\subset
\CC^{2}, \text{ $Z$ is $G_{\Delta}$ invariant and $H^{0}(\O_{Z})\cong
\rho $} \right\}.
\end{equation*}

Let $\{\rho_{0},\dotsc ,\rho_{n} \}$ be the irreducible
representations of $G_{\Delta}$ where $\rho_{0}$ is the trivial
representation. We note that $n$ is also the rank of $\Delta$. We
define
\[
Z_{[\CC^{2}/G_{\Delta}]} (q_{0},\dotsc ,q_{n}) = \sum_{m_{0},\dotsc
,m_{n}=0}^{\infty} e\left(\Hilb^{m_{0}\rho_{0}+\dotsb
+m_{n}\rho_{n}}([\CC^{2}/G_{\Delta}]) \right) q_{0}^{m_{0}}\dotsb
q_{n}^{m_{n}} .
\]

Recall that our local partition function $Z_{\Delta}(q)$ is defined by
\[
Z_{\Delta}(q) = \sum_{n=0}^{\infty}
e\left(\Hilb^{n}(\CC^{2})^{G_{\Delta}} \right) q^{n-\frac{1}{24}}. 
\]
We then readily see that
\[
Z_{\Delta}(q) = q^{\frac{-1}{24}}\cdot  Z_{[\CC^{2}/G_{\Delta}]}
(q_{0},\dotsc ,q_{n})|_{q_{i}=q^{d_{i}}}
\]
where
\[
d_{i} =\dim \rho_{i}.
\]

The following formula is given explicitly in \cite[Thm~1.3]{gyenge2015euler}, but its content is already present in the work of Nakajima \cite{nakajima2002geometric}:
\begin{theorem} \label{thm: Zorbifold formula}
Let $C_{\Delta}$ be the Cartan matrix of the root system $\Delta$,
then 
\[
Z_{[\CC^{2}/G_{\Delta}]} (q_{0},\dotsc ,q_{n}) = \prod_{m=1}^{\infty}
(1-Q^{m})^{-n-1} \cdot \sum_{\mvec \in \ZZ^{n}} q_{1}^{m_{1}}\dotsb
q_{n}^{m_{n}} \cdot Q^{\half \mvec^{\top}\cdot C_{\Delta}\cdot \mvec}
\]
where $Q=q_{0}^{d_{0}}q_{1}^{d_{1}}\dotsb q_{n}^{d_{n}}$.
\end{theorem}
We note that under the specialization $q_{i}=q^{d_{i}}$, 
\[
Q=q^{d_{0}^{2}+\dotsb +d_{n}^{2}} = q^{k}
\]
where $k=|G|$ is the order of the group $G$.

We then obtain
\[
Z_{\Delta}(q) = q^{\frac{-1}{24}}\cdot
\prod_{m=1}^{\infty}(1-q^{km})^{-n-1} \cdot \sum_{\mvec \in \ZZ^{n}}
q^{\mvec^{\top}\cdot \dvec} \cdot q^{\frac{k}{2}\mvec^{\top}\cdot C_{\Delta}\cdot \mvec}
\]
where $\dvec =(d_{1},\dotsc ,d_{n})$.

Let $M_{\Delta}$ be the root lattice of $\Delta$ which we identify
with $\ZZ^{n}$ via the basis given by $\alpha_{1},\dotsc ,\alpha_{n}$,
the simple positive roots of $\Delta$. Under this identification, the
standard Weyl invariant bilinear form is given by
\[
(\uvec |\vvec ) = \uvec^{\top}\cdot C_{\Delta}\cdot \vvec 
\]
and $\dvec$ is identified with the longest root. 
We define
\[
\zetavec = C_{\Delta}^{-1} \cdot \dvec 
\]
so that 
\[
\mvec^{\top}\cdot \dvec = \mvec^{\top}\cdot C_{\Delta} \cdot \zetavec
= (\mvec |\zetavec ).
\]
We may then write
\begin{align*}
Z_{\Delta}(q)& = q^{\frac{-1}{24}}\cdot
\prod_{m=1}^{\infty}(1-q^{km})^{-n-1}\cdot \sum _{\mvec \in M_{\Delta}} q^{(\mvec |\zetavec )+\frac{k}{2}(\mvec |\mvec)}\\
&= q^{A} \cdot \left(q^{\frac{k}{24}}\prod_{m=1}^{\infty}(1-q^{km})
\right)^{-n-1} \cdot \sum_{\mvec \in M_{\Delta}} q^{\frac{k}{2}\left(\mvec +\frac{1}{k}\zetavec |\mvec +\frac{1}{k}\zetavec  \right)}\\
& = q^{A}\cdot  \eta (k\tau )^{-n-1}\cdot  \theta_{\Delta} (\tau )
\end{align*}
where 
\[
A \quad = \quad \frac{-1}{24} + \frac{k(n+1)}{24} - \frac{1}{2k}(\zetavec
|\zetavec )
\]
and $\theta_{\Delta}(\tau )$ is the shifted theta function:
\[
\theta_{\Delta}(\tau ) = \sum_{\mvec \in M_{\Delta}} q^{\frac{k}{2}\left(\mvec +\frac{1}{k}\zetavec |\mvec +\frac{1}{k}\zetavec  \right)}
\]
where as throughout this paper we have identified $q=\exp\left(2\pi i\tau  \right)$.

In Appendix~\ref{sec: another strange formula}, we will prove the
following formula which for $\Delta =A_{n}$ coincides with the
``strange formula'' of Freudenthal and de Vries
\cite{freudenthal1969linear}:
\[
 \frac{k(n+1)-1}{24}  = \frac{(\zetavec |\zetavec )}{2k}.
\]
It follows that $A=0$ and we obtain the following:
\begin{lemma}\label{lem: local series as theta/eta} The local series
$Z_{\Delta}(q)$ is given by
\[
Z_{\Delta}(q) = \frac{\theta_{\Delta}(\tau )}{\eta (k\tau )^{n+1}}. 
\]
\end{lemma}


\section{The global series}\label{sec: the global series}

Recall that $p_{1},\dotsc ,p_{r}\in X/G$ are the singular points of
$X/G$ with corresponding stabilizer subgroups $G_{i}\subset G$ of
order $k_{i}$ and ADE type $\Delta_{i}$. Let $\{x_{i}^{1},\dotsc
,x_{i}^{k/k_{i}} \}$ be the orbit of $G$ in $X$ corresponding to the
point $p_{i}$ (recall that $k=|G|$).  We may stratify $\Hilb (X)^{G}$
according to the orbit types of subscheme as follows.

Let $Z\subset X$ be a $G$-invariant subscheme of length $nk$ whose
support lies on free orbits. Then $Z$ determines and is determined by
a length $n$ subscheme of 
\[
(X/G)^{o}  = X/G\setminus \{p_{1},\dotsc ,p_{r} \},
\]
i.e. a point
in $\Hilb^{n}((X/G)^{o})$.

On the other hand, suppose $Z\subset X$ is a $G$-invariant subscheme
of length $\frac{nk}{k_{i}}$ supported on the orbit
$\{x_{i}^{1},\dotsc ,x_{i}^{k/k_{i}} \}$. Then $Z$ determines and is
determined by the length $n$ component of $Z$ supported on a formal
neighborhood of one of the points, say $x_{i}^{1}$. Choosing a
$G_{i}$-equivariant isomorphism of the formal neighborhood of
$x_{i}^{1}$ in $X$ with the formal neighborhood of the origin in
$\CC^{2}$, we see that $Z$ determines and is determined by a point in
$\Hilb_{0}^{n}(\CC^{2})^{G_{i}}$, where $\Hilb_{0}^{n}(\CC^{2})\subset
\Hilb^{n}(\CC^{2})$ is the punctual Hilbert scheme parameterizing
subschemes supported on a formal neighborhood of the origin in
$\CC^{2}$.

By decomposing an arbitrary $G$-invariant subscheme into components of
the above types, we obtain a stratification of $\Hilb (X)^{G}$ into
strata which are given by products of $\Hilb ((X/G)^{o})$ and
$\Hilb_{0}(\CC^{2})^{G_{1}},\dotsc ,\Hilb_{0}(\CC^{2})^{G_{r}}$. Then
using the fact that Euler characteristic is additive under
stratifications and multiplicative under products, we arrive at the
following equation of generating functions:
\begin{align}\label{eqn: stratification formula for sum e(hilb(X)G)}
\nonumber\sum_{n=0}^{\infty} e\left(\Hilb^{n}(X)^{G} \right)\, q^{n}
=&\left(\sum_{n=0}^{\infty} e\left(\Hilb^{n}((X/G)^{o}) \right)\,
q^{kn} \right)\\
& \cdot \prod_{i=1}^{r}\left( \sum_{n=0}^{\infty}
e\left(\Hilb_{0}^{n}(\CC^{2})^{G_{i}} \right) \, q^{\frac{nk}{k_{i}}}
\right) .
\end{align}

As in the introduction, let $a = e(X/G)-r=e\left((X/G)^{o}
\right)$. Then by G\"ottsche's formula \cite{gottsche1990betti},
\begin{align*}
\sum_{n=0}^{\infty} e\left(\Hilb^{n}((X/G)^{0} \right) q^{kn} &=
\prod_{m=1}^{\infty} (1-q^{km})^{-a}\\
&= q^{\frac{ak}{24}} \cdot \eta (k\tau )^{-a}. 
\end{align*}

We also note that $e\left(\Hilb_{0}^{n}(\CC^{2})^{G_{i}}
\right)=e\left(\Hilb^{n}(\CC^{2})^{G_{i}} \right)$ since the natural
$\CC^{*}$ action on both $\Hilb_{0}^{n}(\CC^{2})^{G_{i}}$ and
$\Hilb^{n}(\CC^{2})^{G_{i}}$ have the same fixed points. Thus we may
write
\begin{align*}
\sum_{n=0}^{\infty} e\left(\Hilb_{0}^{n}(\CC^{2})^{G_{i}} \right) \,
q^{\frac{nk}{k_{i}}} &= \sum_{n=0}^{\infty} e\left(\Hilb^{n}(\CC^{2})^{G_{i}} \right) \,
q^{\frac{nk}{k_{i}}}\\
&= q^{\frac{k}{24k_{i}}} \cdot Z_{\Delta_{i}} \left(\frac{k\tau}{k_{i}}
\right) .
\end{align*}

Multiplying equation~\eqref{eqn: stratification formula for sum
e(hilb(X)G)} by $q^{-1}$ and substituting the above formulas, we find
that

\[
Z_{X,G}(q) = q^{-1 +\frac{ak}{24} + \sum \frac{k}{24k_{i}} } \cdot 
\eta (k\tau )^{-a}\cdot 
\prod_{i=1}^{r}Z_{\Delta_{i}}\left(\frac{k\tau}{k_{i}} \right) .
\]

From the following Euler characteristic calculation, we see that the exponent of $q$ in the above equation is zero:
\begin{align*}
24 = e(X) &= e\left(X-\cup_{i=1}^{r} \{x_{i}^{1},\dotsc
,x_{i}^{k/k_{i}} \} \right) + \sum_{i=1}^{r} \frac{k}{k_{i}} \\
&= k \cdot e\left((X/G)^{o} \right) + \sum_{i=1}^{r} \frac{k}{k_{i}} \\
&= k\cdot a + \sum_{i=1}^{r}\frac{k}{k_{i}} 
\end{align*}

This completes the proof of Theorem~\ref{thm: eta product formula for
Z}.  \qed


\section{Proof of Theorem \ref{thm: formula for local series (in
intro)}}\label{sec: proof of formula for local series}

\subsection{Proof of Theorem~\ref{thm: formula for local series (in
intro)} in the $A_{n}$ case.}\label{subsec: proof of An case of
local series}

We wish to prove
\[
Z_{A_{n}}(q) = \frac{1}{\eta (\tau )}
\]
which is equivalent to the statement
\[
\sum_{m=0}^{\infty} e\left(\Hilb^{m} (\CC^{2})^{\ZZ /(n+1)} \right)
\,q^{m} = \prod_{m=1}^{\infty} (1-q^{m})^{-1}.
\]
The action of $\ZZ /(n+1)$ on $\CC^{2}$ commutes with the action of
$\CC^{*}\times \CC^{*}$ on $\CC^{2}$ and consequently, the Euler
characteristics on the left hand side may be computed by counting
the $\CC^{*}\times \CC^{*}$-fixed subschemes, namely those given by
monomial ideals. Such subschemes of length $m$ have a well known
bijection with integer partitions of $m$, whose generating function is
given by the right hand side.\qed 

\subsection{Proof of Theorem~\ref{thm: formula for local series (in
intro)} in the $D_{n}$ and $E_{n}$ cases.}\label{subsec: proof of Dn
and En cases of local series}

Our proof of Theorem~\ref{thm: formula for local series (in intro)} in
the $D_{n}$ and $E_{n}$ cases uses a trick exploiting the derived
McKay correspondence between $\X =[\CC^{2}/\{\pm 1 \}]$ and
$Y=\operatorname{Tot}(K_{\PP^{1}})$. 

Let $G\subset SU(2)$ be a subgroup where the corresponding root system
$\Delta$ is of $D$ or $E$ type. Then $\{\pm 1 \}\subset G$ and let
$H\subset SO(3)$ be the quotient
\[
H=G/\{\pm 1 \}.
\]
The induced action of $H$ on $\PP^{1}\cong S^{2}$ is by
rotations. Indeed, $H$ is the symmetry group of a regular polyhedral
decomposition of $S^{2}$ which is given by the platonic solids in the
$E_{n}$ case and the decomposition into two hemispherical $(n-2)$-gons
in the $D_{n}$ case. $H$ is generated by rotations of order $\varp$,
$\varq$, $\varr$, obtained by rotating about the center of an edge, a
face, or a vertex respectively. $H$ has a group presentation:
\[
H=\{\left\langle a,b,c \right\rangle :\quad a^{\varp} = b^{\varq} =
c^{\varr} = abc=1 \}.
\]

Let $M=|H|$ be the order of $H$ and let $E,F,V$ be the number of
edges, faces, and vertices respectively. Then
\[
M=\varp E = \varq F = \varr V
\]
and since the stabilizer of an edge is always order 2 we have $\varp =2$
and so $M=2E$. Then since $F+V-E=2$ we find
\[
E+F+V = 2 + M
\]

We summarize this information below:

\begin{center}
\begin{tabular}{|c|c|c|c|c|}
\hline
Type &	$H$ &	$M$&	$(\varp ,\varq ,\varr )$ &	$(E,F,V)$\\ \hline \hline 
%$A_{n}$     & cyclic & $n+1$ & $(1,n+1,n+1)$& $(n+1,1,1)$	\\ \hline
$D_{n}$     & dihedral & $2n-2$ & $(2,n-2,2)$& $(n-1,2,n-1)$	\\ \hline
$E_{6}$     & tetrahedral & $12$ & $(2,3,3)$& $(6,4,4)$	\\ \hline
$E_{7}$     & octahedral & $24$ & $(2,3,4)$& $(12,8,6)$	\\ \hline
$E_{8}$     & icosohedral & $60$ & $(2,3,5)$& $(30,20,12)$	\\ \hline
\end{tabular}
\end{center}

\bigskip

Now let $\X$ be the stack quotient
\[
\X=[\CC^{2}/ \{\pm 1 \}]
\]
and let
\[
Y\cong \operatorname{Tot}(K_{\PP^{1}})
\]
be the minimal resolution of the singular space $X=\CC^{2}/\{\pm 1
\}$.

The stack quotient $[\PP^{1}/H]$ has three stacky points with
stabilizers of order $\varp ,\varq ,\varr$, and consequently the stack
quotient $[Y/H]$ has three orbifold points locally of the form
$[\CC^{2}/\ZZ_{\vara }]$ for $\vara \in \{\varp ,\varq ,\varr \}$.


We observe that 
\[
 [\CC^{2}/G ] \cong [\X /H ]
\]
and consequently
\[
\Hilb^{n}(\CC^{2})^{G}  \cong \Hilb^{n}(\X )^{H} .
\]
Recall from section \ref{sec: local partition functions} that $
\Hilb^{n}(\X )$ decomposes into components $\Hilb^{m_{0},m_{1}}(\X )$
with $n=m_{0}+m_{1}$ where the corresponding $\{\pm 1 \}$ invariant
subschemes $Z\subset \CC^{2}$ have the property that as a $\{\pm 1
\}$-representation, $H^{0}(\O_{Z})$ has $m_{0}$ copies of the trivial
representation and $m_{1}$ copies of the non-trivial representation.

We will prove in section~\ref{sec: McKay correspondence} that as a
consequence of the derived McKay correspondence between $\X$ and $Y$,
we have the following:
\begin{proposition}\label{prop: Hilb(X,m0,m1) = Hilb(Y,m0-(m0-m1)^2)}
$\Hilb^{m_{0},m_{1}}(\X )^{H}$ is deformation equivalent to and hence
diffeomorphic to $\Hilb^{m_{0}-(m_{0}-m_{1})^{2}}(Y)^{H}$. In
particular
\[
e\left(\Hilb^{m_{0},m_{1}}(\X )^{H} \right)
=e\left(\Hilb^{m_{0}-(m_{0}-m_{1})^{2}}(Y)^{H} \right). 
\]
\end{proposition}

Let
\[
j=m_{1}-m_{0}, \quad n=m_{0}-(m_{0}-m_{1})^{2} 
\]
so that
\[
m_{0}+m_{1} = 2n +j + 2j^{2}. 
\]
We then can compute:
\begin{align*}
q^{\frac{1}{24}} Z_{\Delta}(q) &= \sum_{m_{0},m_{1}= 0}^{\infty } 
e\left(\Hilb^{m_{0},m_{1}}(\X )^{H} \right) \, q^{m_{0}+m_{1}} \\
&= \sum_{j\in \ZZ}\,  \sum_{n=0}^{\infty} e\left(\Hilb^{n}(Y)^{H}
\right)\, q^{2n+j+2j^{2}}.
\end{align*}
For the root lattice of $A_{1}$, we have $M_{A_{1}}\cong \ZZ$,  $C_{A_{1}}=(2)$, and $k=2$,
$\zetavec =\frac{1}{2}$ so by definition
\begin{align*}
\theta_{A_{1}}(\tau ) &= \sum_{m\in \ZZ}
q^{\frac{2}{2}\left(m+\frac{1}{4}|m+\frac{1}{4} \right)} \\
&=\sum_{m\in \ZZ} q^{2(m+\frac{1}{4})^{2}} \\
&=q^{\frac{1}{8}} \sum_{j\in \ZZ} q^{2j^{2}+j}
\end{align*}
Substituting into the previous equation multiplied by
$q^{\frac{1}{8}}$ we find
\[
q^{\frac{1}{6}} Z_{\Delta}(q) = \theta_{A_{1}}(\tau ) \cdot
\sum_{n=0}^{\infty}  e\left(\Hilb^{n}(Y)^{H}
\right)\, q^{2n}.
\]

We can now compute the summation factor in the above equation by the
same method we used to compute the global series in section~\ref{sec:
the global series}. Here we utilize the fact that the singularities of
$Y/H$ are all of type $A$ and we have already proven our formula for
the local series in the $A_{n}$ case. Indeed, the quotient $[Y/H]$ has
three stacky points with stabilizers $\ZZ_{\varp}$, $\ZZ_{\varq}$, and
$\ZZ_{\varr}$ and the complement of those points $(Y/H)^{o}$ has Euler
characteristic $-1$. Proceeding then by the same argument we used in
section~\ref{sec: the global series} to get equation~\eqref{eqn:
stratification formula for sum e(hilb(X)G)}, we get
\begin{align*}
\sum_{n=0}^{\infty}  e\left(\Hilb^{n}(Y)^{H}\right)\, q^{2n}&=
\left(\sum_{n=0}^{\infty} e\left(\Hilb^{n}\left((Y/H)^{o} \right)
\right) q^{2Mn} \right)  \\
&\quad \quad  \cdot \prod_{\vara \in \{\varp ,\varq
,\varr  \}}\, \left( \sum_{n=0}^{\infty} e\left(\Hilb_{0}^{n}(\CC^{2})^{\ZZ_{\vara }} \right) q^{\frac{2Mn}{\vara }}  \right)\\
&=\prod_{m=1}^{\infty} \frac{\left(1-q^{2Mn} \right)}{\left(1-q^{\frac{2Mn}{\varp}} \right)\left(1-q^{\frac{2Mn}{\varq}} \right)\left(1-q^{\frac{2Mn}{\varr}} \right)}\\
&=\prod_{m=1}^{\infty} \frac{\left(1-q^{4En} \right)}{\left(1-q^{2En} \right)\left(1-q^{2Fn} \right)\left(1-q^{2Vn} \right)}\\
&=q^{\frac{1}{24}(-2E+2F+2V)}\cdot  \frac{\eta (4E\tau )}{\eta (2E\tau )\eta
(2F\tau )\eta (2V\tau )} \\
&=  \frac{q^{\frac{1}{6}}\,\eta (4E\tau )}{\eta (2E\tau )\eta
(2F\tau )\eta (2V\tau )}.
\end{align*}
Substituting into the previous equation and cancelling the factors of
$q^{\frac{1}{6}}$, we have thus proved
\[
Z_{\Delta}(q) = \theta_{A_{1}}(\tau )\cdot \frac{\eta (4E\tau )}{\eta
(2E\tau )\eta (2F\tau )\eta (2V\tau )}.
\]
Finally, by Lemma~\ref{lem: local series as theta/eta} and the $A_{1}$
case of Theorem~\ref{thm: formula for local series (in intro)} (which
we've already proved), we have that 
\[
\theta_{A_{1}}(\tau ) =\frac{\eta^{2}(2\tau )}{\eta (\tau )}
\]
which, when substituted into the above, completes the proof of
Theorem~\ref{thm: formula for local series (in intro)} in the general
case. \qed 



\section{The derived McKay correspondence and the proof of
Proposition~\ref{prop: Hilb(X,m0,m1) =
Hilb(Y,m0-(m0-m1)^2)}}\label{sec: McKay correspondence}

The derived McKay correspondence uses a Fourier-Mukai transform to
give an equivalence of derived categories \cite{BKR, Kapranov-Vasserot}:
\[
\FM :D^{b}(\X )\to D^{b}(Y).
\]
The induced map on the numerical $K$-groups
\[
K_{0}(\X ) \to K_{0}(Y)
\]
is well known to take
\begin{align*}
[\O_{\X}]& \longmapsto \quad [\O_{Y}] \\
[\O_{0}\otimes \rho_{0}] &\longmapsto  \quad [\O_{C}]\\
[\O_{0}\otimes \rho_{1}] &\longmapsto \, -[\O_{C}(-1)]
\end{align*}
where $\O_{0}$ is the skyscraper sheaf of the origin in $\CC^{2}$,
$\rho_{0}$ and $\rho_{1}$ are the trivial and non-trivial irreducible
representations of $\{\pm 1 \}$, and $C\subset Y$ is the exceptional
curve (see for example \cite{Gonzalez-Sprinberg-Verdier} or \cite{Kapranov-Vasserot}).

Let $\M^{m_{0},m_{1}}(\X )$ be the moduli stack of objects $F_{\bullet }$ in
$D^{b}(\X )$ having numerical $K$-theory class given by
\[
[F_{\bullet}] = [\O_{\X}] - m_{0}[\O_{0}\otimes \rho_{0}]  - m_{1}[\O_{0}\otimes \rho_{1}] .
\]

Then $\Hilb^{m_{0},m_{1}}(\X )$ may be regarded as the  open substack
of $\M^{m_{0},m_{1}}(\X )$ parameterizing ideal sheaves $I_{Z}$ viewed
as objects in $D^{b}(\X )$ supported in degree 0.

The derived McKay equivalence then induces an equivalence of stacks
\[
\M^{m_{0},m_{1}}(\X ) \cong  \M^{m_{0},m_{1}}(Y)
\]
where $ \M^{m_{0},m_{1}}(Y)$ is the moduli space of objects
$F_{\bullet}$ in $D^{b}(Y)$ having numerical $K$-theory class given by
\[
[F_{\bullet}] = [\O_{Y}] - m_{0} [\O_{C}]+ m_{1} [\O_{C}(-1)]. 
\]
Numerical $K$-theory on $Y$ is isomorphic to $H^{*}(Y)$ via the Chern
character. Then since
\begin{align*}
ch([\O_{Y}]) &= (1,0,0)\\
ch([\O_{C}]) &= (0,C,1)\\
ch([\O_{C}(-1)]) &= (0,C,0)
\end{align*}
we have
\[
ch( [\O_{Y}] - m_{0} [\O_{C}]+ m_{1} [\O_{C}(-1)]) = (1,(m_{1}-m_{0})C,-m_{0}).
\]
We switch notation so that $\M_{(a,b,c)}(Y)$ denotes the moduli stack
of objects in $D^{b}(Y)$ having Chern character $(a,b,c)\in
H^{*}(Y)$. Then we can rewrite the derived McKay equivalence as
\[
\M^{m_{0},m_{1}}(\X ) \cong \M_{(1,(m_{1}-m_{0})C,-m_{0})}(Y). 
\]
Tensoring by the line bundle
\[
L = \O_{Y}((m_{0}-m_{1})C)
\]
induces an equivalence
\[
\M_{(1,(m_{1}-m_{0})C,-m_{0})}(Y) \cong \M_{(1,0,-n)}(Y)
\]
where
\[
n=m_{0} -(m_{1}-m_{0})^{2}. 
\]
Indeed, this follows from
\[
ch(L) = \left(1,(m_{0}-m_{1})C,\tfrac{1}{2} (m_{0}-m_{1})^{2}C^{2} \right)
\]
and 
\begin{align*}
(1,(m_{1}-m_{0})C,&-m_{0}) \cdot  (1,(m_{0}-m_{1})C,\tfrac{1}{2}
(m_{0}-m_{1})^{2}C^{2})\\
 &= (1,0, - (m_{0}-m_{1})^{2}C^{2} +\tfrac{1}{2}
(m_{0}-m_{1})^{2}C^{2} -m_{0})\\
&= (1,0,(m_{0}-m_{1})^{2}-m_{0})
\end{align*}
where we have used $C^{2}=-2$.

Thus we have an equivalence of stacks
\[
\M^{m_{0},m_{1}}(\X ) \cong \M_{(1,0,-n)}(Y)
\]
which takes the open stack $\Hilb^{m_{0},m_{1}}(\X )$ isomorphically
onto an open substack of $\M_{(1,0,-n)} (Y)$. Noting that
$\Hilb^{n}(Y)$ is also an open substack of $\M_{(1,0,-n)} (Y)$ which
intersects the image of $\Hilb^{m_{1},m_{1}}(\X )$ non-trivially, we
find the equivalence induces a birational map
\[
\Hilb^{m_{0},m_{1}}(\X ) \dashrightarrow \Hilb^{n}(Y).
\]
Both sides of the above birational equivalence are smooth, holomorphic
symplectic varieties, and they are both symplectic resolutions of the
singular affine symplectic variety $\Sym^{n} (\CC^{2}/\{\pm 1
\})$. Consequently, $\Hilb^{m_{0},m_{1}}(\X )$ is deformation
equivalent to, and hence diffeomorphic to $\Hilb^{n}(Y)$. This
assertion follows from Nakajima \cite[Cor~4.2]{Nakajima1994Duke} by
viewing both Hilbert schemes as moduli spaces of quiver
representations of the $A_{1}$ Nakajima quiver variety having the same
dimension vector but using different stability conditions. Writing
these Hilbert schemes as Nakajima quiver varieties is discussed in
detail in \cite{Kuznetsov}.


The above constructions are compatible with the $H$ action. Indeed,
the derived McKay equivalence and tensoring with the line bundle $L$
both commute with the $H$ action and so we have an equivalence of
stacks: 
\[
\M^{m_{0},m_{1}}(\X )^{H} \cong \M_{(1,0,-n)}(Y)^{H}
\]
and a birational equivalence of holomorphic symplectic varieties:
\[
\Hilb^{m_{0},m_{1}}(\X )^{H} \dashrightarrow \Hilb^{n}(Y)^{H}.
\]
As before, both sides of the above birational equivalence can be
viewed as Nakajima quiver varieties (in this case for the Nakajima
quiver associated to the Dynkin diagram corresponding to $G$), and so
by \cite[Cor~4.2]{Nakajima1994Duke} are deformation equivalent. This
then completes the proof of Proposition~\ref{prop: Hilb(X,m0,m1) =
Hilb(Y,m0-(m0-m1)^2)}. \qed


\section{Proof of Theorem~\ref{thm: Formula for Zbir}}
\label{sec: proof of thm about Zbir}

Let $\Zcal  =[X/G]$ be the stack quotient of $X$ by $G$ and let $Y\to
X/G$ be the minimal resolution. %Let furthermore $p_{1},\dotsc ,p_{r}\in \Zcal$
%be the orbifold points with nontrivial stabilizers.
The Hilbert scheme of zero dimensional substacks of $\Zcal$ is
naturally identified with the $G$-fixed Hilbert scheme of $X$:
\[
\Hilb (\Zcal )\cong \Hilb (X)^{G}.
\]
Components of $\Hilb (\Zcal )$ are indexed by the numerical $K$-theory
class of $\O_{Z}$ for $Z\subset \Zcal$. The $K$-theory class of
$\O_{Z}$ can be written in a basis for $K$-theory as follows:
\[
[\O_{Z}] = n[\O_{p}] + \sum_{i=1}^{r} \sum_{j=1}^{n(i)}
m_{j}(i)[\O_{p_{i}}\otimes \rho_{j}(i)] 
\]
where $p\in \Zcal$ is a generic point and $p_{1},\dotsc ,p_{r}\in \Zcal$
are the orbifold points. The local group of $\Zcal$ at $p_{i}$ is
$G_{\Delta (i)}\subset SL_{2}\CC$  and has corresponding root system
$\Delta (i)$ of rank $n(i)$, and has irreducible representations
$\rho_{0}(i),\rho_{1}(i),\dotsc ,\rho_{n(i)}(i)$ where $\rho_{0}(i)$
is the trivial representation. 
We note that we do not need to include $ [\O_{p_{i}}\otimes
\rho_{0}(i)]$ in our basis for $K$-theory because of the following observation.
\begin{lemma} We have the $K$-theory relation
\[
[\O_{p}] = [\O_{p_{i}}\otimes \rho_{\reg}(i)]
\]
where $\rho_{\reg}(i)$ is the regular representation of $G_{\Delta
	(i)}$.\end{lemma}
%which explains why 

We abbreviate the data $\left\{m_{j}(i) \right\}$ appearing in the $K$-theory class above  by the symbol
$\mdata$ and we denote by 
\[
\Hilb^{n,\mdata}(\Zcal ) \subset \Hilb (\Zcal )
\]
the corresponding component. Let
\[
D_{\mdata} = \sum_{i=1}^{r} \sum_{j=1}^{n(i)} \, m_{j}(i)\, E_{j}(i)
\]
where
$E_{1}(i),\dotsc ,E_{n(i)}(i)$ are
the exceptional curves over $p_{i}$. We can organize the data $\mdata = \left\{m_{j}(i) \right\}$ into
$\mvec (i)\in M_{\Delta (i)}$, i.e.  the vectors in the root lattice
of $\Delta (i)$ having components $ m_{1}(i),\dotsc
,m_{n(i)}(i)$. Under this identification
\[
D_{\mdata}^{2} = -\sum_{i=1}^{r} \left(\mvec (i)|\mvec (i)
\right)_{\Delta (i)} 
\]
since the intersection form of the exceptional curves over $p_{i}$ is
the negative of the corresponding Cartan matrix $C_{\Delta (i)}$. 

%Let
%\[
%d=n+\half D_{\mdata}^{2} = n-\half \sum_{i=1}^{r} \left(\mvec (i)|\mvec (i)
%\right)_{\Delta (i)} .
%\]

\begin{proposition}\label{prop: Hilb(Z) is birational to Hilb(Y)}
	$\Hilb^{n,\mdata}(\Zcal)$ is birational to $\Hilb^{n+\half
		D_{\mdata}^{2}}(Y)$. 
\end{proposition}

%\begin{lemma}
%A sheaf with $K$-theory $n,\mdata$
%\end{lemma}
\begin{lemma}
\label{lem:rigidsh}
	\begin{enumerate}
		\item Let $M$ be a finite ADE type root lattice and $G \subset SL_{2}\CC$ the corresponding finite subgroup. To any element $\mvec \in M$ there is a unique rigid $G$-invariant subscheme of $\mathbb{C}^2$ with $K$-theory class 
		$\half (\mvec|\mvec)[\O_{p}] + \sum_{j=1}^{n}
		m_{j}[\O_{0}\otimes \rho_{j}]$ where $p \in [\mathbb{C}^2/G]$ is a generic point.
		%The (graded) colength of the ideal of this subscheme is $\mathbf{m}+\half \delta( \mathbf{m}| \mathbf{m})$.
		\item For every datum $\mdata$ there is a unique rigid substack $Z_{\mdata} \subset \mathcal{Z}$ with $K$-theory class \begin{gather*}\sum_{i=1}^r\frac{(\mvec(i)|\mvec(i))_{\Delta (i)}}{2}[\O_{p}] + \sum_{j=1}^{n}
		m_{j}(i)[\O_{p_i}\otimes \rho_{j}(i)]\\=-\frac{
		D_{\mdata}^{2}}{2}[\O_{p}] + \sum_{j=1}^{n}
		m_{j}(i)[\O_{p_i}\otimes \rho_{j}(i)]
		\end{gather*} where $p \in \Zcal$ is a generic point.
%This subscheme has $\mathrm{colength}(\mathcal{I}_{Z_{\mdata}} \subset \mathcal{O}_{\Zcal})=\mdata -\half |G| D_{\mdata}^{2}$.
	\end{enumerate}
\end{lemma}
\begin{proof}
	Part (2) is implied by Part (1) since we can take the union of the rigid subschemes supported at the orbifold points $p_1,\dots,p_r \in \mathcal{Z}$. So we need only prove the local case.
	
	Nakajima has shown in \cite[Section 2]{nakajima2002geometric} that the $G$-fixed Hilbert scheme decomposes into smooth quiver varieties
	$\mathrm{Hilb}(\mathbb{C}^2)^G = \bigsqcup_{\mathbf{v}} \mathcal{M}(\vvec,\wvec)$ associated with the corresponding affine Dynkin quiver equipped with a fixed framing vector $\wvec=(1,0,\dots,0)$.
	Due to Theorem \ref{thm: Zorbifold formula} a sheaf whose $K$-theory class has $\{\rho_1,\dots, \rho_n\}$-part equal to $\mvec$ can only have
	a total class of the form
	$\mvec+\left( \frac{( \mvec| \mvec) }{2}+l \right)\delta $
	where $\mvec$ is interpreted as $\sum_{j=1}^{n}
	m_{j}[\O_{0}\otimes \rho_{j}]$, $\delta=[\O_{p}]$ and $l$ is a nonnegative integer. The particular case with $l=0$ contains the sheaves with the required class $\mvec+\half( \mvec| \mvec)\delta $.
	
	%is in the component of $\mathrm{Hilb}(\mathbb{C}^2)^G$ whose dimension vector is
	%=\mathbf{m}+\delta \left( \frac{( \mathbf{m}| \mathbf{m}) }{2} \right)
	
	%\[ \mathbf{v}=\mathbf{m}+\delta \left( \frac{( \mathbf{m}| \mathbf{m}) }{2} \right)= \left(\frac{a_0 (\mathbf{m}| \mathbf{m}) }{2},m_1+\frac{a_1 ( \mathbf{m}| \mathbf{m} ) }{2},\dots,m_n+\frac{a_n( \mathbf{m}| \mathbf{m} ) }{2} \right) \]
	%where $\delta=\sum_i\alpha_i$.
	By 
	\cite[(2.6)]{nakajima1994instantons}
	the dimension of $\mathcal{M}(\vvec,\wvec)$ in general is
	\[ 2\wvec \cdot \vvec - \langle \vvec, \vvec \rangle.  \]
	Here $\langle \cdot, \cdot \rangle$ is the inner product with respect to the affine invariant form and $\vvec,\wvec$ are elements in the affine root lattice.
	In our case this reads as
	\begin{align*} 2v_0 - \langle \vvec, \vvec \rangle & = \langle \mvec, \mvec \rangle - \left\langle \mvec+\frac{( \mvec| \mvec )\delta}{2}, \mvec+\frac{( \mvec| \mvec )\delta}{2} \right\rangle \\
	& =\langle \mvec, \mvec \rangle - \langle \mvec, \mvec \rangle - ( \mvec| \mvec) \langle \mvec, \delta \rangle  -\frac{1}{4} ( \mvec| \mvec)\langle \delta, \delta \rangle =
	0  \end{align*}
	where we have used that $\langle M, \delta \rangle = 0$ and $\langle \delta, \delta \rangle = 0$. Hence, any sheaf in the component with $\vvec=\mvec+\half( \mvec| \mvec)\delta$ is rigid.
	
	The formula in Theorem \ref{thm: Zorbifold formula} also implies that the above components with $l=0$ have Euler characteristic 1. This is because the numerator is a series with coefficient 1, and the expansion of the denominator always increases $l$ except for the starting term 1.
	
	%I think this can be read-off again from the formula of the generating
	%series (Theorem 2.1). Suppose first that the denominators are all 1.
	%Then the numerator counts the Euler characteristic of the required
	%cell, which is 1. If we take into account the denominators, they, when
	%expanded, can only increase the content coming from the numerator. So
	%for each root the appropriate component has Euler characteristic at
	%least 1. More mathematically, the generating series has only
	%nonnegative integer coefficients and the component corresponding to
	%each root appears with coefficient at least 1.
\end{proof}
\begin{proof}[Proof of Proposition \ref{prop: Hilb(Z) is birational to Hilb(Y)}]
Let $U\subset \Zcal \setminus \{p_{1},\dotsc ,p_{r} \}$ be
the  Zariski open part with trivial stabilizers. %This stack is equivalent to the scheme $W$.
Let $V \subset Y$ be the complement of the exceptional divisors.
Let furthermore $Z_{\mdata} \subset \mathcal{Z}$ be the rigid substack corresponding to the $K$-theory datum $\mdata$ provided by Lemma \ref{lem:rigidsh}. 
The Zariski open substack of $\mathrm{Hilb}^{n,\mdata}(\Zcal)$ parameterizing substacks of $\Zcal$ of the form $P \cup Z_{\mdata}$ where $P$ is a colength $n+\half
D_{\mdata}^{2}$ subscheme of $U$ is isomorphic to $\mathrm{Hilb}^{n+\half
D_{\mdata}^{2}}(U)$. This is because  $Z_{\mdata}$ was rigid and it had the $K$-theory class $-\half
D_{\mdata}^{2}[\O_{p}] + \sum_{j=1}^{n}
m_{j}(i)[\O_{p_i}\otimes \rho_{j}(i)]$. On the other hand, the Zariski open subset of $\mathrm{Hilb}^{n+\half
D_{\mdata}^{2}}(Y)$ parameterizing subschemes supported on $V\subset Y$ is isomorphic to $\mathrm{Hilb}^{n+\half
D_{\mdata}^{2}}(V)$. Finally, $\mathrm{Hilb}^{n+\half
D_{\mdata}^{2}}(U)\cong \mathrm{Hilb}^{n+\half
D_{\mdata}^{2}}(V)$ since $U$ and $V$ are canonically isomorphic.
\end{proof}





\begin{comment}
Our approach to proving
theorem~\ref{thm: Formula for Zbir} follows a similar idea as our
proof of theorem~\ref{thm: formula for local series (in intro)} in
Section~\ref{sec: proof of formula for local series} . We exploit the
derived McKay equivalence:
\[
\FM :D^{b}(\Zcal) \to D^{b}(Y).
\]



As a consequence of the above equivalence, we get the following
\begin{proposition}\label{prop: Hilb(Z) is birational to Hilb(Y)}
$\Hilb^{n,\mdata}(\Zcal)$ is birational to $\Hilb^{n+\half
D_{\mdata}^{2}}(Y)$. 
\end{proposition}
\begin{proof}
	
Since 
\[
\Hilb^{n,\mdata}(\Zcal )\subset \M^{\alpha}(\Zcal)\quad \text{ and } \quad \Hilb^{n+\half D_{\mdata}^{2}}(Y)\subset \M_{(1,0,-n-\half D_{\mdata}^{2})}(Y)
\]
are open substacks, we need only show that the image of
$\Hilb^{n,\mdata}(\Zcal )$ under the equivalence $\FM \circ
(L_{\mdata}\otimes -)$ intersects $\Hilb^{n+\half D_{\mdata}^{2}}(Y)$
non-trivially. We can even suppose that $n=0$, because of the motivic property of the Hilbert scheme.
Let $\mathcal{I}$ be the ideal sheaf of the rigid substack supported at the stacky points of $\Zcal$ whose K-theory class is $0,\mdata$.
	

Since 
\[
\Hilb^{n,\mdata}(\Zcal )\subset \M^{\alpha}(\Zcal)\quad \text{ and } \quad \Hilb^{n+\half D_{\mdata}^{2}}(Y)\subset \M_{(1,0,-n-\half D_{\mdata}^{2})}(Y)
\]
are open substacks, we need only show that the image of
$\Hilb^{n,\mdata}(\Zcal )$ under the equivalence $\FM \circ
(L_{\mdata}\otimes -)$ intersects $\Hilb^{n+\half D_{\mdata}^{2}}(Y)$
non-trivially. To this end, let $l=n+\half D_{\mdata}^{2}$, let
$q_{1},\dotsc ,q_{l}\in Y$ be distinct points disjoint from the
exceptional curves $E_{j}(i)$, and let ${r}_{1},\dotsc ,r_{l}\in
\Zcal$ denote the corresponding (non-stacky) points in $\Zcal$. We
show that $\FM ^{-1}(I_{\{q_{1},\dotsc ,q_{l} \}}\otimes
L_{\mdata}^{-1})$ is the ideal sheaf of a substack $Z\subset \Zcal$
where $Z=\{r_{1},\dotsc ,r_{l} \}\cup Z_{\mdata}$ where is a certain
rigid substack supported at the stacky points of $\Zcal$.

\bigskip

PROOF NEEDS TO BE FINISHED!

\bigskip


\end{proof}
\end{comment}


With Proposition~\ref{prop: Hilb(Z) is birational to Hilb(Y)}, we can now prove Theorem~\ref{thm: Formula for
Zbir}.  Using the identification \[\Hilb(X)^{G} = \Hilb (\Zcal )\]
and identifying discrete parameters we get
\begin{align*}
Z^{\bir}_{X,G}(q)& = \sum_{a=0}^{\infty} \left[\Hilb^{a}(X)^{G}
\right]_{\bir} \, \, q^{a-1} \\
& = \sum_{n,\mdata} \Hilb^{n,\mdata}(\Zcal )\,  q^{D(n,\mdata )-1}
\end{align*}
where
\[
D(n,\mdata ) = kn +\sum_{i=1}^{r}\frac{k}{k_{i}}\sum_{j=1}^{n(i)}
m_{j}(i) d_{j}(i).  
\]
\begin{comment}
We can organize the data $\mdata = \left\{m_{j}(i) \right\}$ into
$\mvec (i)\in M_{\Delta (i)}$, i.e.  the vectors in the root lattice
of $\Delta (i)$ having components $ m_{1}(i),\dotsc
,m_{n(i)}(i)$. Under this identification, we see that
\[
D_{\mdata}^{2} = -\sum_{i=1}^{r} \left(\mvec (i)|\mvec (i)
\right)_{\Delta (i)} 
\]
since the intersection form of the exceptional curves over $p_{i}$ is
the negative of the corresponding Cartan matrix $C_{\Delta (i)}$. 
\end{comment}
Let
\[
d=n+\half D_{\mdata}^{2} = n-\half \sum_{i=1}^{r} \left(\mvec (i)|\mvec (i)
\right)_{\Delta (i)} .
\]
Then
\[
Z^{\bir}_{X,G}(q) = \sum_{d=0}\left[\Hilb^{d}(Y) \right]_{\bir}
\prod_{i=1}^{r}\sum_{\mvec (i)\in M_{\Delta (i)}} q^{D(n,\mdata )-1} 
\]
with
\begin{align*}
D(n,\mdata )-1& = -1 + k\left(d+\half \sum_{i=1}^{r} \left(\mvec (i)|\mvec (i)
\right)_{\Delta (i)}  \right)
+\sum_{i=1}^{r}\frac{k}{k_{i}}\sum_{j=1}^{n(i)} m_{j}(i) d_{j}(i) \\
&=kd-1 +\frac{k}{2}\sum_{i=1}^{r} \left\{  \left(\mvec (i)|\mvec (i)
\right)_{\Delta (i)} +\frac{2}{k_{i}}  \left(\mvec (i)|\zetavec (i)
\right)_{\Delta (i)} \right\}
\end{align*}
where $\zetavec (i) \in M_{\Delta (i)}\otimes \QQ $ is as in
Section~\ref{sec: local partition functions}. 

Completing the square and using the formula
\[
\frac{1}{k_{i}^{2}}  \left(\zetavec (i)|\zetavec (i)
\right)_{\Delta (i)} = \frac{2}{k_{i}}\left(\frac{k_{i}(n(i)+1)-1}{24} \right),
\]
which follows from Lemma~\ref{lem: another strange formula}, we get 
\begin{gather*}
D(n,\mdata )-1 = kd-1 \\  - \sum_{i=1}^{r}
\frac{k}{k_{i}}\left(\frac{k_{i}(n(i)+1)-1}{24} \right)
+\frac{k}{2}\sum_{i=1}^{r} \left(\mvec (i)+\frac{1}{k_{i}}\zetavec
(i)\,\,\Big{|}\,\,\mvec (i)+\frac{1}{k_{i}}\zetavec (i) \right)_{\Delta (i)} . 
\end{gather*}
It then follows that
\begin{gather*}
Z^{\bir}_{X,G}(q) = \\ q^{A} \sum_{d=0}^{\infty} \left[\Hilb^{d}(Y)
\right]_{\bir} \,\, q^{kd-k} \,\, \prod_{i=1}^{r} \sum_{\mvec (i)\in
M_{\Delta (i)}} q^{\frac{k}{2} \left(\mvec (i)+\frac{1}{k_{i}}\zetavec
(i)\,\,\Big{|}\,\,\mvec (i)+\frac{1}{k_{i}}\zetavec (i) \right)_{\Delta (i)} } 
\end{gather*}
where
\[
A = k-1 -\frac{k}{24} \sum_{i=1}^{r} \left(n(i) + 1 - \frac{1}{k_{i}}
\right). 
\]
Since 
\begin{align*}
24 = e(Y) &= e\left(X/G - \{p_{1},\dotsc ,p_{r} \} \right) +
\sum_{i=1}^{r} (n(i)+1)\\
&=\frac{1}{k} \left(24 - \sum_{i=1}^{r} \frac{k}{k_{i}} \right)
+\sum_{i=1}^{r} (n(i)+1)\\
&= \frac{24}{k} + \sum_{i=1}^{r} \left(n(i)+1 -\frac{1}{k_{i}} \right)
\end{align*}
we see that $A=0$.

Thus we have
\begin{align*}
Z^{\bir}_{X,G} (q) &= Z^{\bir}_{Y}(q^{k}) \prod_{i=1}^{r}
\theta_{\Delta (i)} \left(\frac{k\tau}{k_{i}} \right) \\
&= Z^{\bir}_{Y}(q^{k}) \prod_{i=1}^{r}
Z_{\Delta (i)} \left(\frac{k\tau}{k_{i}} \right)\eta (k\tau )^{n(i)+1} \\
&=Z_{Y}^{\bir} (q^{k})\cdot  \eta (k\tau )^{B}\cdot  Z_{X,G}(q) 
\end{align*}
where we've used Theorem~\ref{thm: eta product for theta function},
Theorem~\ref{thm: eta product formula for Z}, and we've set
\[
B=\frac{24}{k} + \sum_{i=1}^{r} \left(n(i)+1 -\frac{1}{k_{i}}
\right). 
\]
The previous equation which showed that $A=0$ also shows that
$B=24$. Then since $\Delta (\tau ) = \eta (\tau )^{24}$, we see that
Theorem~\ref{thm: Formula for Zbir} follows. \qed

\begin{remark}
	It is not needed in our proof, but it is possible to show that the birational equivalence of Proposition \ref{prop: Hilb(Z) is birational to Hilb(Y)} is realized by the derived McKay equivalence:
	\[
	\FM :D^{b}(\Zcal) \to D^{b}(Y).
	\]
	We
	may regard $\Hilb^{n,\mdata}(\Zcal )$ as parameterizing ideal
	sheaves and as such, it is given as an open substack
	\[
	\Hilb^{n,\mdata}(\Zcal )\subset \M^{\alpha}(\Zcal ) 
	\]
	of the moduli stack of objects in $D^{b}(\Zcal )$ having $K$-theory
	class
	\[
	\alpha = [\O_{\Zcal}] - n[\O_{p}] - \sum_{i=1}^{r} \sum_{j=1}^{n(i)}
	m_{j}(i)[\O_{p_{i}}\otimes \rho_{j}(i)] .
	\]
	The derived McKay equivalence then induces a stack equivalence
	\[
	\M^{\alpha} (\Zcal )\cong \M^{\FM (\alpha )}(Y).
	\]
	We identify numerical $K$-theory on $Y$ with cohomology via the Chern
	character and index the moduli stack of objects on $Y$ by their Chern
	character so that our equivalence reads
	\[
	\M^{\alpha}(\Zcal )\cong  \M_{\beta}(Y)
	\]
	where $\beta =ch(\FM (\alpha ))$. The derived McKay equivalence induces the
	following map in $K$-theory:
	\begin{align*}
	[\O_{\Zcal }] & \quad \longmapsto \quad  [\O_{Y}]\\
	[\O_{p }] & \quad \longmapsto \quad  [\O_{p}]\\
	[\O_{p_{i}}\otimes \rho_{j}(i)] & \quad \longmapsto \quad  -[\O_{E_{j}(i)}(-1)],\quad j=1,\dotsc, n(i) \\
	[\O_{p_{i}}\otimes \rho_{0}(i)] & \quad \longmapsto \quad  \sum_{j=1}^{\infty }d_{j}(i)[\O_{E_{j}(i)}]\\
	\end{align*}
	where $p\in Y$ is a generic point, $E_{1}(i),\dotsc ,E_{n(i)}(i)$ are
	the exceptional curves over $p_{i}$, and 
	\[
	d_{j}(i) = \dim \rho_{j}(i).
	\]
	The Chern characters are given by
	\begin{align*}
	ch(\O_{p}) & = (0,0,1) \\
	ch(\O_{Y}) & = (1,0,0) \\
	ch(\O_{E_{j}(i)}(-1)) & = (0,E_{j}(i),0)
	\end{align*}
	where the right hand side is in $H^{0}(Y)\oplus H^{2}(Y)\oplus
	H^{4}(Y)$. The fact that $ch_{2}(\O_{E_{j}(i)}(-1)) = 0$ follows from
	Hirzebruch-Riemann-Roch and the fact that $\chi
	(\O_{E_{j}(i)}(-1))=0$.
	We then see that 
	\[
	\beta  = ch(\FM (\alpha )) = (1, D_{\mdata},-n)
	\]
	where
	\[
	D_{\mdata} = \sum_{i=1}^{r} \sum_{j=1}^{n(i)} \, m_{j}(i)\, E_{j}(i).
	\]
	Tensoring by $L_{\mdata} = \O (-D_{\mdata})$ induces an equivalence of categories and
	thus the composition
	\[
	\begin{tikzcd}[column sep = large]
	\M^{\alpha}(\Zcal ) \arrow[r,"{\FM }"]&
	\M_{\beta}(Y) \arrow[r,"{L_{\mdata}\otimes -}"] &
	\M_{\beta \cdot (1,-D_{\mdata},\half D_{\mdata}^{2})}(Y)
	\end{tikzcd}
	\] 
	induces an equivalence
	\[
	\M^{\alpha}(\Zcal )\cong \M_{(1,0,-n-\half D_{\mdata}^{2})}(Y).
	\]
	Since our moduli spaces
	\[
	\Hilb^{n,\mdata}(\Zcal )\subset \M^{\alpha}(\Zcal)\quad \text{ and } \quad \Hilb^{n+\half D_{\mdata}^{2}}(Y)\subset \M_{(1,0,-n-\half D_{\mdata}^{2})}(Y)
	\]
	are open substacks, the image of
	$\Hilb^{n,\mdata}(\Zcal )$ under the equivalence $\FM \circ
	(L_{\mdata}\otimes -)$ intersects $\Hilb^{n+\half D_{\mdata}^{2}}(Y)$ nontrivially. Hence, $\FM \circ
	(L_{\mdata}\otimes -)$ induces an isomorphism on open subsets of the two Hilbert schemes. 
\end{remark}



\appendix
\section{Another Strange Formula}\label{sec: another strange
formula}

We recall the notation from Section~\ref{sec: local partition
functions}. Let $\Delta$ be an ADE root system of rank $n$. Let
$\alpha_{1},\dotsc \alpha_{n}$ be a system of positive simple roots
and let
\[
\dvec  = \sum_{i=1}^{n} d_{i} \alpha_{i}
\]
be the largest root. Let $(\cdot |\cdot )$ be the Weyl invariant
bilinear form with $(\alpha_{i}|\alpha_{i})=2$ and let $\zetavec$ be
the dual vector to $\dvec$ in the sense that
\begin{equation}\label{eqn: d = sum (zeta|ai)ai}
\sum_{i=1}^{n} (\zetavec |\alpha_{i}) \alpha_{i} = \dvec .
\end{equation}
Let 
\begin{equation}\label{eqn: k=1+(zeta|d)}
k=1+\sum_{i=1}^{n}d_{i}^{2} =  1+(\zetavec |\dvec ). 
\end{equation}

The identity of the following lemma coincides with Freudenthal and de Vries's
``strange formula'' when $\Delta$ is $A_{n}$. 
\begin{lemma}\label{lem: another strange formula}
Let $k$, $n$, and $\zetavec$ be as above. Then,
\[
\frac{k(n+1)-1}{24} =\frac{(\zetavec |\zetavec )}{2k} . 
\]
\end{lemma}
\begin{proof}
\textbf{The case of $\Delta =A_{n}$:}
For any ADE root system we have $(\rhovec |\alpha )=1$ for all positive
roots where $\rhovec =\half \sum_{\alpha \in R^{+}}\alpha$ is half the
sum of the positive roots. Since for $A_{n}$, $d_{i}=1$, it follows
from equation~\eqref{eqn: d = sum (zeta|ai)ai} that $\zetavec
=\rhovec$, and it follows from equation~\eqref{eqn: k=1+(zeta|d)}
that $k=n+1=h$ is the Coxeter number. The lemma is then
\[
\frac{(n+1)^{2}-1}{24} = \frac{(\rhovec |\rhovec )}{2h}.
\]
Since the Lie algebra associated to $A_{n}$, namely $\mathfrak{sl}_{n+1}$,
has dimension $(n+1)^{2}-1$ and the Killing form satisfies $\kappa
(\cdot ,\cdot )=\frac{1}{2h}(\cdot |\cdot )$, the lemma may be
rewritten as
\[
\frac{\dim \mathfrak{sl}_{n+1}}{24} =\kappa (\rhovec ,\rhovec )
\]
which is Freudenthal and de Vries's ``strange formula''
\cite[47.11]{freudenthal1969linear}.

\bigskip
\textbf{The case of $\Delta =D_{n}$:} Let $e_{1},\dotsc ,e_{n}$ be the
standard orthonormal basis of $\RR^{n}$. Then the collection $\{\pm
e_{i} \pm e_{j}, i<j \}$ is a $D_{n}$ root system and we may take 
\[
\alpha_{i} = \begin{cases}
e_{i}-e_{i+1},& i=1,\dotsc ,n-1\\
e_{n-1}+e_{n},&i=n
\end{cases}
\]
as a system of simple positive roots. Then the fundamental weights
$\omega_{i}$, which are defined by the condition $(\omega_{i}
|\alpha_{j}) = \delta_{ij} $, are given by \cite[Appendix C]{Knapp}
\[
\omega_{i} = \begin{cases}
e_{1}+\dotsb +e_{i},& i\leq n-2\\
\half (e_{1}+\dotsb +e_{n-1}-e_{n}),& i=n-1\\
\half (e_{1}+\dotsb +e_{n-1}+e_{n}),& i=n.
\end{cases}
\]
Then since
\[
d_{i}=\begin{cases}
1&i=1,n-1,n,\\
2&i=2,\dotsc , n-2,
\end{cases}
\]
we have
\begin{align*}
\zetavec & = \omega_{1}+2\omega_{2}+2\omega_{3}+\dotsb +2\omega_{n-2}+\omega_{n-1}+\omega_{n}\\
&= 2(n-2) e_{1} + \sum_{i=2}^{n-2}\left(2(n-1-i)+1 \right)\, e_{i}  +e_{n-1}
\end{align*}
and so
\begin{align*}
(\zetavec |\zetavec ) &= 4(n-2)^{2} + 1 + \sum_{i=2}^{n-2} \left(2(n-1-i)+1 \right)^{2}\\
&= \frac{4}{3}n^{3}-4n^{2}-\frac{1}{3}n+6.
\end{align*}
Finally since $k=1+\sum_{i=1}^{n}d_{i} = 4(n-2)$ the lemma becomes
\[
\frac{4(n-2)(n+1)-1}{24} = \frac{ \frac{4}{3}n^{3}-4n^{2}-\frac{1}{3}n+6}{8(n-2)}
\]
which is readily verified.

\textbf{The case of $\Delta =E_{6},E_{7},E_{8}$:} These three
individual cases are easily checked one by one.

    
\end{proof}



\clearpage

\section{Table of eta products}\label{app:tableeta}
\renewcommand{\arraystretch}{1.5}



The following table provides the list of the modular forms
$Z_{X,G}^{-1}$, expressed as eta products, for each of the 82 possible
symplectic actions of a group $G$ on a $K3$ surface $X$. Our numbering
matches Xiao's \cite{xiao1996galois} whose table we refer to for a
description of each group.
\begin{longtable}{|l|l|l|l|l|}
  \hline
$\# $ & $|G|$ & Singularities of $X/G$&  The modular form $Z_{X,G}^{-1}$ & Weight \\ 
  \hline
0 & 1 &  & $ \eta \left( \tau \right)   ^{24}$ & 12 \\ 
  1 & 2 & $8 A_{1}$ & $ \eta \left( 2\tau \right)   ^{8}  \eta \left( \tau \right)   ^{8}$ & 8 \\ 
  2 & 3 & $6 A_{2}$ & $ \eta \left( 3\tau \right)   ^{6}  \eta \left( \tau \right)   ^{6}$ & 6 \\ 
  3 & 4 & $12 A_{1}$ & $ \eta \left( 2\tau \right)   ^{12}$ & 6 \\ 
  4 & 4 & $2 A_{1} + 4 A_{3}$ & $ \eta \left( 4\tau \right)   ^{4}  \eta \left( 2\tau \right)   ^{2}  \eta \left( \tau \right)   ^{4}$ & 5 \\ 
  5 & 5 & $4 A_{4}$ & $ \eta \left( 5\tau \right)   ^{4}  \eta \left( \tau \right)   ^{4}$ & 4 \\ 
  6 & 6 & $8 A_{1} + 3 A_{2}$ & ${\frac {  \eta \left( 3\tau \right)   ^{8}  \eta \left( 2\tau \right)   ^{3}}{\eta \left( 6\tau \right) }}$ & 5 \\ 
  7 & 6 & $2 A_{1} + 2 A_{2} + 2 A_{5}$ & $ \eta \left( 6\tau \right)   ^{2}  \eta \left( 3\tau \right)   ^{2}  \eta \left( 2\tau \right)   ^{2} \mbox{}  \eta \left( \tau \right)   ^{2}$ & 4 \\ 
  8 & 7 & $3 A_{6}$ & $ \eta \left( 7\tau \right)   ^{3}  \eta \left( \tau \right)   ^{3}$ & 3 \\ 
  9 & 8 & $14 A_{1}$ & ${\frac {  \eta \left( 4\tau \right)   ^{14}}{  \eta \left( 8\tau \right)   ^{4}}}$ & 5 \\ 
  10 & 8 & $9 A_{1} + 2 A_{3}$ & ${\frac {  \eta \left( 4\tau \right)   ^{9}  \eta \left( 2\tau \right)   ^{2}}{  \eta \left( 8\tau \right)   ^{2}}}$ & 9/2 \\ 
  11 & 8 & $4 A_{1} + 4 A_{3}$ & $ \eta \left( 4\tau \right)   ^{4}  \eta \left( 2\tau \right)   ^{4}$ & 4 \\ 
  12 & 8 & $3 A_{3} + 2 D_{4}$ & ${\frac {  \eta \left( \tau \right)   ^{2}  \eta \left( 4\tau \right)   ^{6}}{\eta \left( 2\tau \right) }}$ & 7/2 \\ 
  13 & 8 & $ A_{1} + 4 D_{4}$ & ${\frac {  \eta \left( 4\tau \right)   ^{13}  \eta \left( \tau \right)   ^{4}}{  \eta \left( 8\tau \right)   ^{2} \mbox{}  \eta \left( 2\tau \right)   ^{8}}}$ & 7/2 \\ 
  14 & 8 & $ A_{1} +  A_{3} + 2 A_{7}$ & $ \eta \left( 8\tau \right)   ^{2}\eta \left( 4\tau \right) \eta \left( 2\tau \right)   \eta \left( \tau \right)   ^{2}$ & 3 \\ 
  15 & 9 & $8 A_{2}$ & $ \eta \left( 3\tau \right)   ^{8}$ & 4 \\ 
  16 & 10 & $8 A_{1} + 2 A_{4}$ & ${\frac {  \eta \left( 5\tau \right)   ^{8}  \eta \left( 2\tau \right)   ^{2}}{  \eta \left( 10\tau \right)   ^{2}}}$ & 4 \\ 
  17 & 12 & $4 A_{1} + 6 A_{2}$ & ${\frac {  \eta \left( 6\tau \right)   ^{4}  \eta \left( 4\tau \right)   ^{6}}{  \eta \left( 12\tau \right)   ^{2}}}$ & 4 \\ 
  18 & 12 & $9 A_{1} +  A_{2} +  A_{5}$ & ${\frac {  \eta \left( 6\tau \right)   ^{9}\eta \left( 4\tau \right) \eta \left( 2\tau \right) }{  \eta \left( 12\tau \right)   ^{3}}}$ & 4 \\ 
  19 & 12 & $3 A_{1} + 3 A_{5}$ & $ \eta \left( 6\tau \right)   ^{3}  \eta \left( 2\tau \right)   ^{3}$ & 3 \\ 
  20 & 12 & $ A_{2} + 2 A_{3} + 2 D_{5}$ & ${\frac {  \eta \left( 4\tau \right)   ^{3}  \eta \left( 3\tau \right)   ^{2}  \eta \left( \tau \right)   ^{2} \mbox{}  \eta \left( 6\tau \right)   ^{4}}{\eta \left( 12\tau \right)   \eta \left( 2\tau \right)   ^{4}}}$ & 3 \\ 
  21 & 16 & $15 A_{1}$ & ${\frac {  \eta \left( 8\tau \right)   ^{15}}{  \eta \left( 16\tau \right)   ^{6}}}$ & 9/2 \\ 
  22 & 16 & $10 A_{1} + 2 A_{3}$ & ${\frac {  \eta \left( 8\tau \right)   ^{10}  \eta \left( 4\tau \right)   ^{2}}{  \eta \left( 16\tau \right)   ^{4}}}$ & 4 \\ 
  23 & 16 & $5 A_{1} + 4 A_{3}$ & ${\frac {  \eta \left( 8\tau \right)   ^{5}  \eta \left( 4\tau \right)   ^{4}}{  \eta \left( 16\tau \right)   ^{2}}}$ & 7/2 \\ 
  24 & 16 & $6 A_{1} +  A_{3} + 2 D_{4}$ & ${\frac {  \eta \left( 8\tau \right)   ^{12}  \eta \left( 2\tau \right)   ^{2}}{  \eta \left( 16\tau \right)   ^{4} \mbox{}  \eta \left( 4\tau \right)   ^{3}}}$ & 7/2 \\ 
  25 & 16 & $6 A_{3}$ & $ \eta \left( 4\tau \right)   ^{6}$ & 3 \\ 
  26 & 16 & $4 A_{1} +  A_{3} +  A_{7} +  D_{4}$ & ${\frac {  \eta \left( 8\tau \right)   ^{7}  \eta \left( 2\tau \right)   ^{2}}{  \eta \left( 16\tau \right)   ^{2} \mbox{}\eta \left( 4\tau \right) }}$ & 3 \\ 
  27 & 16 & $2 A_{1} + 4 D_{4}$ & ${\frac {  \eta \left( 8\tau \right)   ^{14}  \eta \left( 2\tau \right)   ^{4}}{  \eta \left( 4\tau \right)   ^{8} \mbox{}  \eta \left( 16\tau \right)   ^{4}}}$ & 3 \\ 
  28 & 16 & $2 A_{1} +  A_{3} + 2 A_{7}$ & $ \eta \left( 8\tau \right)   ^{2}\eta \left( 4\tau \right)   \eta \left( 2\tau \right)   ^{2}$ & 5/2 \\ 
  29 & 16 & $ A_{3} +  D_{4} + 2 D_{6}$ & ${\frac {\eta \left( 4\tau \right)   \eta \left( 8\tau \right)   ^{7}  \eta \left( \tau \right)   ^{2}}{  \eta \left( 16\tau \right)   ^{2} \mbox{}  \eta \left( 2\tau \right)   ^{3}}}$ & 5/2 \\ 
  30 & 18 & $8 A_{1} + 4 A_{2}$ & ${\frac {  \eta \left( 9\tau \right)   ^{8}  \eta \left( 6\tau \right)   ^{4}}{  \eta \left( 18\tau \right)   ^{4}}}$ & 4 \\ 
  31 & 18 & $2 A_{1} + 3 A_{2} + 2 A_{5}$ & ${\frac {  \eta \left( 9\tau \right)   ^{2}  \eta \left( 6\tau \right)   ^{3}  \eta \left( 3\tau \right)   ^{2} \mbox{}}{\eta \left( 18\tau \right) }}$ & 3 \\ 
  32 & 20 & $2 A_{1} + 4 A_{3} +  A_{4}$ & ${\frac {  \eta \left( 10\tau \right)   ^{2}  \eta \left( 5\tau \right)   ^{4}\eta \left( 4\tau \right) }{\eta \left( 20 \mbox{}\tau \right) }}$ & 3 \\ 
  33 & 21 & $6 A_{2} +  A_{6}$ & ${\frac {  \eta \left( 7\tau \right)   ^{6}\eta \left( 3\tau \right) }{\eta \left( 21\tau \right) }}$ & 3 \\ 
  34 & 24 & $5 A_{1} + 3 A_{2} + 2 A_{3}$ & ${\frac {  \eta \left( 12\tau \right)   ^{5}  \eta \left( 8\tau \right)   ^{3}  \eta \left( 6\tau \right)   ^{2} \mbox{}}{  \eta \left( 24\tau \right)   ^{3}}}$ & 7/2 \\ 
  35 & 24 & $4 A_{1} + 2 A_{2} + 2 A_{5}$ & ${\frac {  \eta \left( 12\tau \right)   ^{4}  \eta \left( 8\tau \right)   ^{2}  \eta \left( 4\tau \right)   ^{2} \mbox{}}{  \eta \left( 24\tau \right)   ^{2}}}$ & 3 \\ 
  36 & 24 & $5 A_{1} +  A_{3} +  A_{5} +  D_{5}$ & ${\frac {  \eta \left( 12\tau \right)   ^{7}\eta \left( 6\tau \right) \eta \left( 2\tau \right) \eta \left( 8\tau \right) }{  \eta \left( 24\tau \right)   ^{3} \mbox{}\eta \left( 4\tau \right) }}$ & 3 \\ 
  37 & 24 & $2 A_{2} +  A_{5} +  D_{4} +  E_{6}$ & ${\frac {  \eta \left( 8\tau \right)   ^{4}\eta \left( 4\tau \right) \eta \left( 3\tau \right)   \eta \left( 12\tau \right)   ^{4} \mbox{}\eta \left( \tau \right) }{  \eta \left( 6\tau \right)   ^{2}  \eta \left( 24\tau \right)   ^{2}  \eta \left( 2\tau \right)   ^{2}}}$ & 5/2 \\ 
  38 & 24 & $2 A_{2} +  A_{3} + 2 E_{6}$ & ${\frac {  \eta \left( 8\tau \right)   ^{6}\eta \left( 6\tau \right)   \eta \left( \tau \right)   ^{2}  \eta \left( 12\tau \right)   ^{2} \mbox{}}{  \eta \left( 2\tau \right)   ^{4}  \eta \left( 24\tau \right)   ^{2}}}$ & 5/2 \\ 
  39 & 32 & $8 A_{1} + 3 A_{3}$ & ${\frac {  \eta \left( 16\tau \right)   ^{8}  \eta \left( 8\tau \right)   ^{3}}{  \eta \left( 32\tau \right)   ^{4}}}$ & 7/2 \\ 
  40 & 32 & $9 A_{1} + 2 D_{4}$ & ${\frac {  \eta \left( 16\tau \right)   ^{15}  \eta \left( 4\tau \right)   ^{2}}{  \eta \left( 32\tau \right)   ^{6} \mbox{}  \eta \left( 8\tau \right)   ^{4}}}$ & 7/2 \\ 
  41 & 32 & $3 A_{1} + 5 A_{3}$ & ${\frac {  \eta \left( 16\tau \right)   ^{3}  \eta \left( 8\tau \right)   ^{5}}{  \eta \left( 32\tau \right)   ^{2}}}$ & 3 \\ 
  42 & 32 & $4 A_{1} + 2 A_{3} + 2 D_{4}$ & ${\frac {  \eta \left( 16\tau \right)   ^{10}  \eta \left( 4\tau \right)   ^{2}}{  \eta \left( 32\tau \right)   ^{4} \mbox{}  \eta \left( 8\tau \right)   ^{2}}}$ & 3 \\ 
  43 & 32 & $5 A_{1} + 2 A_{7}$ & ${\frac {  \eta \left( 16\tau \right)   ^{5}  \eta \left( 4\tau \right)   ^{2}}{  \eta \left( 32\tau \right)   ^{2}}}$ & 5/2 \\ 
  44 & 32 & $2 A_{1} + 2 A_{3} +  A_{7} +  D_{4}$ & ${\frac {  \eta \left( 16\tau \right)   ^{5}  \eta \left( 4\tau \right)   ^{2}}{  \eta \left( 32\tau \right)   ^{2}}}$ & 5/2 \\ 
  45 & 32 & $3 A_{1} +  D_{4} + 2 D_{6}$ & ${\frac {  \eta \left( 16\tau \right)   ^{10}  \eta \left( 2\tau \right)   ^{2}}{  \eta \left( 32\tau \right)   ^{4} \mbox{}  \eta \left( 4\tau \right)   ^{3}}}$ & 5/2 \\ 
  46 & 36 & $2 A_{1} + 2 A_{2} + 4 A_{3}$ & ${\frac {  \eta \left( 18\tau \right)   ^{2}  \eta \left( 12\tau \right)   ^{2}  \eta \left( 9\tau \right)   ^{4} \mbox{}}{  \eta \left( 36\tau \right)   ^{2}}}$ & 3 \\ 
  47 & 36 & $ A_{1} + 6 A_{2} +  A_{5}$ & ${\frac {\eta \left( 18\tau \right)   \eta \left( 12\tau \right)   ^{6}\eta \left( 6\tau \right) }{  \eta \left( 36 \mbox{}\tau \right)   ^{2}}}$ & 3 \\ 
  48 & 36 & $6 A_{1} +  A_{2} + 2 A_{5}$ & ${\frac {  \eta \left( 18\tau \right)   ^{6}\eta \left( 12\tau \right)   \eta \left( 6\tau \right)   ^{2}}{  \eta \left( 36 \mbox{}\tau \right)   ^{3}}}$ & 3 \\ 
  49 & 48 & $5 A_{1} + 6 A_{2}$ & ${\frac {  \eta \left( 24\tau \right)   ^{5}  \eta \left( 16\tau \right)   ^{6}}{  \eta \left( 48\tau \right)   ^{4}}}$ & 7/2 \\ 
  50 & 48 & $6 A_{2} + 2 A_{3}$ & ${\frac {  \eta \left( 16\tau \right)   ^{6}  \eta \left( 12\tau \right)   ^{2}}{  \eta \left( 48\tau \right)   ^{2}}}$ & 3 \\ 
  51 & 48 & $5 A_{1} +  A_{2} + 2 A_{3} +  A_{5}$ & ${\frac {  \eta \left( 24\tau \right)   ^{5}\eta \left( 16\tau \right)   \eta \left( 12\tau \right)   ^{2} \mbox{}\eta \left( 8\tau \right) }{  \eta \left( 48\tau \right)   ^{3}}}$ & 3 \\ 
  52 & 48 & $4 A_{1} + 3 A_{5}$ & ${\frac {  \eta \left( 24\tau \right)   ^{4}  \eta \left( 8\tau \right)   ^{3}}{  \eta \left( 48\tau \right)   ^{2}}}$ & 5/2 \\ 
  53 & 48 & $ A_{1} +  A_{2} + 2 A_{3} + 2 D_{5}$ & ${\frac {  \eta \left( 24\tau \right)   ^{5}  \eta \left( 16\tau \right)   ^{3}  \eta \left( 12\tau \right)   ^{2} \mbox{}  \eta \left( 4\tau \right)   ^{2}}{  \eta \left( 48\tau \right)   ^{3}  \eta \left( 8\tau \right)   ^{4}}}$ & 5/2 \\ 
  54 & 48 & $4 A_{1} +  A_{2} +  A_{7} +  E_{6}$ & ${\frac {  \eta \left( 24\tau \right)   ^{5}  \eta \left( 16\tau \right)   ^{3}\eta \left( 6\tau \right)  \mbox{}\eta \left( 2\tau \right) }{  \eta \left( 48\tau \right)   ^{3}  \eta \left( 4\tau \right)   ^{2}}}$ & 5/2 \\ 
  55 & 60 & $4 A_{1} + 3 A_{2} + 2 A_{4}$ & ${\frac {  \eta \left( 30\tau \right)   ^{4}  \eta \left( 20\tau \right)   ^{3}  \eta \left( 12\tau \right)   ^{2} \mbox{}}{  \eta \left( 60\tau \right)   ^{3}}}$ & 3 \\ 
  56 & 64 & $5 A_{1} + 3 A_{3} +  D_{4}$ & ${\frac {  \eta \left( 32\tau \right)   ^{8}\eta \left( 16\tau \right) \eta \left( 8\tau \right) }{  \eta \left( 64 \mbox{}\tau \right)   ^{4}}}$ & 3 \\ 
  57 & 64 & $6 A_{1} + 3 D_{4}$ & ${\frac {  \eta \left( 32\tau \right)   ^{15}  \eta \left( 8\tau \right)   ^{3}}{  \eta \left( 64\tau \right)   ^{6} \mbox{}  \eta \left( 16\tau \right)   ^{6}}}$ & 3 \\ 
  58 & 64 & $3 A_{1} + 3 A_{3} +  A_{7}$ & ${\frac {  \eta \left( 32\tau \right)   ^{3}  \eta \left( 16\tau \right)   ^{3}\eta \left( 8\tau \right)  \mbox{}}{  \eta \left( 64\tau \right)   ^{2}}}$ & 5/2 \\ 
  59 & 64 & $5 A_{3} +  D_{4}$ & ${\frac {  \eta \left( 32\tau \right)   ^{3}  \eta \left( 16\tau \right)   ^{3}\eta \left( 8\tau \right)  \mbox{}}{  \eta \left( 64\tau \right)   ^{2}}}$ & 5/2 \\ 
  60 & 64 & $4 A_{1} +  A_{3} + 2 D_{6}$ & ${\frac {  \eta \left( 32\tau \right)   ^{8}  \eta \left( 16\tau \right)   ^{3}  \eta \left( 4\tau \right)   ^{2} \mbox{}}{  \eta \left( 64\tau \right)   ^{4}  \eta \left( 8\tau \right)   ^{4}}}$ & 5/2 \\ 
  61 & 72 & $4 A_{1} + 3 A_{2} +  A_{3} +  D_{5}$ & ${\frac {  \eta \left( 36\tau \right)   ^{6}  \eta \left( 24\tau \right)   ^{4}\eta \left( 18\tau \right)  \mbox{}\eta \left( 6\tau \right) }{  \eta \left( 72\tau \right)   ^{4}  \eta \left( 12\tau \right)   ^{2}}}$ & 3 \\ 
  62 & 72 & $3 A_{1} + 2 A_{3} + 2 A_{5}$ & ${\frac {  \eta \left( 36\tau \right)   ^{3}  \eta \left( 18\tau \right)   ^{2}  \eta \left( 12\tau \right)   ^{2} \mbox{}}{  \eta \left( 72\tau \right)   ^{2}}}$ & 5/2 \\ 
  63 & 72 & $ A_{2} + 3 A_{3} + 2 D_{4}$ & ${\frac {\eta \left( 24\tau \right)   \eta \left( 9\tau \right)   ^{2}  \eta \left( 36\tau \right)   ^{6}}{  \eta \left( 72\tau \right)   ^{3} \mbox{}\eta \left( 18\tau \right) }}$ & 5/2 \\ 
  64 & 80 & $3 A_{1} + 4 A_{4}$ & ${\frac {  \eta \left( 40\tau \right)   ^{3}  \eta \left( 16\tau \right)   ^{4}}{  \eta \left( 80\tau \right)   ^{2}}}$ & 5/2 \\ 
  65 & 96 & $3 A_{1} + 3 A_{2} + 3 A_{3}$ & ${\frac {  \eta \left( 48\tau \right)   ^{3}  \eta \left( 32\tau \right)   ^{3}  \eta \left( 24\tau \right)   ^{3} \mbox{}}{  \eta \left( 96\tau \right)   ^{3}}}$ & 3 \\ 
  66 & 96 & $2 A_{1} + 2 A_{2} +  A_{3} + 2 A_{5}$ & ${\frac {  \eta \left( 48\tau \right)   ^{2}  \eta \left( 32\tau \right)   ^{2}\eta \left( 24\tau \right)  \mbox{}  \eta \left( 16\tau \right)   ^{2}}{  \eta \left( 96\tau \right)   ^{2}}}$ & 5/2 \\ 
  67 & 96 & $2 A_{1} + 3 A_{2} +  A_{7} +  D_{4}$ & ${\frac {  \eta \left( 48\tau \right)   ^{5}  \eta \left( 32\tau \right)   ^{3}  \eta \left( 12\tau \right)   ^{2} \mbox{}}{  \eta \left( 96\tau \right)   ^{3}  \eta \left( 24\tau \right)   ^{2}}}$ & 5/2 \\ 
  68 & 96 & $3 A_{1} + 2 A_{3} +  A_{5} +  D_{5}$ & ${\frac {  \eta \left( 48\tau \right)   ^{5}  \eta \left( 24\tau \right)   ^{2}\eta \left( 8\tau \right)  \mbox{}\eta \left( 32\tau \right) }{  \eta \left( 96\tau \right)   ^{3}\eta \left( 16\tau \right) }}$ & 5/2 \\ 
  69 & 96 & $3 A_{1} + 2 A_{2} + 2 E_{6}$ & ${\frac {  \eta \left( 48\tau \right)   ^{5}  \eta \left( 32\tau \right)   ^{6}  \eta \left( 4\tau \right)   ^{2} \mbox{}}{  \eta \left( 96\tau \right)   ^{4}  \eta \left( 8\tau \right)   ^{4}}}$ & 5/2 \\ 
  70 & 120 & $2 A_{1} +  A_{2} + 2 A_{3} +  A_{4} +  A_{5}$ & ${\frac {  \eta \left( 60\tau \right)   ^{2}\eta \left( 40\tau \right)   \eta \left( 30\tau \right)   ^{2} \mbox{}\eta \left( 24\tau \right) \eta \left( 20\tau \right) }{  \eta \left( 120\tau \right)   ^{2}}}$ & 5/2 \\ 
  71 & 128 & $3 A_{1} + 2 A_{3} +  D_{4} +  D_{6}$ & ${\frac {  \eta \left( 64\tau \right)   ^{8}\eta \left( 32\tau \right) \eta \left( 8\tau \right) }{  \eta \left( 128\tau \right)   ^{4} \mbox{}\eta \left( 16\tau \right) }}$ & 5/2 \\ 
  72 & 144 & $ A_{1} + 4 A_{2} + 2 A_{5}$ & ${\frac {\eta \left( 72\tau \right)   \eta \left( 48\tau \right)   ^{4}  \eta \left( 24\tau \right)   ^{2}}{  \eta \left( 144 \mbox{}\tau \right)   ^{2}}}$ & 5/2 \\ 
  73 & 160 & $2 A_{1} + 3 A_{3} + 2 A_{4}$ & ${\frac {  \eta \left( 80\tau \right)   ^{2}  \eta \left( 40\tau \right)   ^{3}  \eta \left( 32\tau \right)   ^{2} \mbox{}}{  \eta \left( 160\tau \right)   ^{2}}}$ & 5/2 \\ 
  74 & 168 & $ A_{1} + 3 A_{2} + 2 A_{3} +  A_{6}$ & ${\frac {\eta \left( 84\tau \right)   \eta \left( 56\tau \right)   ^{3}  \eta \left( 42\tau \right)   ^{2} \mbox{}\eta \left( 24\tau \right) }{  \eta \left( 168\tau \right)   ^{2}}}$ & 5/2 \\ 
  75 & 192 & $2 A_{1} + 6 A_{2} +  D_{4}$ & ${\frac {  \eta \left( 96\tau \right)   ^{5}  \eta \left( 64\tau \right)   ^{6}\eta \left( 24\tau \right)  \mbox{}}{  \eta \left( 192\tau \right)   ^{4}  \eta \left( 48\tau \right)   ^{2}}}$ & 3 \\ 
  76 & 192 & $2 A_{1} +  A_{2} + 2 A_{3} +  A_{5} +  D_{4}$ & ${\frac {  \eta \left( 96\tau \right)   ^{5}\eta \left( 64\tau \right) \eta \left( 32\tau \right)  \mbox{}\eta \left( 24\tau \right) }{  \eta \left( 192\tau \right)   ^{3}}}$ & 5/2 \\ 
  77 & 192 & $2 A_{1} +  A_{2} + 3 A_{3} +  E_{6}$ & ${\frac {  \eta \left( 96\tau \right)   ^{3}  \eta \left( 64\tau \right)   ^{3}  \eta \left( 48\tau \right)   ^{3} \mbox{}\eta \left( 8\tau \right) }{  \eta \left( 192\tau \right)   ^{3}  \eta \left( 16\tau \right)   ^{2}}}$ & 5/2 \\ 
  78 & 288 & $2 A_{1} + 2 A_{2} +  A_{3} + 2 D_{5}$ & ${\frac {  \eta \left( 144\tau \right)   ^{6}  \eta \left( 96\tau \right)   ^{4} \mbox{}\eta \left( 72\tau \right)   \eta \left( 24\tau \right)   ^{2}}{  \eta \left( 288\tau \right)   ^{4}  \eta \left( 48\tau \right)   ^{4}}}$ & 5/2 \\ 
  79 & 360 & $ A_{1} + 2 A_{2} + 2 A_{3} + 2 A_{4}$ & ${\frac {\eta \left( 180\tau \right)   \eta \left( 120\tau \right)   ^{2}  \eta \left( 90\tau \right)   ^{2} \mbox{}  \eta \left( 72\tau \right)   ^{2}}{  \eta \left( 360\tau \right)   ^{2}}}$ & 5/2 \\ 
  80 & 384 & $ A_{1} + 3 A_{2} + 2 A_{3} +  D_{6}$ & ${\frac {  \eta \left( 192\tau \right)   ^{3}  \eta \left( 128\tau \right)   ^{3} \mbox{}  \eta \left( 96\tau \right)   ^{3}\eta \left( 24\tau \right) }{  \eta \left( 384\tau \right)   ^{3}  \eta \left( 48\tau \right)   ^{2}}}$ & 5/2 \\ 
  81 & 960 & $ A_{1} + 3 A_{2} + 2 A_{4} +  D_{4}$ & ${\frac {  \eta \left( 480\tau \right)   ^{4}  \eta \left( 320\tau \right)   ^{3} \mbox{}  \eta \left( 192\tau \right)   ^{2}\eta \left( 120\tau \right) }{  \eta \left( 960\tau \right)   ^{3}  \eta \left( 240\tau \right)   ^{2}}}$ & 5/2 \\ 
   \hline
\caption{Table of the modular forms $Z_{X,G}^{-1}$ for all symplectic
$G$ actions.} \label{table: list of eta products}
\end{longtable}


\bibliography{Bryan-Gyenge}
\bibliographystyle{plain}




\end{document}
