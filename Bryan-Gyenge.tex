\documentclass{article}

\title{$G$-fixed Hilbert schemes on $K3$ surfaces and modular forms.}
\author{Jim Bryan and \'{A}d\'{a}m~Gyenge} 
\date{\today}

%\address{
%Department of Mathematics\\
%University of British Columbia \\
%Room 121, 1984 Mathematics Road  \\
%Vancouver, B.C., Canada V6T 1Z2  
%}

%\usepackage{diagrams}

\usepackage{amsmath}
\usepackage{bm}
\usepackage{verbatim}
\usepackage{amsmath,amsthm,amsfonts}
\usepackage{amssymb}
\usepackage{times}
\usepackage{longtable}
%\usepackage{amstex}


\newtheorem{theorem}{Theorem}%[section]
\newtheorem{proposition}[theorem]{Proposition}
\newtheorem{conjecture}[theorem]{Conjecture}
\newtheorem{lemma}[theorem]{Lemma}
\newtheorem{corollary}[theorem]{Corollary}
\newtheorem{convention}{Convention}[theorem]

\theoremstyle{definition}

\newtheorem{def-theorem}[theorem]{Definition-Theorem}
\newtheorem{remark}[theorem]{Remark}
\newtheorem{definition}[theorem]{Definition}
\newtheorem{example}[theorem]{Example}


\newcommand{\half}{\frac{1}{2}}
\newcommand{\CC} {{\mathbb C}}          % complex numbers
\newcommand{\NN} {{\mathbb N}}		% natural numbers
\newcommand{\RR} {{\mathbb R}}		% real numbers
\newcommand{\ZZ} {{\mathbb Z}}		% integers
\newcommand{\QQ} {{\mathbb Q}}		% rationals
\newcommand{\PP}{\mathbb{P}}
\newcommand{\HH}{\mathbb{H}}
\newcommand{\LL}{\mathbb{L}}
\renewcommand{\O}{\mathcal{O}}
%\renewcommand{\top}{\mathrm{t}}
\renewcommand{\top}{\,\mathsf{t}}


\newcommand{\mvec}{\bm{m}}
\newcommand{\zetavec}{\bm{\zeta }}
\newcommand{\dvec}{\bm{d }}
\newcommand{\uvec}{\bm{u }}
\newcommand{\vvec}{\bm{v }}

%\newcommand{\mvec}{\vec{m}}
%\newcommand{\zetavec}{\vec{\zeta }}
%\newcommand{\dvec}{\vec{d }}
%\newcommand{\uvec}{\vec{u }}
%\newcommand{\vvec}{\vec{v }}


\newcommand{\Hom}{\operatorname{Hom}}
\newcommand{\Ker}{\operatorname{Ker}}
\newcommand{\End}{\operatorname{End}}
\newcommand{\Tr}{\operatorname{tr}}
\newcommand{\tr}{\operatorname{tr}}
\newcommand{\Sym}{\operatorname{Sym}}
\newcommand{\Coker}{\operatorname{Coker}}
\newcommand{\im}{\operatorname{Im}}
\newcommand{\Hilb}{\operatorname{Hilb}}

\begin{document}

\maketitle


\begin{abstract}
Let $X$ be a complex $K3$ surface with an effective action of a group
$G$ which preserves the holomorphic symplectic form. Let 
\[
Z_{X,G}(q) = \sum_{n=0}^{\infty} e(\Hilb^{n}(X)^{G})q^{n-1}
\]
be the generating function for the Euler characteristics of Hilbert
scheme of $G$-invariant length $n$ subschemes. We show that its
reciprocal, $Z_{X,G}(q)^{-1}$ is the Fourier expansion of a modular
cusp form of weight $\half e(X/G)$ and index $|G|$. We give an
explicit formula for $Z_{X,G}$ in terms of the Dedekind eta function
for all 82 possible $(X,G)$. 
\end{abstract}



%\markboth{???}  {???}



%\tableofcontents
%\pagebreak


\section{Introduction}

Let $X$ be a complex $K3$ surface with an effective action of a group
$G$ which preserves the holomorphic symplectic form. Mukai showed that
such $G$ are precisely the subgroups of the Mathieu group
$M_{23}\subset M_{24}$ such that the induced action on the set
$\{1,\dots ,24 \}$ has at least five orbits
\cite{mukai1988finite}. Xiao classified all possible actions into
82 possible topological types of the quotient $X/G$ \cite{xiao1996galois}.

The \emph{$G$-fixed Hilbert scheme} of $X$ parameterizes
$G$-invariant length $n$ subschemes $Z\subset X$. It can be
identified with the $G$-fixed point locus in the Hilbert scheme of
points: 
\[
\Hilb^{n}(X)^{G} \subset \Hilb^{n}(X)
\]

We define the corresponding \emph{$G$-fixed partition function} of
$X$ by
\[
Z_{X,G}(q) = \sum_{n=0}^{\infty} e(\Hilb^{n}(X)^{G}) q^{n-1} 
\]
where $e(-)$ is topological Euler characteristic.

Throughout this paper we set
\[
q=\exp\left(2\pi i \tau  \right)
\]
so that we may regard $Z_{X,G}$ as a function of $\tau \in \HH$ where
$\HH$ is the upper half-plane.

Our main result is the following:

\begin{theorem}
The function $Z_{X,G}(q)^{-1}$ is a modular cusp form\footnote{See
section \S~\ref{} for notation and definitions regarding modular
forms.} of weight $\half e(X/G)$ for the congruence subgroup
$\Gamma_{0}(|G|)$.
\end{theorem}
   

Our theorem specializes in the case where $G$ is the trivial group to
a famous result of G\"ottsche \cite{gottsche1990betti}. The case where $G$ is a
cyclic group was proved in \cite{bryan2018chl}. One can interpret
our result as an instance of the Vafa-Witten S-duality conjecture for
the orbifold $[X/G]$ (see Remark~???). The partition function
$Z_{X,G}(q)$ also has an interpretation in enumerative geometry: its
coefficients count $G$-invariant rational curves on $X$ (see
Remark~???).

We also give an explicit formula for $Z_{X,G}(q)$ in terms of the
Dedekind eta function
\[
\eta (\tau ) = q^{\frac{1}{24}}\prod_{n=1}^{\infty} (1-q^{n})
\]
as follows. Let $p_{1},\dots ,p_{r}$ be the singular points of $X/G$
and let $G_{1},\dots ,G_{r}$ be the corresponding stabilizer subgroups
of $G$. The singular points are necessarily of ADE type: they are
locally given by $\CC^{2}/G_{i}$ where $G_{i}\subset SU(2)$. Finite
subgroups of $SU(2)$ have an ADE classification and we let
$\Delta_{1},\dots ,\Delta_{r}$ denote the corresponding ADE root
systems.

For any finite subgroup $G_{\Delta}\subset SU(2)$ with associated root
system $\Delta$ we define the \emph{local $G_{\Delta }$-fixed
partition function} by
\[
Z_{\Delta} (q) = \sum_{n=0}^{\infty}
e\left(\Hilb^{n}(\CC^{2})^{G_{\Delta}} \right) \, q^{n-\frac{1}{24}} .
\]
We will prove in Lemma~\ref{lem: local series as theta/eta}  that
\[
Z_{\Delta}(q) =\frac{\theta_{\Delta}(\tau)}{\eta (k\tau )^{N+1}}
\]
where $\theta_{\Delta}(\tau )$ is a shifted theta function for the
root lattice of $\Delta$, $N$ is the rank of the root system, and
$k=|G_{\Delta}|$.




The 82 possible collections of ADE root systems $\Delta_{1},\dots
,\Delta_{r}$ associated to $(X,G)$ a $K3$ surface with a symplectic
$G$ action, are given in table~\ref{table: list of eta products} and
we note that $\Delta_{i}\in \{A_{1},\dots
,A_{7},D_{4},D_{5},D_{6},E_{6} \}$. We let $k=|G|$, $k_{i}=|G_{i}|$,
and
\[
a = e(X/G) - r=\frac{24}{k}-\sum_{i=1}^{r} \frac{1}{k_{i}}.
\]
\begin{theorem}\label{thm: eta product formula for Z}
With the above notation we have
\[
Z_{X,G}(q) = \eta^{-a}(k\tau )\prod_{i=1}^{r}
Z_{\Delta_{i}}\left(\frac{k\tau}{k_{i}} \right)
\]
where 
\begin{align*}
Z_{A_{n}} (\tau ) &=\frac{1}{\eta (\tau )}, \quad n\geq 1\\
Z_{D_{n}} (\tau ) &=\frac{\eta^{2}(2\tau )\eta((4n-8)\tau )}{\eta (\tau )\eta (4\tau )\eta^{2}((2n-4)\tau )}, \quad 4\leq n\leq 6\\
Z_{E_{6}} (\tau ) &=\frac{\eta^{2}(2\tau )\eta(24\tau )}{\eta (\tau
)\eta^{2} (8\tau )\eta(12\tau )}
\end{align*}
\end{theorem}
We conjecture in \ref{} that the formula for $Z_{D_{n}}$ holds for all $n\geq 4$
and we provide explicit conjectural formulas for $Z_{E_{7}}$ and
$Z_{E_{8}}$. In table~\ref{table: list of eta products}  we have listed explictly the eta product
of the modular form $(Z_{X,G})^{-1}$ for all 82 possible cases of $(X,G)$.

Having obtained explicit eta product expressions for $Z_{X,G}(q)$ in all 82 possible
cases allows us to make several observational corollaries:

\begin{corollary}\label{cor: if G is a subgp of E then Zinv is a Hecke
eigenform}
If $G$ is a finite subgroup of an elliptic curve $E$, i.e. $G$ is
isomorphic to a product of one or two cyclic groups, then
$Z_{X,G}(q)^{-1}$ is a Hecke eigenform. On table~\ref{table: list of
eta products} these are the 13 cases having Xiao number in the set
$\{0,1,2,3,4,5,7,8,11,14,15,19,25 \}$. Moreover, in each of these
cases, the dimension of the Hecke eigenspace is one. 
\end{corollary}

We remark that in these cases, we may form a Calabi-Yau threefold
called a CHL model by taking the free group quotient
\[
(X\times E)/G
\]
Then the partition function $Z_{X,G}(q)$ gives the (modified)
Donaldson-Thomas invariants of $(X\times E)/G$ in curve classes which
are degree zero over $X/G$ (see \cite{bryan2018chl}).
For any eta product expression of a modular form, one may easily compute the
order of vanishing (or pole) at any of the cusps
\cite[Cor~2.2]{kohler2011eta}. Performing this computation on the 82
cases yields the following

\begin{corollary}\label{cor: vanishing at cusps}
The modular form $Z_{X,G}(q)^{-1}$ always vanishes at the cusps
$i\infty$ and $0$. Moreover,
\begin{itemize}
\item $Z_{X,G}(q)^{-1}$ vanishes at all cusps except for the eleven
cases with Xiao number in the set $\{13,20,27,29,37,38,45,53,54,60,69 \}$.
\item $Z_{X,G}(q)^{-1}$ is holomorphic except for the two cases with
Xiao number 38 or 69, which have poles at the cusps $1/2$ and $1/8$
respectively. These are precisely the cases where $X/G$ has two
singularities of type $E_{6}$.
\end{itemize}
\end{corollary}

\subsection{Enumerative applications}\label{subsec: enumerative applications}

We have already mentioned above the enumerative application to the CHL
Calabi-Yau threefold $(X\times E)/G$ in the case where $G\subset E$ is
a finite subgroup of an elliptic curve. Another application is the
following generalization of the Yau-Zaslow formula counting rational
curves on $X$.

Let $X\subset \PP^{g}$ be an embedding obtained from a $G$-equivariant
ample line bundle $L$ with $c_{1}(L)$ a primitive class of square
$2g-2$. Then the coefficient of $q^{g-1}$ in $Z_{X,G}(q)$ is the
number of hyperplane sections which are $G$-invariant rational curves,
counted with multiplicity.

\dots add discussion of the above. Formulate as proposition?


\subsection{Structure of the paper}


\section{The local partition functions}\label{sec: local partition functions}

The classical McKay correspondence associates an ADE root system
$\Delta$ to any finite subgroup $ G_{\Delta}\subset SL_{2}(\CC)$. Using
the work of Nakajima \cite{nakajima2002geometric}, the partition function of the
Euler characteristics of the Hilbert scheme of points on the stack
quotient $[\CC^{2}/G_{\Delta}]$ was computed explicitly in
\cite{gyenge2015euler} in terms of the root data of $\Delta$.

The local partition functions $Z_{\Delta}(q)$ considered in this paper
are obtained from a specialization of the partition functions of the
stack $[\CC^{2} /G_{\Delta}]$ and in this section, we use this to express
$Z_{\Delta}(q)$ in terms of a shifted theta function for the root
lattice of $\Delta$.

A zero-dimensional substack $Z\subset [\CC^{2}/G_{\Delta}]$ may be
regarded as a $G_{\Delta}$ invariant, zero-dimensional subscheme of
$\CC^{2}$. Consequently, we may identify the Hilbert scheme of points
on the stack $[\CC^{2}/G_{\Delta}]$ with the $G_{\Delta}$ fixed locus
of the Hilbert scheme of points on $\CC^{2}$: 
\[
\Hilb \left([\CC^{2}/G_{\Delta}] \right) = \Hilb
(\CC^{2})^{G_{\Delta}} .
\]

This Hilbert scheme has components indexed by representations $\rho$
of $G_{\Delta}$ as follows
\begin{equation*}
\Hilb^{\rho} \left([\CC^{2}/G_{\Delta}] \right) = \left\{ Z\subset
\CC^{2}, \text{ $Z$ is $G_{\Delta}$ invariant and $H^{0}(\O_{Z})\cong
\rho $} \right\}.
\end{equation*}

Let $\{\rho_{0},\dotsc ,\rho_{N} \}$ be the irreducible
representations of $G_{\Delta}$ where $\rho_{0}$ is the trivial
representation. We note that $N$ is also the rank of $\Delta$. We
define
\[
Z_{[\CC^{2}/G_{\Delta}]} (q_{0},\dotsc ,q_{N}) = \sum_{m_{0},\dotsc
,m_{N}=0}^{\infty} e\left(\Hilb^{m_{0}\rho_{0}+\dotsb
+m_{N}\rho_{M}}([\CC^{2}/G_{\Delta}]) \right) q_{0}^{m_{0}}\dotsb
q_{N}^{m_{N}} .
\]

Recall that our local partition function $Z_{\Delta}(q)$ is defined by
\[
Z_{\Delta}(q) = \sum_{n=0}^{\infty}
e\left(\Hilb^{n}(\CC^{2})^{G_{\Delta}} \right) q^{n-\frac{1}{24}}. 
\]
We then readily see that
\[
Z_{\Delta}(q) = q^{\frac{-1}{24}}\cdot  Z_{[\CC^{2}/G_{\Delta}]}
(q_{0},\dotsc ,q_{N})|_{q_{i}=q^{d_{i}}}
\]
where
\[
d_{i} =\dim \rho_{i}.
\]

The following theorem is given in \cite[Thm~1.3]{gyenge2015euler}
where it is attributed to Nakajima \cite{nakajima2002geometric}:
\begin{theorem} \label{thm: Zorbifold formula}
Let $C_{\Delta}$ be the Cartan matrix of the root system $\Delta$,
then 
\[
Z_{[\CC^{2}/G_{\Delta}]} (q_{0},\dotsc ,q_{N}) = \prod_{m=1}^{\infty}
(1-Q^{m})^{-N-1} \cdot \sum_{\mvec \in \ZZ^{N}} q_{1}^{m_{1}}\dotsb
q_{N}^{m_{N}} \cdot Q^{\half \mvec^{\top}\cdot C_{\Delta}\cdot \mvec}
\]
where $Q=q_{0}^{d_{0}}q_{1}^{d_{1}}\dotsb d_{N}^{d_{N}}$.
\end{theorem}
We note that under the specialization $q_{i}=q^{d_{i}}$, 
\[
Q=q^{d_{0}^{2}+\dotsb +d_{N}^{2}} = q^{k}
\]
where $k=|G|$ is the order of the group $G$.

We then obtain
\[
Z_{\Delta}(q) = q^{\frac{-1}{24}}\cdot
\prod_{m=1}^{\infty}(1-q^{km})^{-N-1} \cdot \sum_{\mvec \in \ZZ^{N}}
q^{\mvec^{\top}\cdot \dvec} \cdot q^{\frac{k}{2}\mvec^{\top}\cdot C_{\Delta}\cdot \mvec}
\]
where $\dvec =(d_{1},\dotsc ,d_{N})$.

Let $M_{\Delta}$ be the root lattice of $\Delta$ which we identify
with $\ZZ^{N}$ via the basis given by $\alpha_{1},\dotsc ,\alpha_{N}$,
the simple positive roots of $\Delta$. Under this identification, the
standard Weyl invariant bilinear form is given by
\[
(\uvec |\vvec ) = \uvec^{\top}\cdot C_{\Delta}\cdot \vvec .
\]
We define
\[
\zetavec = C_{\Delta}^{-1} \cdot \dvec 
\]
so that 
\[
\mvec^{\top}\cdot \dvec = \mvec^{\top}\cdot C_{\Delta} \cdot \zetavec
= (\mvec |\zetavec ).
\]
We may then write
\begin{align*}
Z_{\Delta}(q)& = q^{\frac{-1}{24}}\cdot
\prod_{m=1}^{\infty}(1-q^{km})^{-N-1}\cdot \sum _{\mvec \in M_{\Delta}} q^{(\mvec |\zetavec )+\frac{k}{2}(\mvec |\mvec)}\\
&= q^{A} \cdot \left(q^{\frac{k}{24}}\prod_{m=1}^{\infty}(1-q^{km})
\right)^{-N-1} \cdot \sum_{\mvec \in M_{\Delta}} q^{\frac{k}{2}\left(\mvec +\frac{1}{k}\zetavec |\mvec +\frac{1}{k}\zetavec  \right)}\\
& = q^{A}\cdot  \eta (k\tau )^{-N-1}\cdot  \theta_{\Delta} (\tau )
\end{align*}
where 
\[
A \quad = \quad \frac{-1}{24} + \frac{k(N+1)}{24} - \frac{1}{2k}(\zetavec
|\zetavec ) \quad = \quad \frac{k(N+1)-1}{24} - \frac{1}{2k}\dvec^{\top}\cdot
C_{\Delta}^{-1} \cdot \dvec 
\]
and $\theta_{\Delta}(\tau )$ is the shifted theta function:
\[
\theta_{\Delta}(\tau ) = \sum_{\mvec \in M_{\Delta}} q^{\frac{k}{2}\left(\mvec +\frac{1}{k}\zetavec |\mvec +\frac{1}{k}\zetavec  \right)}
\]
where as throughout this paper we have identified $q=\exp\left(2\pi i\tau  \right)$.

In section~???, lemma~??? we will prove that the identity $A=0$ holds
for all $\Delta$ and hence we obtain the following:
\begin{lemma}\label{lem: local series as theta/eta} The local series
$Z_{\Delta}(q)$ is given by
\[
Z_{\Delta}(q) = \frac{\theta_{\Delta}(\tau )}{\eta (k\tau )^{N+1}}. 
\]
\end{lemma}

We make the following conjecture which provides explicit eta product
expressions for the theta function $\theta_{\Delta}(\tau )$.

\begin{conjecture}\label{conj: eta product for theta function}
$\theta_{\Delta}(\tau )$ is given by
\begin{align}\label{eqn: theta function eta products}
\theta_{A_{n}}(\tau ) &= \frac{\eta^{n+1}((n+1)\tau )}{\eta (\tau )} ,\quad n\geq 1\\
\theta_{D_{n}}(\tau ) &= \frac{\eta^{2}(2\tau )\,  \eta^{n+2}((4n-8)\tau )}{\eta (\tau )\,  \eta (4\tau )\,  \eta^{2}((2n-4)\tau )} ,\quad n\geq 4\\
\theta_{E_{6}}(\tau ) & = \frac{\eta^{2}(2\tau )\, \eta^{8}(24\tau
)}{\eta (\tau )\, \eta^{2}(8\tau )\, \eta (12\tau )} ,\\
\theta_{E_{7}}(\tau ) & = \frac{\eta^{2}(2\tau )\, \eta^{9}(48\tau
)}{\eta (\tau )\, \eta(12\tau )\, \eta(16\tau )\, \eta (24\tau )} ,\\
\theta_{E_{8}}(\tau ) & = \frac{\eta^{2}(2\tau )\, \eta^{10}(120\tau
)}{\eta (\tau )\, \eta(24\tau )\, \eta(40\tau )\, \eta (60\tau )}.
\end{align}
\end{conjecture}

Since both sides of the above equations are explicit modular forms of known
weight and index, any given formula can be proved with a finite number
of computations. We will give a uniform geometric proof in the $A_{n}$ case for
all $n$ below, and we will give computational proofs for the cases of
$D_{4}$, $D_{5}$, $D_{6}$, and $E_{6}$ (Theorem \ref{}). These are the only cases needed
for our application to K3 surfaces. It would be desirable to have a
purely root theoretic way of writing the eta products and a pure root
theoretic proof of the conjecture. 

\begin{theorem}\label{thm: conjecture holds for An}
Conjecture~\ref{conj: eta product for theta function} holds for the
case of $A_{n}$. 
\end{theorem}

\begin{proof}
By Lemma~\ref{lem: local series as theta/eta} , the conjecture is
equivalent to the statment that 
\[
Z_{A_{n}}(q) = \frac{1}{\eta (\tau )}
\]
which is in turn equivalent to the statement
\[
\sum_{n=0}^{\infty} e\left(\Hilb (\CC^{2})^{\ZZ /(n+1)} \right)
\,q^{n} = \prod_{m=1}^{\infty} (1-q^{m})^{-1}.
\]
The action of $\ZZ /(n+1)$ on $\CC^{2}$ commutes with the action of
$\CC^{*}\times \CC^{*}$ on $\CC^{2}$ and consequently, the Euler
characteristics on the left hand side may be computed by counting
the $\CC^{*}\times \CC^{*}$-fixed subschemes, namely those given by
monomial ideals. Such subschemes of length $n$ have a well known
bijection with integer partitions of $n$, whose generating function is
given by the right hand side.
\end{proof}




\section{The Global series}\label{sec: the global series}

Recall that $p_{1},\dotsc ,p_{r}\in X/G$ are the singular points of
$X/G$ with corresponding stabilizer subgroups $G_{i}\subset G$ of
order $k_{i}$ and ADE type $\Delta_{i}$. Let $\{x_{i}^{1},\dotsc
,x_{i}^{k/k_{i}} \}$ be the orbit of $G$ in $X$ corresponding to the
point $p_{i}$ (recall that $k=|G|$).  We may stratify $\Hilb (X)^{G}$
according to the orbit types of subscheme as follows:

Let $Z\subset X$ be a $G$-invariant subscheme of length $nk$ whose
support lies on free orbits. Then $Z$ determines and is determined by
a length $n$ subscheme of $X/G-\{p_{1},\dotsc ,p_{r} \}$, i.e. a point
in $\Hilb^{n}(X/G-\{p_{1},\dotsc ,p_{r} \})$.

On the other hand, suppose $Z\subset X$ is a $G$-invariant subscheme
of length $\frac{nk}{k_{i}}$ supported on the orbit
$\{x_{i}^{1},\dotsc ,x_{i}^{k/k_{i}} \}$. Then $Z$ determines and is
determined by the length $n$ component of $Z$ supported on a formal
neighborhood of one of the points, say $x_{i}^{1}$. Choosing a
$G_{i}$-equivariant isomorphism of the formal neighborhood of
$x_{i}^{1}$ in $S$ with the formal neighborhood of the origin in
$\CC^{2}$, we see that $Z$ determines and is determined by a point in
$\Hilb_{0}^{n}(\CC^{2})^{G_{i}}$, where $\Hilb_{0}^{n}(\CC^{2})\subset
\Hilb^{n}(\CC^{2})$ is the punctual Hilbert scheme parameterizing
subschemes supported on a formal neighborhood of the origin in
$\CC^{2}$.

By decomposing an arbitrary $G$-invariant subscheme into components of
the above types, we obtain a stratification of $\Hilb (X)^{G}$ into
strata which are given by products of $\Hilb (X/G - \{p_{1},\dotsc
,p_{r} \})$ and $\Hilb_{0}(\CC^{2})^{G_{1}},\dotsc
,\Hilb_{0}(\CC^{2})^{G_{r}}$. Then using the fact that Euler
characteristic is additive under stratifications and multiplicative
under products, we arrive at the following equation of generating
functions:
\begin{align}\label{eqn: stratification formula for sum e(hilb(X)G)}
 \nonumber \sum_{n=0}^{\infty} e\left(\Hilb^{n}(X)^{G} \right)\, q^{n}
=&\left(\sum_{n=0}^{\infty} e\left(\Hilb^{n}(X/G-\{p_{1},\dotsc ,p_{r}
\}) \right)\, q^{kn}  \right)\\
& \cdot \prod_{i=1}^{r}\left( \sum_{n=0}^{\infty}
e\left(\Hilb_{0}^{n}(\CC^{2})^{G_{i}} \right) \,
q^{\frac{nk}{k_{i}}} \right) .
\end{align}

As in the introduction, let $a = e(X/G-\{p_{1},\dotsc ,p_{r}
\})$. Then by G\"ottsche's formula \cite{gottsche1990betti},
\begin{align*}
\sum_{n=0}^{\infty} e\left(\Hilb^{n}(X)^{G} \right) q^{n} &=
\prod_{m=1}^{\infty} (1-q^{km})^{-a}\\
&= q^{\frac{ak}{24}} \eta (k\tau )^{-a}. 
\end{align*}

We also note that $e\left(\Hilb_{0}^{n}(\CC^{2})^{G_{i}}
\right)=e\left(\Hilb^{n}(\CC^{2})^{G_{i}} \right)$ since the natural
$\CC^{*}$ action on both $\Hilb_{0}^{n}(\CC^{2})^{G_{i}}$ and
$\Hilb^{n}(\CC^{2})^{G_{i}}$ have the same fixed points. Thus we may
write
\begin{align*}
\sum_{n=0}^{\infty} e\left(\Hilb_{0}^{n}(\CC^{2})^{G_{i}} \right) \,
q^{\frac{nk}{k_{i}}} &= \sum_{n=0}^{\infty} e\left(\Hilb^{n}(\CC^{2})^{G_{i}} \right) \,
q^{\frac{nk}{k_{i}}}\\
&= q^{\frac{k}{24k_{i}}} Z_{\Delta_{i}} \left(\frac{k\tau}{k_{i}}
\right) .
\end{align*}

Multiplying equation~\eqref{eqn: stratification formula for sum
e(hilb(X)G)} by $q^{-1}$ and substituting the above formulas, we find
that

\[
Z_{X,G}(q) = q^{-1 +\frac{ak}{24} + \sum \frac{k}{24k_{i}} } \cdot 
\eta (k\tau )^{-a}\cdot 
\prod_{i=1}^{r}Z_{\Delta_{i}}\left(\frac{k\tau}{k_{i}} \right) .
\]

The exponent of $q$ in the above equation is zero as is readily seen
from the following Euler characteristic calculation:
\begin{align*}
24 = e(S) &= e\left(S-\cup_{i=1}^{r} \{x_{i}^{1},\dotsc
,x_{i}^{k/k_{i}} \} \right) + \sum_{i=1}^{r} \frac{k}{k_{i}} \\
&= k \cdot e\left(S/G - \{p_{1},\dotsc ,p_{r} \} \right) + \sum_{i=1}^{r} \frac{k}{k_{i}} \\
&= k\cdot a + \sum_{i=1}^{r}\frac{k}{k_{i}} 
\end{align*}

We have thus proved that the first equation in Theorem~\ref{thm: eta
product formula for Z} always holds. Then since the only root systems
which can occur as singularities of $X/G$ are of type $A_{n}$ or
$D_{4}$, $D_{5}$, $D_{6}$, or $E_{6}$, we use Theorem~\ref{thm:
conjecture holds for An} and Theorem~\ref{} and we have completed the
proof of Theorem~\ref{thm: eta product formula for Z}.

\section{Modular forms}\label{sec: modular forms}

\subsection{Modular forms with multiplier systems and congruence subgroups}
\subsection{Multiplier systems and congruence subgroups of eta products}
\subsection{Multiplier systems and congruence subgroups of shifted theta functions}
\subsection{Sturm bounds and the proof of Theorem~???}



\pagebreak


\appendix
\section{Table of eta products}
\renewcommand{\arraystretch}{1.5}



The following table provides the list of the modular forms
$Z_{X,G}^{-1}$, expressed as eta products, for each of the 82 possible
symplectic actions of a group $G$ on a $K3$ surface $X$. We follow the
numbering in Xiao's list \cite{xiao1996galois}.
\begin{longtable}{|l|l|l|l|l|}
  \hline
Xiao $\# $ & $|G|$ & Singularities of $X/G$&  The modular form $Z_{X,G}^{-1}$ & Weight \\ 
  \hline
0 & 1 &  & $ \eta \left( \tau \right)   ^{24}$ & 12 \\ 
  1 & 2 & $8 A_{1}$ & $ \eta \left( 2\tau \right)   ^{8}  \eta \left( \tau \right)   ^{8}$ & 8 \\ 
  2 & 3 & $6 A_{2}$ & $ \eta \left( 3\tau \right)   ^{6}  \eta \left( \tau \right)   ^{6}$ & 6 \\ 
  3 & 4 & $12 A_{1}$ & $ \eta \left( 2\tau \right)   ^{12}$ & 6 \\ 
  4 & 4 & $2 A_{1} + 4 A_{3}$ & $ \eta \left( 4\tau \right)   ^{4}  \eta \left( 2\tau \right)   ^{2}  \eta \left( \tau \right)   ^{4}$ & 5 \\ 
  5 & 5 & $4 A_{4}$ & $ \eta \left( 5\tau \right)   ^{4}  \eta \left( \tau \right)   ^{4}$ & 4 \\ 
  6 & 6 & $8 A_{1} + 3 A_{2}$ & ${\frac {  \eta \left( 3\tau \right)   ^{8}  \eta \left( 2\tau \right)   ^{3}}{\eta \left( 6\tau \right) }}$ & 5 \\ 
  7 & 6 & $2 A_{1} + 2 A_{2} + 2 A_{5}$ & $ \eta \left( 6\tau \right)   ^{2}  \eta \left( 3\tau \right)   ^{2}  \eta \left( 2\tau \right)   ^{2} \mbox{}  \eta \left( \tau \right)   ^{2}$ & 4 \\ 
  8 & 7 & $3 A_{6}$ & $ \eta \left( 7\tau \right)   ^{3}  \eta \left( \tau \right)   ^{3}$ & 3 \\ 
  9 & 8 & $14 A_{1}$ & ${\frac {  \eta \left( 4\tau \right)   ^{14}}{  \eta \left( 8\tau \right)   ^{4}}}$ & 5 \\ 
  10 & 8 & $9 A_{1} + 2 A_{3}$ & ${\frac {  \eta \left( 4\tau \right)   ^{9}  \eta \left( 2\tau \right)   ^{2}}{  \eta \left( 8\tau \right)   ^{2}}}$ & 9/2 \\ 
  11 & 8 & $4 A_{1} + 4 A_{3}$ & $ \eta \left( 4\tau \right)   ^{4}  \eta \left( 2\tau \right)   ^{4}$ & 4 \\ 
  12 & 8 & $3 A_{3} + 2 D_{4}$ & ${\frac {  \eta \left( \tau \right)   ^{2}  \eta \left( 4\tau \right)   ^{6}}{\eta \left( 2\tau \right) }}$ & 7/2 \\ 
  13 & 8 & $ A_{1} + 4 D_{4}$ & ${\frac {  \eta \left( 4\tau \right)   ^{13}  \eta \left( \tau \right)   ^{4}}{  \eta \left( 8\tau \right)   ^{2} \mbox{}  \eta \left( 2\tau \right)   ^{8}}}$ & 7/2 \\ 
  14 & 8 & $ A_{1} +  A_{3} + 2 A_{7}$ & $ \eta \left( 8\tau \right)   ^{2}\eta \left( 4\tau \right) \eta \left( 2\tau \right)   \eta \left( \tau \right)   ^{2}$ & 3 \\ 
  15 & 9 & $8 A_{2}$ & $ \eta \left( 3\tau \right)   ^{8}$ & 4 \\ 
  16 & 10 & $8 A_{1} + 2 A_{4}$ & ${\frac {  \eta \left( 5\tau \right)   ^{8}  \eta \left( 2\tau \right)   ^{2}}{  \eta \left( 10\tau \right)   ^{2}}}$ & 4 \\ 
  17 & 12 & $4 A_{1} + 6 A_{2}$ & ${\frac {  \eta \left( 6\tau \right)   ^{4}  \eta \left( 4\tau \right)   ^{6}}{  \eta \left( 12\tau \right)   ^{2}}}$ & 4 \\ 
  18 & 12 & $9 A_{1} +  A_{2} +  A_{5}$ & ${\frac {  \eta \left( 6\tau \right)   ^{9}\eta \left( 4\tau \right) \eta \left( 2\tau \right) }{  \eta \left( 12\tau \right)   ^{3}}}$ & 4 \\ 
  19 & 12 & $3 A_{1} + 3 A_{5}$ & $ \eta \left( 6\tau \right)   ^{3}  \eta \left( 2\tau \right)   ^{3}$ & 3 \\ 
  20 & 12 & $ A_{2} + 2 A_{3} + 2 D_{5}$ & ${\frac {  \eta \left( 4\tau \right)   ^{3}  \eta \left( 3\tau \right)   ^{2}  \eta \left( \tau \right)   ^{2} \mbox{}  \eta \left( 6\tau \right)   ^{4}}{\eta \left( 12\tau \right)   \eta \left( 2\tau \right)   ^{4}}}$ & 3 \\ 
  21 & 16 & $15 A_{1}$ & ${\frac {  \eta \left( 8\tau \right)   ^{15}}{  \eta \left( 16\tau \right)   ^{6}}}$ & 9/2 \\ 
  22 & 16 & $10 A_{1} + 2 A_{3}$ & ${\frac {  \eta \left( 8\tau \right)   ^{10}  \eta \left( 4\tau \right)   ^{2}}{  \eta \left( 16\tau \right)   ^{4}}}$ & 4 \\ 
  23 & 16 & $5 A_{1} + 4 A_{3}$ & ${\frac {  \eta \left( 8\tau \right)   ^{5}  \eta \left( 4\tau \right)   ^{4}}{  \eta \left( 16\tau \right)   ^{2}}}$ & 7/2 \\ 
  24 & 16 & $6 A_{1} +  A_{3} + 2 D_{4}$ & ${\frac {  \eta \left( 8\tau \right)   ^{12}  \eta \left( 2\tau \right)   ^{2}}{  \eta \left( 16\tau \right)   ^{4} \mbox{}  \eta \left( 4\tau \right)   ^{3}}}$ & 7/2 \\ 
  25 & 16 & $6 A_{3}$ & $ \eta \left( 4\tau \right)   ^{6}$ & 3 \\ 
  26 & 16 & $4 A_{1} +  A_{3} +  A_{7} +  D_{4}$ & ${\frac {  \eta \left( 8\tau \right)   ^{7}  \eta \left( 2\tau \right)   ^{2}}{  \eta \left( 16\tau \right)   ^{2} \mbox{}\eta \left( 4\tau \right) }}$ & 3 \\ 
  27 & 16 & $2 A_{1} + 4 D_{4}$ & ${\frac {  \eta \left( 8\tau \right)   ^{14}  \eta \left( 2\tau \right)   ^{4}}{  \eta \left( 4\tau \right)   ^{8} \mbox{}  \eta \left( 16\tau \right)   ^{4}}}$ & 3 \\ 
  28 & 16 & $2 A_{1} +  A_{3} + 2 A_{7}$ & $ \eta \left( 8\tau \right)   ^{2}\eta \left( 4\tau \right)   \eta \left( 2\tau \right)   ^{2}$ & 5/2 \\ 
  29 & 16 & $ A_{3} +  D_{4} + 2 D_{6}$ & ${\frac {\eta \left( 4\tau \right)   \eta \left( 8\tau \right)   ^{7}  \eta \left( \tau \right)   ^{2}}{  \eta \left( 16\tau \right)   ^{2} \mbox{}  \eta \left( 2\tau \right)   ^{3}}}$ & 5/2 \\ 
  30 & 18 & $8 A_{1} + 4 A_{2}$ & ${\frac {  \eta \left( 9\tau \right)   ^{8}  \eta \left( 6\tau \right)   ^{4}}{  \eta \left( 18\tau \right)   ^{4}}}$ & 4 \\ 
  31 & 18 & $2 A_{1} + 3 A_{2} + 2 A_{5}$ & ${\frac {  \eta \left( 9\tau \right)   ^{2}  \eta \left( 6\tau \right)   ^{3}  \eta \left( 3\tau \right)   ^{2} \mbox{}}{\eta \left( 18\tau \right) }}$ & 3 \\ 
  32 & 20 & $2 A_{1} + 4 A_{3} +  A_{4}$ & ${\frac {  \eta \left( 10\tau \right)   ^{2}  \eta \left( 5\tau \right)   ^{4}\eta \left( 4\tau \right) }{\eta \left( 20 \mbox{}\tau \right) }}$ & 3 \\ 
  33 & 21 & $6 A_{2} +  A_{6}$ & ${\frac {  \eta \left( 7\tau \right)   ^{6}\eta \left( 3\tau \right) }{\eta \left( 21\tau \right) }}$ & 3 \\ 
  34 & 24 & $5 A_{1} + 3 A_{2} + 2 A_{3}$ & ${\frac {  \eta \left( 12\tau \right)   ^{5}  \eta \left( 8\tau \right)   ^{3}  \eta \left( 6\tau \right)   ^{2} \mbox{}}{  \eta \left( 24\tau \right)   ^{3}}}$ & 7/2 \\ 
  35 & 24 & $4 A_{1} + 2 A_{2} + 2 A_{5}$ & ${\frac {  \eta \left( 12\tau \right)   ^{4}  \eta \left( 8\tau \right)   ^{2}  \eta \left( 4\tau \right)   ^{2} \mbox{}}{  \eta \left( 24\tau \right)   ^{2}}}$ & 3 \\ 
  36 & 24 & $5 A_{1} +  A_{3} +  A_{5} +  D_{5}$ & ${\frac {  \eta \left( 12\tau \right)   ^{7}\eta \left( 6\tau \right) \eta \left( 2\tau \right) \eta \left( 8\tau \right) }{  \eta \left( 24\tau \right)   ^{3} \mbox{}\eta \left( 4\tau \right) }}$ & 3 \\ 
  37 & 24 & $2 A_{2} +  A_{5} +  D_{4} +  E_{6}$ & ${\frac {  \eta \left( 8\tau \right)   ^{4}\eta \left( 4\tau \right) \eta \left( 3\tau \right)   \eta \left( 12\tau \right)   ^{4} \mbox{}\eta \left( \tau \right) }{  \eta \left( 6\tau \right)   ^{2}  \eta \left( 24\tau \right)   ^{2}  \eta \left( 2\tau \right)   ^{2}}}$ & 5/2 \\ 
  38 & 24 & $2 A_{2} +  A_{3} + 2 E_{6}$ & ${\frac {  \eta \left( 8\tau \right)   ^{6}\eta \left( 6\tau \right)   \eta \left( \tau \right)   ^{2}  \eta \left( 12\tau \right)   ^{2} \mbox{}}{  \eta \left( 2\tau \right)   ^{4}  \eta \left( 24\tau \right)   ^{2}}}$ & 5/2 \\ 
  39 & 32 & $8 A_{1} + 3 A_{3}$ & ${\frac {  \eta \left( 16\tau \right)   ^{8}  \eta \left( 8\tau \right)   ^{3}}{  \eta \left( 32\tau \right)   ^{4}}}$ & 7/2 \\ 
  40 & 32 & $9 A_{1} + 2 D_{4}$ & ${\frac {  \eta \left( 16\tau \right)   ^{15}  \eta \left( 4\tau \right)   ^{2}}{  \eta \left( 32\tau \right)   ^{6} \mbox{}  \eta \left( 8\tau \right)   ^{4}}}$ & 7/2 \\ 
  41 & 32 & $3 A_{1} + 5 A_{3}$ & ${\frac {  \eta \left( 16\tau \right)   ^{3}  \eta \left( 8\tau \right)   ^{5}}{  \eta \left( 32\tau \right)   ^{2}}}$ & 3 \\ 
  42 & 32 & $4 A_{1} + 2 A_{3} + 2 D_{4}$ & ${\frac {  \eta \left( 16\tau \right)   ^{10}  \eta \left( 4\tau \right)   ^{2}}{  \eta \left( 32\tau \right)   ^{4} \mbox{}  \eta \left( 8\tau \right)   ^{2}}}$ & 3 \\ 
  43 & 32 & $5 A_{1} + 2 A_{7}$ & ${\frac {  \eta \left( 16\tau \right)   ^{5}  \eta \left( 4\tau \right)   ^{2}}{  \eta \left( 32\tau \right)   ^{2}}}$ & 5/2 \\ 
  44 & 32 & $2 A_{1} + 2 A_{3} +  A_{7} +  D_{4}$ & ${\frac {  \eta \left( 16\tau \right)   ^{5}  \eta \left( 4\tau \right)   ^{2}}{  \eta \left( 32\tau \right)   ^{2}}}$ & 5/2 \\ 
  45 & 32 & $3 A_{1} +  D_{4} + 2 D_{6}$ & ${\frac {  \eta \left( 16\tau \right)   ^{10}  \eta \left( 2\tau \right)   ^{2}}{  \eta \left( 32\tau \right)   ^{4} \mbox{}  \eta \left( 4\tau \right)   ^{3}}}$ & 5/2 \\ 
  46 & 36 & $2 A_{1} + 2 A_{2} + 4 A_{3}$ & ${\frac {  \eta \left( 18\tau \right)   ^{2}  \eta \left( 12\tau \right)   ^{2}  \eta \left( 9\tau \right)   ^{4} \mbox{}}{  \eta \left( 36\tau \right)   ^{2}}}$ & 3 \\ 
  47 & 36 & $ A_{1} + 6 A_{2} +  A_{5}$ & ${\frac {\eta \left( 18\tau \right)   \eta \left( 12\tau \right)   ^{6}\eta \left( 6\tau \right) }{  \eta \left( 36 \mbox{}\tau \right)   ^{2}}}$ & 3 \\ 
  48 & 36 & $6 A_{1} +  A_{2} + 2 A_{5}$ & ${\frac {  \eta \left( 18\tau \right)   ^{6}\eta \left( 12\tau \right)   \eta \left( 6\tau \right)   ^{2}}{  \eta \left( 36 \mbox{}\tau \right)   ^{3}}}$ & 3 \\ 
  49 & 48 & $5 A_{1} + 6 A_{2}$ & ${\frac {  \eta \left( 24\tau \right)   ^{5}  \eta \left( 16\tau \right)   ^{6}}{  \eta \left( 48\tau \right)   ^{4}}}$ & 7/2 \\ 
  50 & 48 & $6 A_{2} + 2 A_{3}$ & ${\frac {  \eta \left( 16\tau \right)   ^{6}  \eta \left( 12\tau \right)   ^{2}}{  \eta \left( 48\tau \right)   ^{2}}}$ & 3 \\ 
  51 & 48 & $5 A_{1} +  A_{2} + 2 A_{3} +  A_{5}$ & ${\frac {  \eta \left( 24\tau \right)   ^{5}\eta \left( 16\tau \right)   \eta \left( 12\tau \right)   ^{2} \mbox{}\eta \left( 8\tau \right) }{  \eta \left( 48\tau \right)   ^{3}}}$ & 3 \\ 
  52 & 48 & $4 A_{1} + 3 A_{5}$ & ${\frac {  \eta \left( 24\tau \right)   ^{4}  \eta \left( 8\tau \right)   ^{3}}{  \eta \left( 48\tau \right)   ^{2}}}$ & 5/2 \\ 
  53 & 48 & $ A_{1} +  A_{2} + 2 A_{3} + 2 D_{5}$ & ${\frac {  \eta \left( 24\tau \right)   ^{5}  \eta \left( 16\tau \right)   ^{3}  \eta \left( 12\tau \right)   ^{2} \mbox{}  \eta \left( 4\tau \right)   ^{2}}{  \eta \left( 48\tau \right)   ^{3}  \eta \left( 8\tau \right)   ^{4}}}$ & 5/2 \\ 
  54 & 48 & $4 A_{1} +  A_{2} +  A_{7} +  E_{6}$ & ${\frac {  \eta \left( 24\tau \right)   ^{5}  \eta \left( 16\tau \right)   ^{3}\eta \left( 6\tau \right)  \mbox{}\eta \left( 2\tau \right) }{  \eta \left( 48\tau \right)   ^{3}  \eta \left( 4\tau \right)   ^{2}}}$ & 5/2 \\ 
  55 & 60 & $4 A_{1} + 3 A_{2} + 2 A_{4}$ & ${\frac {  \eta \left( 30\tau \right)   ^{4}  \eta \left( 20\tau \right)   ^{3}  \eta \left( 12\tau \right)   ^{2} \mbox{}}{  \eta \left( 60\tau \right)   ^{3}}}$ & 3 \\ 
  56 & 64 & $5 A_{1} + 3 A_{3} +  D_{4}$ & ${\frac {  \eta \left( 32\tau \right)   ^{8}\eta \left( 16\tau \right) \eta \left( 8\tau \right) }{  \eta \left( 64 \mbox{}\tau \right)   ^{4}}}$ & 3 \\ 
  57 & 64 & $6 A_{1} + 3 D_{4}$ & ${\frac {  \eta \left( 32\tau \right)   ^{15}  \eta \left( 8\tau \right)   ^{3}}{  \eta \left( 64\tau \right)   ^{6} \mbox{}  \eta \left( 16\tau \right)   ^{6}}}$ & 3 \\ 
  58 & 64 & $3 A_{1} + 3 A_{3} +  A_{7}$ & ${\frac {  \eta \left( 32\tau \right)   ^{3}  \eta \left( 16\tau \right)   ^{3}\eta \left( 8\tau \right)  \mbox{}}{  \eta \left( 64\tau \right)   ^{2}}}$ & 5/2 \\ 
  59 & 64 & $5 A_{3} +  D_{4}$ & ${\frac {  \eta \left( 32\tau \right)   ^{3}  \eta \left( 16\tau \right)   ^{3}\eta \left( 8\tau \right)  \mbox{}}{  \eta \left( 64\tau \right)   ^{2}}}$ & 5/2 \\ 
  60 & 64 & $4 A_{1} +  A_{3} + 2 D_{6}$ & ${\frac {  \eta \left( 32\tau \right)   ^{8}  \eta \left( 16\tau \right)   ^{3}  \eta \left( 4\tau \right)   ^{2} \mbox{}}{  \eta \left( 64\tau \right)   ^{4}  \eta \left( 8\tau \right)   ^{4}}}$ & 5/2 \\ 
  61 & 72 & $4 A_{1} + 3 A_{2} +  A_{3} +  D_{5}$ & ${\frac {  \eta \left( 36\tau \right)   ^{6}  \eta \left( 24\tau \right)   ^{4}\eta \left( 18\tau \right)  \mbox{}\eta \left( 6\tau \right) }{  \eta \left( 72\tau \right)   ^{4}  \eta \left( 12\tau \right)   ^{2}}}$ & 3 \\ 
  62 & 72 & $3 A_{1} + 2 A_{3} + 2 A_{5}$ & ${\frac {  \eta \left( 36\tau \right)   ^{3}  \eta \left( 18\tau \right)   ^{2}  \eta \left( 12\tau \right)   ^{2} \mbox{}}{  \eta \left( 72\tau \right)   ^{2}}}$ & 5/2 \\ 
  63 & 72 & $ A_{2} + 3 A_{3} + 2 D_{4}$ & ${\frac {\eta \left( 24\tau \right)   \eta \left( 9\tau \right)   ^{2}  \eta \left( 36\tau \right)   ^{6}}{  \eta \left( 72\tau \right)   ^{3} \mbox{}\eta \left( 18\tau \right) }}$ & 5/2 \\ 
  64 & 80 & $3 A_{1} + 4 A_{4}$ & ${\frac {  \eta \left( 40\tau \right)   ^{3}  \eta \left( 16\tau \right)   ^{4}}{  \eta \left( 80\tau \right)   ^{2}}}$ & 5/2 \\ 
  65 & 96 & $3 A_{1} + 3 A_{2} + 3 A_{3}$ & ${\frac {  \eta \left( 48\tau \right)   ^{3}  \eta \left( 32\tau \right)   ^{3}  \eta \left( 24\tau \right)   ^{3} \mbox{}}{  \eta \left( 96\tau \right)   ^{3}}}$ & 3 \\ 
  66 & 96 & $2 A_{1} + 2 A_{2} +  A_{3} + 2 A_{5}$ & ${\frac {  \eta \left( 48\tau \right)   ^{2}  \eta \left( 32\tau \right)   ^{2}\eta \left( 24\tau \right)  \mbox{}  \eta \left( 16\tau \right)   ^{2}}{  \eta \left( 96\tau \right)   ^{2}}}$ & 5/2 \\ 
  67 & 96 & $2 A_{1} + 3 A_{2} +  A_{7} +  D_{4}$ & ${\frac {  \eta \left( 48\tau \right)   ^{5}  \eta \left( 32\tau \right)   ^{3}  \eta \left( 12\tau \right)   ^{2} \mbox{}}{  \eta \left( 96\tau \right)   ^{3}  \eta \left( 24\tau \right)   ^{2}}}$ & 5/2 \\ 
  68 & 96 & $3 A_{1} + 2 A_{3} +  A_{5} +  D_{5}$ & ${\frac {  \eta \left( 48\tau \right)   ^{5}  \eta \left( 24\tau \right)   ^{2}\eta \left( 8\tau \right)  \mbox{}\eta \left( 32\tau \right) }{  \eta \left( 96\tau \right)   ^{3}\eta \left( 16\tau \right) }}$ & 5/2 \\ 
  69 & 96 & $3 A_{1} + 2 A_{2} + 2 E_{6}$ & ${\frac {  \eta \left( 48\tau \right)   ^{5}  \eta \left( 32\tau \right)   ^{6}  \eta \left( 4\tau \right)   ^{2} \mbox{}}{  \eta \left( 96\tau \right)   ^{4}  \eta \left( 8\tau \right)   ^{4}}}$ & 5/2 \\ 
  70 & 120 & $2 A_{1} +  A_{2} + 2 A_{3} +  A_{4} +  A_{5}$ & ${\frac {  \eta \left( 60\tau \right)   ^{2}\eta \left( 40\tau \right)   \eta \left( 30\tau \right)   ^{2} \mbox{}\eta \left( 24\tau \right) \eta \left( 20\tau \right) }{  \eta \left( 120\tau \right)   ^{2}}}$ & 5/2 \\ 
  71 & 128 & $3 A_{1} + 2 A_{3} +  D_{4} +  D_{6}$ & ${\frac {  \eta \left( 64\tau \right)   ^{8}\eta \left( 32\tau \right) \eta \left( 8\tau \right) }{  \eta \left( 128\tau \right)   ^{4} \mbox{}\eta \left( 16\tau \right) }}$ & 5/2 \\ 
  72 & 144 & $ A_{1} + 4 A_{2} + 2 A_{5}$ & ${\frac {\eta \left( 72\tau \right)   \eta \left( 48\tau \right)   ^{4}  \eta \left( 24\tau \right)   ^{2}}{  \eta \left( 144 \mbox{}\tau \right)   ^{2}}}$ & 5/2 \\ 
  73 & 160 & $2 A_{1} + 3 A_{3} + 2 A_{4}$ & ${\frac {  \eta \left( 80\tau \right)   ^{2}  \eta \left( 40\tau \right)   ^{3}  \eta \left( 32\tau \right)   ^{2} \mbox{}}{  \eta \left( 160\tau \right)   ^{2}}}$ & 5/2 \\ 
  74 & 168 & $ A_{1} + 3 A_{2} + 2 A_{3} +  A_{6}$ & ${\frac {\eta \left( 84\tau \right)   \eta \left( 56\tau \right)   ^{3}  \eta \left( 42\tau \right)   ^{2} \mbox{}\eta \left( 24\tau \right) }{  \eta \left( 168\tau \right)   ^{2}}}$ & 5/2 \\ 
  75 & 192 & $2 A_{1} + 6 A_{2} +  D_{4}$ & ${\frac {  \eta \left( 96\tau \right)   ^{5}  \eta \left( 64\tau \right)   ^{6}\eta \left( 24\tau \right)  \mbox{}}{  \eta \left( 192\tau \right)   ^{4}  \eta \left( 48\tau \right)   ^{2}}}$ & 3 \\ 
  76 & 192 & $2 A_{1} +  A_{2} + 2 A_{3} +  A_{5} +  D_{4}$ & ${\frac {  \eta \left( 96\tau \right)   ^{5}\eta \left( 64\tau \right) \eta \left( 32\tau \right)  \mbox{}\eta \left( 24\tau \right) }{  \eta \left( 192\tau \right)   ^{3}}}$ & 5/2 \\ 
  77 & 192 & $2 A_{1} +  A_{2} + 3 A_{3} +  E_{6}$ & ${\frac {  \eta \left( 96\tau \right)   ^{3}  \eta \left( 64\tau \right)   ^{3}  \eta \left( 48\tau \right)   ^{3} \mbox{}\eta \left( 8\tau \right) }{  \eta \left( 192\tau \right)   ^{3}  \eta \left( 16\tau \right)   ^{2}}}$ & 5/2 \\ 
  78 & 288 & $2 A_{1} + 2 A_{2} +  A_{3} + 2 D_{5}$ & ${\frac {  \eta \left( 144\tau \right)   ^{6}  \eta \left( 96\tau \right)   ^{4} \mbox{}\eta \left( 72\tau \right)   \eta \left( 24\tau \right)   ^{2}}{  \eta \left( 288\tau \right)   ^{4}  \eta \left( 48\tau \right)   ^{4}}}$ & 5/2 \\ 
  79 & 360 & $ A_{1} + 2 A_{2} + 2 A_{3} + 2 A_{4}$ & ${\frac {\eta \left( 180\tau \right)   \eta \left( 120\tau \right)   ^{2}  \eta \left( 90\tau \right)   ^{2} \mbox{}  \eta \left( 72\tau \right)   ^{2}}{  \eta \left( 360\tau \right)   ^{2}}}$ & 5/2 \\ 
  80 & 384 & $ A_{1} + 3 A_{2} + 2 A_{3} +  D_{6}$ & ${\frac {  \eta \left( 192\tau \right)   ^{3}  \eta \left( 128\tau \right)   ^{3} \mbox{}  \eta \left( 96\tau \right)   ^{3}\eta \left( 24\tau \right) }{  \eta \left( 384\tau \right)   ^{3}  \eta \left( 48\tau \right)   ^{2}}}$ & 5/2 \\ 
  81 & 960 & $ A_{1} + 3 A_{2} + 2 A_{4} +  D_{4}$ & ${\frac {  \eta \left( 480\tau \right)   ^{4}  \eta \left( 320\tau \right)   ^{3} \mbox{}  \eta \left( 192\tau \right)   ^{2}\eta \left( 120\tau \right) }{  \eta \left( 960\tau \right)   ^{3}  \eta \left( 240\tau \right)   ^{2}}}$ & 5/2 \\ 
   \hline
\caption{Table of the modular forms $Z_{X,G}^{-1}$ for all symplectic
$G$ actions. } \label{table: list of eta products}
\end{longtable}


\bibliography{k3hilb}
\bibliographystyle{plain}

\end{document}

