\documentclass{article}

\title{$G$-fixed Hilbert schemes on $K3$ surfaces and modular forms}
\author{Jim Bryan and \'{A}d\'{a}m~Gyenge} 
\date{\today}

%\address{
%Department of Mathematics\\
%University of British Columbia \\
%Room 121, 1984 Mathematics Road  \\
%Vancouver, B.C., Canada V6T 1Z2  
%}

%\usepackage{diagrams}

\usepackage{amsmath}
\usepackage{bm}
\usepackage{verbatim}
\usepackage{amsmath,amsthm,amsfonts}
\usepackage{amssymb}
\usepackage{times}
\usepackage{longtable}
%\usepackage{amstex}


\newtheorem{theorem}{Theorem}%[section]
\newtheorem{proposition}[theorem]{Proposition}
\newtheorem{conjecture}[theorem]{Conjecture}
\newtheorem{lemma}[theorem]{Lemma}
\newtheorem{corollary}[theorem]{Corollary}
\newtheorem{convention}{Convention}[theorem]

\theoremstyle{definition}

\newtheorem{def-theorem}[theorem]{Definition-Theorem}
\newtheorem{remark}[theorem]{Remark}
\newtheorem{definition}[theorem]{Definition}
\newtheorem{example}[theorem]{Example}


\newcommand{\half}{\frac{1}{2}}
\newcommand{\CC} {{\mathbb C}}          % complex numbers
\newcommand{\NN} {{\mathbb N}}		% natural numbers
\newcommand{\RR} {{\mathbb R}}		% real numbers
\newcommand{\ZZ} {{\mathbb Z}}		% integers
\newcommand{\QQ} {{\mathbb Q}}		% rationals
\newcommand{\PP}{\mathbb{P}}
\newcommand{\HH}{\mathbb{H}}
\newcommand{\LL}{\mathbb{L}}
\renewcommand{\O}{\mathcal{O}}
%\renewcommand{\top}{\mathrm{t}}
\renewcommand{\top}{\,\mathsf{t}}




\newcommand{\mvec}{\bm{m}}
\newcommand{\zetavec}{\bm{\zeta }}
\newcommand{\dvec}{\bm{d }}
\newcommand{\uvec}{\bm{u }}
\newcommand{\vvec}{\bm{v }}

%\newcommand{\mvec}{\vec{m}}
%\newcommand{\zetavec}{\vec{\zeta }}
%\newcommand{\dvec}{\vec{d }}
%\newcommand{\uvec}{\vec{u }}
%\newcommand{\vvec}{\vec{v }}


\newcommand{\Hom}{\operatorname{Hom}}
\newcommand{\Ker}{\operatorname{Ker}}
\newcommand{\End}{\operatorname{End}}
\newcommand{\Tr}{\operatorname{tr}}
\newcommand{\tr}{\operatorname{tr}}
\newcommand{\Sym}{\operatorname{Sym}}
\newcommand{\Coker}{\operatorname{Coker}}
\newcommand{\im}{\operatorname{Im}}
\newcommand{\Hilb}{\operatorname{Hilb}}

\begin{document}

\maketitle

\tableofcontents

\begin{abstract}
Let $X$ be a complex $K3$ surface with an effective action of a group
$G$ which preserves the holomorphic symplectic form. Let 
\[
Z_{X,G}(q) = \sum_{n=0}^{\infty} e(\Hilb^{n}(X)^{G})q^{n-1}
\]
be the generating function for the Euler characteristics of Hilbert
scheme of $G$-invariant length $n$ subschemes. We show that its
reciprocal, $Z_{X,G}(q)^{-1}$ is the Fourier expansion of a modular
cusp form of weight $\half e(X/G)$ and index $|G|$. We give an
explicit formula for $Z_{X,G}$ in terms of the Dedekind eta function
for all 82 possible $(X,G)$. 
\end{abstract}



%\markboth{???}  {???}



%\tableofcontents
%\pagebreak


\section{Introduction}

Let $X$ be a complex $K3$ surface with an effective action of a group
$G$ which preserves the holomorphic symplectic form. Mukai showed that
such $G$ are precisely the subgroups of the Mathieu group
$M_{23}\subset M_{24}$ such that the induced action on the set
$\{1,\dots ,24 \}$ has at least five orbits
\cite{mukai1988finite}. Xiao classified all possible actions into
82 possible topological types of the quotient $X/G$ \cite{xiao1996galois}.

The \emph{$G$-fixed Hilbert scheme} of $X$ parameterizes
$G$-invariant length $n$ subschemes $Z\subset X$. It can be
identified with the $G$-fixed point locus in the Hilbert scheme of
points: 
\[
\Hilb^{n}(X)^{G} \subset \Hilb^{n}(X)
\]

We define the corresponding \emph{$G$-fixed partition function} of
$X$ by
\[
Z_{X,G}(q) = \sum_{n=0}^{\infty} e\left(\Hilb^{n}(X)^{G} \right) q^{n-1} 
\]
where $e(-)$ is topological Euler characteristic.

Throughout this paper we set
\[
q=\exp\left(2\pi i \tau  \right)
\]
so that we may regard $Z_{X,G}$ as a function of $\tau \in \HH$ where
$\HH$ is the upper half-plane.

Our main result is the following:

\begin{theorem}
\label{thm:main}
The function $Z_{X,G}(q)^{-1}$ is a modular cusp form\footnote{See
Section \S~\ref{sec: modular forms}  for notation and definitions regarding modular
forms.} of weight $\half e(X/G)$ for the congruence subgroup
$\Gamma_{0}(|G|)$.
\end{theorem}
   

Our theorem specializes in the case where $G$ is the trivial group to
a famous result of G\"ottsche \cite{gottsche1990betti}. The case where
$G$ is a cyclic group was proved in \cite{bryan2018chl}. One can
interpret our result as an instance of the Vafa-Witten S-duality
conjecture for the orbifold $[X/G]$. The partition function
$Z_{X,G}(q)$ also has an interpretation in enumerative geometry: its coefficients count $G$-invariant rational curves on
$X$ (see \S~\ref{subsec: enumerative applications}), generalizing the
famous Yau-Zaslow formula.

We also give an explicit formula for $Z_{X,G}(q)$ in terms of the
Dedekind eta function
\[
\eta (\tau ) = q^{\frac{1}{24}}\prod_{n=1}^{\infty} (1-q^{n})
\]
as follows. Let $p_{1},\dots ,p_{r}$ be the singular points of $X/G$
and let $G_{1},\dots ,G_{r}$ be the corresponding stabilizer subgroups
of $G$. The singular points are necessarily of ADE type: they are
locally given by $\CC^{2}/G_{i}$ where $G_{i}\subset SU(2)$. Finite
subgroups of $SU(2)$ have an ADE classification and we let
$\Delta_{1},\dots ,\Delta_{r}$ denote the corresponding ADE root
systems.

For any finite subgroup $G_{\Delta}\subset SU(2)$ with associated root
system $\Delta$ we define the \emph{local $G_{\Delta }$-fixed
partition function} by
\[
Z_{\Delta} (q) = \sum_{n=0}^{\infty}
e\left(\Hilb^{n}(\CC^{2})^{G_{\Delta}} \right) \, q^{n-\frac{1}{24}} .
\]
We will prove in Lemma~\ref{lem: local series as theta/eta}  that
\[
Z_{\Delta}(q) =\frac{\theta_{\Delta}(\tau)}{\eta (k\tau )^{N+1}}
\]
where $\theta_{\Delta}(\tau )$ is a shifted theta function for the
root lattice of $\Delta$, $N$ is the rank of the root system, and
$k=|G_{\Delta}|$.




The 82 possible collections of ADE root systems $\Delta_{1},\dots
,\Delta_{r}$ associated to $(X,G)$ a $K3$ surface with a symplectic
$G$ action, are given in Table~\ref{table: list of eta products} and
we note that $\Delta_{i}\in \{A_{1},\dots
,A_{7},D_{4},D_{5},D_{6},E_{6} \}$. We let $k=|G|$, $k_{i}=|G_{i}|$,
and
\[
a = e(X/G) - r=\frac{24}{k}-\sum_{i=1}^{r} \frac{1}{k_{i}}.
\]
Our method to prove Theorem~\ref{thm:main} is based on the next result, which expresses the global series $Z_{X,G}(q)$ as an eta product\footnote{See
	Section \S~\ref{sec:modformIIetaprod}  for notation and definitions regarding eta products.} 
\begin{theorem}\label{thm: eta product formula for Z}
With the above notation we have
\[
Z_{X,G}(q) = \eta^{-a}(k\tau )\prod_{i=1}^{r}
Z_{\Delta_{i}}\left(\frac{k\tau}{k_{i}} \right)
\]
where 
\begin{align*}
Z_{A_{n}} (\tau ) &=\frac{1}{\eta (\tau )}, \quad n\geq 1\\
Z_{D_{n}} (\tau ) &=\frac{\eta^{2}(2\tau )\eta((4n-8)\tau )}{\eta (\tau )\eta (4\tau )\eta^{2}((2n-4)\tau )}, \quad 4\leq n\leq 6\\
Z_{E_{6}} (\tau ) &=\frac{\eta^{2}(2\tau )\eta(24\tau )}{\eta (\tau
)\eta^{2} (8\tau )\eta(12\tau )}
\end{align*}
\end{theorem}
We conjecture that the formula for $Z_{D_{n}}$ holds for all $n\geq 4$
and we provide explicit conjectural formulas for $Z_{E_{7}}$ and
$Z_{E_{8}}$ (see Conjecture~\ref{conj: eta product for theta
function}). In Table~\ref{table: list of eta products} we have listed
explictly the eta product of the modular form $(Z_{X,G})^{-1}$ for all
82 possible cases of $(X,G)$.

Having obtained explicit eta product expressions for $Z_{X,G}(q)$ in all 82 possible
cases allows us to make several observational corollaries:

\begin{corollary}\label{cor: if G is a subgp of E then Zinv is a Hecke
eigenform}
If $G$ is a finite subgroup of an elliptic curve $E$, i.e. $G$ is
isomorphic to a product of one or two cyclic groups, then
$Z_{X,G}(q)^{-1}$ is a Hecke eigenform. On table~\ref{table: list of
eta products} these are the 13 cases having Xiao number in the set
$\{0,1,2,3,4,5,7,8,11,14,15,19,25 \}$. Moreover, in each of these
cases, the dimension of the Hecke eigenspace is one. 
\end{corollary}

We remark that in these cases, we may form a Calabi-Yau threefold
called a CHL model by taking the free group quotient
\[
(X\times E)/G
\]
Then the partition function $Z_{X,G}(q)$ gives the (modified)
Donaldson-Thomas invariants of $(X\times E)/G$ in curve classes which
are degree zero over $X/G$ (see \cite{bryan2018chl}).
For any eta product expression of a modular form, one may easily compute the
order of vanishing (or pole) at any of the cusps
\cite[Cor~2.2]{kohler2011eta}. Performing this computation on the 82
cases yields the following

\begin{corollary}\label{cor: vanishing at cusps}
The modular form $Z_{X,G}(q)^{-1}$ always vanishes with order 1 at the cusps
$i\infty$ and $0$. Moreover,
\begin{itemize}
\item $Z_{X,G}(q)^{-1}$ vanishes at all cusps except for the eleven
cases with Xiao number in the set $\{13,20,27,29,37,38,45,53,54,60,69 \}$.
\item $Z_{X,G}(q)^{-1}$ is holomorphic except for the two cases with
Xiao number 38 or 69, which have poles at the cusps $1/2$ and $1/8$
respectively. These are precisely the cases where $X/G$ has two
singularities of type $E_{6}$.
\end{itemize}
\end{corollary}

%In fact, the first statement of Corollary \ref{cor: vanishing at cusps} can be proven without computer.

\subsection{Enumerative applications}\label{subsec: enumerative applications}

We have already mentioned above the enumerative application to the CHL
Calabi-Yau threefold $(X\times E)/G$ in the case where $G\subset E$ is
a finite subgroup of an elliptic curve. Another application is the
following generalization of the Yau-Zaslow formula counting rational
curves on $X$.

Let $X\subset \PP^{g}$ be an embedding obtained from a $G$-equivariant
ample line bundle $L$ with $c_{1}(L)$ a primitive class of square
$2g-2$. Then the coefficient of $q^{g-1}$ in $Z_{X,G}(q)$ is the
number of hyperplane sections which are $G$-invariant rational curves,
counted with multiplicity.

\dots add discussion of the above. Formulate as proposition?


\subsection{Structure of the paper}

I'm not sure we really need to outline the paper here, but we could.


\section{The local partition functions}\label{sec: local partition functions}

The classical McKay correspondence associates an ADE root system
$\Delta$ to any finite subgroup $ G_{\Delta}\subset SU(2)$. Using
the work of Nakajima \cite{nakajima2002geometric}, the partition function of the
Euler characteristics of the Hilbert scheme of points on the stack
quotient $[\CC^{2}/G_{\Delta}]$ was computed explicitly in
\cite{gyenge2015euler} in terms of the root data of $\Delta$.

The local partition functions $Z_{\Delta}(q)$ considered in this paper
are obtained from a specialization of the partition functions of the
stack $[\CC^{2} /G_{\Delta}]$ and in this section, we use this to express
$Z_{\Delta}(q)$ in terms of a shifted theta function for the root
lattice of $\Delta$.

A zero-dimensional substack $Z\subset [\CC^{2}/G_{\Delta}]$ may be
regarded as a $G_{\Delta}$ invariant, zero-dimensional subscheme of
$\CC^{2}$. Consequently, we may identify the Hilbert scheme of points
on the stack $[\CC^{2}/G_{\Delta}]$ with the $G_{\Delta}$ fixed locus
of the Hilbert scheme of points on $\CC^{2}$: 
\[
\Hilb \left([\CC^{2}/G_{\Delta}] \right) = \Hilb
(\CC^{2})^{G_{\Delta}} .
\]

This Hilbert scheme has components indexed by representations $\rho$
of $G_{\Delta}$ as follows
\begin{equation*}
\Hilb^{\rho} \left([\CC^{2}/G_{\Delta}] \right) = \left\{ Z\subset
\CC^{2}, \text{ $Z$ is $G_{\Delta}$ invariant and $H^{0}(\O_{Z})\cong
\rho $} \right\}.
\end{equation*}

Let $\{\rho_{0},\dotsc ,\rho_{N} \}$ be the irreducible
representations of $G_{\Delta}$ where $\rho_{0}$ is the trivial
representation. We note that $N$ is also the rank of $\Delta$. We
define
\[
Z_{[\CC^{2}/G_{\Delta}]} (q_{0},\dotsc ,q_{N}) = \sum_{m_{0},\dotsc
,m_{N}=0}^{\infty} e\left(\Hilb^{m_{0}\rho_{0}+\dotsb
+m_{N}\rho_{M}}([\CC^{2}/G_{\Delta}]) \right) q_{0}^{m_{0}}\dotsb
q_{N}^{m_{N}} .
\]

Recall that our local partition function $Z_{\Delta}(q)$ is defined by
\[
Z_{\Delta}(q) = \sum_{n=0}^{\infty}
e\left(\Hilb^{n}(\CC^{2})^{G_{\Delta}} \right) q^{n-\frac{1}{24}}. 
\]
We then readily see that
\[
Z_{\Delta}(q) = q^{\frac{-1}{24}}\cdot  Z_{[\CC^{2}/G_{\Delta}]}
(q_{0},\dotsc ,q_{N})|_{q_{i}=q^{d_{i}}}
\]
where
\[
d_{i} =\dim \rho_{i}.
\]

The following formula is given explicitely in \cite[Thm~1.3]{gyenge2015euler}, but its content is already present in the work of Nakajima \cite{nakajima2002geometric}:
\begin{theorem} \label{thm: Zorbifold formula}
Let $C_{\Delta}$ be the Cartan matrix of the root system $\Delta$,
then 
\[
Z_{[\CC^{2}/G_{\Delta}]} (q_{0},\dotsc ,q_{N}) = \prod_{m=1}^{\infty}
(1-Q^{m})^{-N-1} \cdot \sum_{\mvec \in \ZZ^{N}} q_{1}^{m_{1}}\dotsb
q_{N}^{m_{N}} \cdot Q^{\half \mvec^{\top}\cdot C_{\Delta}\cdot \mvec}
\]
where $Q=q_{0}^{d_{0}}q_{1}^{d_{1}}\dotsb d_{N}^{d_{N}}$.
\end{theorem}
We note that under the specialization $q_{i}=q^{d_{i}}$, 
\[
Q=q^{d_{0}^{2}+\dotsb +d_{N}^{2}} = q^{k}
\]
where $k=|G|$ is the order of the group $G$.

We then obtain
\[
Z_{\Delta}(q) = q^{\frac{-1}{24}}\cdot
\prod_{m=1}^{\infty}(1-q^{km})^{-N-1} \cdot \sum_{\mvec \in \ZZ^{N}}
q^{\mvec^{\top}\cdot \dvec} \cdot q^{\frac{k}{2}\mvec^{\top}\cdot C_{\Delta}\cdot \mvec}
\]
where $\dvec =(d_{1},\dotsc ,d_{N})$.

Let $M_{\Delta}$ be the root lattice of $\Delta$ which we identify
with $\ZZ^{N}$ via the basis given by $\alpha_{1},\dotsc ,\alpha_{N}$,
the simple positive roots of $\Delta$. Under this identification, the
standard Weyl invariant bilinear form is given by
\[
(\uvec |\vvec ) = \uvec^{\top}\cdot C_{\Delta}\cdot \vvec .
\]
We define
\[
\zetavec = C_{\Delta}^{-1} \cdot \dvec 
\]
so that 
\[
\mvec^{\top}\cdot \dvec = \mvec^{\top}\cdot C_{\Delta} \cdot \zetavec
= (\mvec |\zetavec ).
\]
We may then write
\begin{align*}
Z_{\Delta}(q)& = q^{\frac{-1}{24}}\cdot
\prod_{m=1}^{\infty}(1-q^{km})^{-N-1}\cdot \sum _{\mvec \in M_{\Delta}} q^{(\mvec |\zetavec )+\frac{k}{2}(\mvec |\mvec)}\\
&= q^{A} \cdot \left(q^{\frac{k}{24}}\prod_{m=1}^{\infty}(1-q^{km})
\right)^{-N-1} \cdot \sum_{\mvec \in M_{\Delta}} q^{\frac{k}{2}\left(\mvec +\frac{1}{k}\zetavec |\mvec +\frac{1}{k}\zetavec  \right)}\\
& = q^{A}\cdot  \eta (k\tau )^{-N-1}\cdot  \theta_{\Delta} (\tau )
\end{align*}
where 
\[
A \quad = \quad \frac{-1}{24} + \frac{k(N+1)}{24} - \frac{1}{2k}(\zetavec
|\zetavec ) \quad = \quad \frac{k(N+1)-1}{24} - \frac{1}{2k}\dvec^{\top}\cdot
C_{\Delta}^{-1} \cdot \dvec 
\]
and $\theta_{\Delta}(\tau )$ is the shifted theta function:
\[
\theta_{\Delta}(\tau ) = \sum_{\mvec \in M_{\Delta}} q^{\frac{k}{2}\left(\mvec +\frac{1}{k}\zetavec |\mvec +\frac{1}{k}\zetavec  \right)}
\]
where as throughout this paper we have identified $q=\exp\left(2\pi i\tau  \right)$.

In section~???, lemma~??? we will prove that the identity $A=0$ holds
for all $\Delta$. Hence we obtain the following:
\begin{lemma}\label{lem: local series as theta/eta} The local series
$Z_{\Delta}(q)$ is given by
\[
Z_{\Delta}(q) = \frac{\theta_{\Delta}(\tau )}{\eta (k\tau )^{N+1}}. 
\]
In particular, $Z_{\Delta}(q)$  is a modular form of weight $1/2$.
\end{lemma}
The particular congruence subgroup and more precise modularity properties of $Z_{\Delta}(q)$ will be calculated in Section...

We make the following conjecture which provides explicit eta product
expressions for the theta function $\theta_{\Delta}(\tau )$.

\begin{conjecture}\label{conj: eta product for theta function}
$\theta_{\Delta}(\tau )$ is given by
\begin{align}\label{eqn: theta function eta products}
\theta_{A_{n}}(\tau ) &= \frac{\eta^{n+1}((n+1)\tau )}{\eta (\tau )} ,\quad n\geq 1\\
\theta_{D_{n}}(\tau ) &= \frac{\eta^{2}(2\tau )\,  \eta^{n+2}((4n-8)\tau )}{\eta (\tau )\,  \eta (4\tau )\,  \eta^{2}((2n-4)\tau )} ,\quad n\geq 4\\
\theta_{E_{6}}(\tau ) & = \frac{\eta^{2}(2\tau )\, \eta^{8}(24\tau
)}{\eta (\tau )\, \eta^{2}(8\tau )\, \eta (12\tau )} ,\\
\theta_{E_{7}}(\tau ) & = \frac{\eta^{2}(2\tau )\, \eta^{9}(48\tau
)}{\eta (\tau )\, \eta(12\tau )\, \eta(16\tau )\, \eta (24\tau )} ,\\
\theta_{E_{8}}(\tau ) & = \frac{\eta^{2}(2\tau )\, \eta^{10}(120\tau
)}{\eta (\tau )\, \eta(24\tau )\, \eta(40\tau )\, \eta (60\tau )}.
\end{align}
\end{conjecture}

Since both sides of the above equations are explicit modular forms of
known weight and index, any given formula can be proved with a finite
number of computations. We will give a uniform geometric proof in the
$A_{n}$ case for all $n$ below, and we will give computational proofs
for the cases of $D_{4}$, $D_{5}$, $D_{6}$, and $E_{6}$ (Theorem
\ref{}). These are the only cases needed for our application to K3
surfaces. It would be desirable to have a purely root theoretic way of
writing the eta products and a pure root theoretic proof of the
conjecture. \footnote{One uniform way to write the conjecture is as
follows. Consider $E,F,V,M\in \NN$ where $E$, $F$, and $V$ each divide
$M$ and satisfy $E+F+V =M+2$. Such tuples $(E,F,V,M)$ can be seen to
have an $ADE$ classification so that to each $ADE$ root system
$\Delta$, we have an associated $(E,F,V,M)$ Then our conjecture is
equivalent to
\[
Z_{\Delta}(q) = \frac{\eta^{2}(2\tau )\eta (2M\tau )}{\eta (\tau )\eta
(2E\tau )\eta (2F\tau )\eta (2V\tau )}. 
\]   }



\begin{theorem}\label{thm: conjecture holds for An}
Conjecture~\ref{conj: eta product for theta function} holds for the
case of $A_{n}$. 
\end{theorem}

\begin{proof}
By Lemma~\ref{lem: local series as theta/eta} , the conjecture is
equivalent to the statment that 
\[
Z_{A_{n}}(q) = \frac{1}{\eta (\tau )}
\]
which is in turn equivalent to the statement
\[
\sum_{n=0}^{\infty} e\left(\Hilb (\CC^{2})^{\ZZ /(n+1)} \right)
\,q^{n} = \prod_{m=1}^{\infty} (1-q^{m})^{-1}.
\]
The action of $\ZZ /(n+1)$ on $\CC^{2}$ commutes with the action of
$\CC^{*}\times \CC^{*}$ on $\CC^{2}$ and consequently, the Euler
characteristics on the left hand side may be computed by counting
the $\CC^{*}\times \CC^{*}$-fixed subschemes, namely those given by
monomial ideals. Such subschemes of length $n$ have a well known
bijection with integer partitions of $n$, whose generating function is
given by the right hand side.
\end{proof}




\section{The global series}\label{sec: the global series}

Recall that $p_{1},\dotsc ,p_{r}\in X/G$ are the singular points of
$X/G$ with corresponding stabilizer subgroups $G_{i}\subset G$ of
order $k_{i}$ and ADE type $\Delta_{i}$. Let $\{x_{i}^{1},\dotsc
,x_{i}^{k/k_{i}} \}$ be the orbit of $G$ in $X$ corresponding to the
point $p_{i}$ (recall that $k=|G|$).  We may stratify $\Hilb (X)^{G}$
according to the orbit types of subscheme as follows:

Let $Z\subset X$ be a $G$-invariant subscheme of length $nk$ whose
support lies on free orbits. Then $Z$ determines and is determined by
a length $n$ subscheme of 
\[
(X/G)^{o}  = X/G\setminus \{p_{1},\dotsc ,p_{r} \},
\]
i.e. a point
in $\Hilb^{n}((X/G)^{o})$.

On the other hand, suppose $Z\subset X$ is a $G$-invariant subscheme
of length $\frac{nk}{k_{i}}$ supported on the orbit
$\{x_{i}^{1},\dotsc ,x_{i}^{k/k_{i}} \}$. Then $Z$ determines and is
determined by the length $n$ component of $Z$ supported on a formal
neighborhood of one of the points, say $x_{i}^{1}$. Choosing a
$G_{i}$-equivariant isomorphism of the formal neighborhood of
$x_{i}^{1}$ in $X$ with the formal neighborhood of the origin in
$\CC^{2}$, we see that $Z$ determines and is determined by a point in
$\Hilb_{0}^{n}(\CC^{2})^{G_{i}}$, where $\Hilb_{0}^{n}(\CC^{2})\subset
\Hilb^{n}(\CC^{2})$ is the punctual Hilbert scheme parameterizing
subschemes supported on a formal neighborhood of the origin in
$\CC^{2}$.

By decomposing an arbitrary $G$-invariant subscheme into components of
the above types, we obtain a stratification of $\Hilb (X)^{G}$ into
strata which are given by products of $\Hilb ((X/G)^{o})$ and
$\Hilb_{0}(\CC^{2})^{G_{1}},\dotsc ,\Hilb_{0}(\CC^{2})^{G_{r}}$. Then
using the fact that Euler characteristic is additive under
stratifications and multiplicative under products, we arrive at the
following equation of generating functions:
\begin{align}\label{eqn: stratification formula for sum e(hilb(X)G)}
\nonumber\sum_{n=0}^{\infty} e\left(\Hilb^{n}(X)^{G} \right)\, q^{n}
=&\left(\sum_{n=0}^{\infty} e\left(\Hilb^{n}((X/G)^{o}) \right)\,
q^{kn} \right)\\
& \cdot \prod_{i=1}^{r}\left( \sum_{n=0}^{\infty}
e\left(\Hilb_{0}^{n}(\CC^{2})^{G_{i}} \right) \, q^{\frac{nk}{k_{i}}}
\right) .
\end{align}

As in the introduction, let $a = e(X/G)-r=e\left((X/G)^{o}
\right)$. Then by G\"ottsche's formula \cite{gottsche1990betti},
\begin{align*}
\sum_{n=0}^{\infty} e\left(\Hilb^{n}((X/G)^{0} \right) q^{kn} &=
\prod_{m=1}^{\infty} (1-q^{km})^{-a}\\
&= q^{\frac{ak}{24}} \cdot \eta (k\tau )^{-a}. 
\end{align*}

We also note that $e\left(\Hilb_{0}^{n}(\CC^{2})^{G_{i}}
\right)=e\left(\Hilb^{n}(\CC^{2})^{G_{i}} \right)$ since the natural
$\CC^{*}$ action on both $\Hilb_{0}^{n}(\CC^{2})^{G_{i}}$ and
$\Hilb^{n}(\CC^{2})^{G_{i}}$ have the same fixed points. Thus we may
write
\begin{align*}
\sum_{n=0}^{\infty} e\left(\Hilb_{0}^{n}(\CC^{2})^{G_{i}} \right) \,
q^{\frac{nk}{k_{i}}} &= \sum_{n=0}^{\infty} e\left(\Hilb^{n}(\CC^{2})^{G_{i}} \right) \,
q^{\frac{nk}{k_{i}}}\\
&= q^{\frac{k}{24k_{i}}} \cdot Z_{\Delta_{i}} \left(\frac{k\tau}{k_{i}}
\right) .
\end{align*}

Multiplying equation~\eqref{eqn: stratification formula for sum
e(hilb(X)G)} by $q^{-1}$ and substituting the above formulas, we find
that

\[
Z_{X,G}(q) = q^{-1 +\frac{ak}{24} + \sum \frac{k}{24k_{i}} } \cdot 
\eta (k\tau )^{-a}\cdot 
\prod_{i=1}^{r}Z_{\Delta_{i}}\left(\frac{k\tau}{k_{i}} \right) .
\]

The exponent of $q$ in the above equation is zero as is readily seen
from the following Euler characteristic calculation:
\begin{align*}
24 = e(X) &= e\left(X-\cup_{i=1}^{r} \{x_{i}^{1},\dotsc
,x_{i}^{k/k_{i}} \} \right) + \sum_{i=1}^{r} \frac{k}{k_{i}} \\
&= k \cdot e\left((X/G)^{o} \right) + \sum_{i=1}^{r} \frac{k}{k_{i}} \\
&= k\cdot a + \sum_{i=1}^{r}\frac{k}{k_{i}} 
\end{align*}

We have thus proved that the first equation in Theorem~\ref{thm: eta
product formula for Z} always holds. Then since the only root systems
which can occur as singularities of $X/G$ are of type $A_{n}$ or
$D_{4}$, $D_{5}$, $D_{6}$, or $E_{6}$, we may now use
Theorem~\ref{thm: conjecture holds for An} and Theorem~\ref{} to
complete the proof of Theorem~\ref{thm: eta product formula for
Z}. \qed

\section{Modular forms I: Theta functions}\label{sec: modular forms}

\subsection{Modular forms with multiplier systems and congruence subgroups}

%In our discussion we will need to work with modular forms of half-integer weight. 
Fix a
subgroup $\Gamma$ of finite index in $\mathrm{SL_2}( \ZZ)$, a function $\vartheta \colon
\Gamma \to  \CC^{\ast}$ with $|\vartheta(A)|=1$ for $A \in \Gamma$, and a half-integer $k$.
Then a holomorphic function $f \colon \mathbb{H} \to  \CC$ is said to transform as a modular
form of weight $k$ with the multiplier system $\vartheta$ for $\Gamma$ if
\[ f\left( \frac{a\tau+b}{c\tau+d}\right)=\vartheta(A)(c\tau+d)^k f(\tau) \quad 
\textrm{ for all } A=\begin{pmatrix}
a & b \\ c & d
\end{pmatrix} \in \Gamma. \]
When $k$ is not an integer, $(c\tau+d)^k$ is understood to be a principal value. If moreover
$f$ is holomorphic at all the cusps of $\Gamma$ on $ \QQ \cup \{i \infty\}$, then $f$ is said
to be a modular form. We will denote the space of modular forms of weight $k$ and multiplier
systems $\chi$ for $\Gamma$ by $M_k(\Gamma, \vartheta)$.

We will need the following congruence subgroups:
\[  \Gamma_0(N) = \left\{ \begin{pmatrix} a & b \\ c & d \end{pmatrix} \in 
\mathrm{SL_2}( \ZZ) : \begin{pmatrix} a & b \\ c & d \end{pmatrix} \equiv 
\begin{pmatrix} \ast & \ast \\ 0 & \ast \end{pmatrix} \;(\mathrm{mod}\;N) \right\}; \]
\[  \Gamma_1(N) = \left\{ \begin{pmatrix} a & b \\ c & d \end{pmatrix} \in 
\mathrm{SL_2}( \ZZ) : \begin{pmatrix} a & b \\ c & d \end{pmatrix} \equiv 
\begin{pmatrix} 1 & \ast \\ 0 & 1 \end{pmatrix} \;(\mathrm{mod}\;N) \right\}; \]
\[  \Gamma^0(N) = \left\{ \begin{pmatrix} a & b \\ c & d \end{pmatrix} \in 
\mathrm{SL_2}( \ZZ) : \begin{pmatrix} a & b \\ c & d \end{pmatrix} \equiv 
\begin{pmatrix} \ast & 0 \\ \ast & \ast \end{pmatrix} \;(\mathrm{mod}\;N) \right\}. \]
%It is automatic, that there is the following containment of subgroups:
%\[ \Gamma(N) \subset \Gamma_1(N) \subset \Gamma_0(N). \]
Moreover, for a divisor $m|N$ let us introduce the following subset of $\mathrm{SL_2}( \ZZ)$:
\[  \Gamma(N,m) = \left\{ \begin{pmatrix} a & b \\ c & d \end{pmatrix}: 
a \equiv \pm 1 \;(\mathrm{mod}\;N/m), b \equiv 0 \;(\mathrm{mod}\;N/m), 
c \equiv 0 \;(\mathrm{mod}\;N) \right\}. \]

\begin{lemma} $\Gamma(N,m)$ is a congruence subgroup of $\mathrm{SL_2}( \ZZ)$.
\end{lemma}
\begin{proof}
Since $ad-bc=1$ and $c\equiv 0  \;(\mathrm{mod}\;N/m)$, we have that 
$ad\equiv 1  \;(\mathrm{mod}\;N/m)$. This and $a \equiv \pm 1 \;(\mathrm{mod}\;N/m)$ 
implies that $a\equiv d  \;(\mathrm{mod}\;N/m)$. Hence the inverse
\[\begin{pmatrix} d & -b \\ -c & a \end{pmatrix} \textrm{ of }
 \begin{pmatrix} a & b \\ c & d \end{pmatrix} \in \Gamma(N,m)\]
belongs to $\Gamma(N,m)$ as well. 
Similarly, the product
\[ 
\begin{pmatrix} a_1a_2+b_1c_2 & a_1b_2+b_1d_2 \\ a_2c_1+c_2d_1 & b_2c_1+d_1d_2 \end{pmatrix}
\]
 of two matrices 
\[\begin{pmatrix} a_1 & b_1 \\ c_1 & d_1 \end{pmatrix}, 
\begin{pmatrix} a_2 & b_2 \\ c_2 & d_2 \end{pmatrix} \in \Gamma(N,m) \]
is contained in $\Gamma(N,m)$. Finally, it follows from the definition of 
$\Gamma(N,m)$ that it contains $\Gamma(N)$.
\end{proof}
\begin{lemma}
	\label{lem:indgNmsl2}
	The index of $\Gamma(N,m)$ inside $\mathrm{SL}_2( \ZZ)$ is
	\begin{equation*} [\mathrm{SL}_2( \ZZ) \, : \, \Gamma(N,m) ] = \begin{cases} 
	2N^2\prod_{p \mid N}\left(1-\frac{1}{p^2} \right),\textrm{ if } N/m =2 \\
	2N^2\prod_{p \mid N}\left(1-\frac{1}{p^2} \right) 
	\prod_{p \mid N/m }\left(\frac{p}{p-1} \right),\textrm{ if } N/m >2.
	\end{cases}
	\end{equation*}
\end{lemma}
The proof of this statement will be given in Appendix \ref{sec:app0}.

%For a congruence subgroup $\Gamma$ we will denote by $ M_k(\Gamma, \vartheta)$ 
%the space of modular forms with respect to the multiplier system $\vartheta$. 


\subsection{Theta functions}

%\subsubsection{Theta functions of lattices}

As mentioned in Section \ref{sec: local partition functions} the numerator of the partition function in Theorem \ref{thm: Zorbifold formula}
is (up to a rational exponent factor of $q$) a shifted theta function of the root lattice of the corresponding 
finite type Lie algebra. We now investigate the modularity properties of such shifted theta functions. For the notations and many results  we will refer to
\cite[Chapter 14]{cohen2017modular}.

Let $L\cong  \ZZ^n$ be an $n$ dimensional lattice equipped with a real quadratic 
form $Q$, which we suppose to be integral and positive definite. That is $Q(L) \in  \ZZ^+$. 
The associated symmetric bilinear form is obtained as
\[ B(\mathbf{a},\mathbf{b})= Q(\mathbf{a}+\mathbf{b})-Q(\mathbf{a})-Q(\mathbf{b}). \]
Then
\[ Q(\mathbf{a})=\frac{1}{2}B(\mathbf{a},\mathbf{a}). \]
Let moreover 
\[ L^{\ast} = \{ \mathbf{b}\in  \RR^n : B(\mathbf{a},\mathbf{b}) \in  
\ZZ \textrm{ for all } \mathbf{a} \in L \}\] 
be the dual lattice to $L$ in $L\otimes  \RR \cong  \RR^n$ with respect to $Q$.

Let $\mathbf{a} \in L^{\ast}$. The theta function associated with the lattice $L$ and 
shifted by $\mathbf{a}$ is defined as
\begin{equation}
\Theta_{\mathbf{a}}(\tau)= \sum_{m \in L+\mathbf{a}} q^{Q(\mathbf{m})} .
\label{eq:thetadef}
\end{equation}
\begin{remark} 
In \cite[Definition 14.3.3]{cohen2017modular} a much more general class of theta functions 
is introduced. We will use the conventions of \cite[Example 14.2.5]{cohen2017modular}. In 
our case the spherical polynomial $P(X)$ which appears in \cite[14.2.5]{cohen2017modular} is 
equal to the constant function 1. Moreover, the number $k$ appearing in  \cite[Example 
14.2.5 and Definition 14.3.3]{cohen2017modular} is equal to $n$ in our case. In particular, 
$k \equiv n \;(\mathrm{mod}\;2)$.
\end{remark}


Recall that $L^{\ast}$ is the dual lattice to $L$ with respect to $B$. 
Then $L \subset L^{\ast}$ always, and there is a smallest positive integer $N$ for which   
\begin{equation} 
\label{eq:Ndef}
NL^{\ast} \subset L \textrm{ and }NQ(\mathbf{a}) \in  \ZZ \textrm{ for all } 
\mathbf{a} \in L^{\ast}.\end{equation} 
This number $N$ is called the level in \cite[Definition 14.3.15]{cohen2017modular}.


Elements of a matrix $A \in \mathrm{SL}_2( \ZZ)$ will be denoted from now on as follows:
\begin{equation}
\label{eq:matrix}
A=\begin{pmatrix}
a & b \\ c & d
\end{pmatrix}.\end{equation}
Then $A \in \Gamma_0(N)$ if and only if
\[  cL^{\ast} \subset L \textrm{ and } cQ(\mathbf{a}) \in  
\ZZ \textrm{ for all } \mathbf{a} \in L^{\ast}. \]
We also introduce the symbol 
\begin{equation}
\label{eq:epsilondef}
\epsilon_d  = \begin{cases}
1 \quad \textrm{if } d\equiv 1 \; (\textrm{mod } 4) \\
i \quad \textrm{if } d\equiv -1 \; (\textrm{mod } 4). \\
\end{cases}
\end{equation}

\begin{theorem} 
	\label{thm:modthm}
	Let $A \in \Gamma_0(N)$ and  $D = \mathrm{det}(B)$.
	%Define $\varepsilon_M(0,1)=1, \varepsilon_M(0,-1)=(-i)^{p-q}$ and for $c\neq0$ set
	%\[ \varepsilon_L(c,d)=(d,(-1)^qD)_{\infty \varepsilon_d^{-n} \left(\frac{2c}{d})^n\right)\left(\frac{D}{d})\right) } \]
	%if $d$ is odd and $\varepsilon_L(c,d)=\varepsilon_L(c,d-c)$ if $d$ is even. %Here $\varepsilon_d=1$ if $d\cong 1 \; \mathrm(mod 4)$ and  
	Then
	\[ \Theta_{\mathbf{a}}(A\tau)= \vartheta(A)(c\tau+d)^{n/2} \Theta_{a\mathbf{a}}(\tau) \]
	for a multiplier system $\vartheta$ such that
	\[ \vartheta(A)=e^{2 \pi i ab Q(\mathbf{a})}
	 \left(\frac{D}{d}\right)\left(\epsilon^{-1}_d \left(\frac{2c}{d}\right)\right)^n, \]
	when $c \neq 0$ and $d$ is odd.
\end{theorem}
\begin{proof}
Corollary 14.3.8 and Theorem 14.3.11 from \cite{cohen2017modular} implies that
\[ \Theta_{\mathbf{a}}(A\tau)= (d,-1^{q}D)_{\infty}\vartheta(A)(c\tau+d)^{n/2}
\Theta_{a\mathbf{a}}(\tau), \]
where $q$ is the number of negative eigenvalues of $Q$, $(d,(-1)^{q}D)_{\infty}= -1$ if 
$d<0$ and $(-1)^{q}D < 0$, and $(d,(-1)^{q}D)_{\infty}= 1$ otherwise. The form $Q$ is 
positive definite. Hence $q=0$ and $D>0$. In turn $(d,(-1)^{q}D)_{\infty}= 1$ always.
\end{proof}


Let $s$ be the smallest integer, such $s\mathbf{a} \in L$. This is, in general, not the same
as $N$, but $s|N$ always. 
\begin{corollary}
\label{cor:mult} 
Suppose that in \eqref{eq:matrix} the element $a \equiv \pm 1 \; \mathrm(mod\; s)$.
Then 
\[ \Theta_{\mathbf{a}}(A\tau)= \vartheta(A)(c\tau+d)^{n/2} \Theta_{\mathbf{a}}(\tau), \]
where $\vartheta$ is as in Theorem \ref{thm:modthm}.
\end{corollary}
\begin{proof}
Since $a \equiv \pm 1 \; \mathrm(mod\; s)$, $a \mathbf{a} \equiv \pm\mathbf{a} \; 
\mathrm(mod\; L)$. It follows from the definition \eqref{eq:thetadef} that 
$\Theta_{\mathbf{a}}(\tau)$ depends only on the class of $\mathbf{a}$ modulo $L$. 
Furthermore, since
\[ Q(\mathbf{m}-\mathbf{a})=Q(-\mathbf{m}+\mathbf{a}), \]
$\Theta_{-\mathbf{a}}(\tau)= \Theta_{\mathbf{a}}(\tau)$.
\end{proof}
\begin{lemma} 
\label{lem:mult2}
Let $\Gamma \subset \Gamma_0(N)$ be a subgroup such that $a \equiv \pm 1 \; \mathrm(mod\; s)$ for all $A \in \Gamma$. Then $\Theta_{\mathbf{a}}(\tau) \in M_{n/2}(\Gamma, \vartheta)$, where $\vartheta$ is as in Theorem \ref{thm:modthm}.
\end{lemma}
\begin{proof} Due to Corollary \ref{cor:mult}, $\Theta_{\mathbf{a}}(\tau)$ 
transforms as a modular form with the multiplier system $\vartheta$ for the elements of $\Gamma$. 
	
Showing that it is holomorphic at the cusps is analogous to the proof of 
\cite[Corollary 14.3.16]{cohen2017modular}. By \cite[Theorem 14.3.7]{cohen2017modular} when 
an element $A \in \textrm{SL}_2( \ZZ)$ acts on the upper half plane, 
$\Theta_{\mathbf{a}}(A\tau)$ decomposes into a finite linear combination:
\[ \Theta_{\mathbf{a}}(A\tau)=\sum_{\mathbf{b} \in 
L^{\ast}/L}c_{\mathbf{b}}(c\tau+d)^{n/2}\Theta_{\mathbf{b},k}(\tau). \]
It is known that the group $\textrm{SL}_2( \ZZ)$ acts transitively on the cusps of 
$\Gamma$.  Hence, to prove that $\Theta_{\mathbf{a}}(\tau)$ is holomorphic at all the cusps 
of $\Gamma$ it is enough to show that $\Theta_{\mathbf{b}}(\tau)$ is holomorphic as $\tau 
\to i\infty$ for any $\mathbf{b} \in L^{\ast}/L$. Since the bilinear form is positive 
definite, $Q(\mathbf{b})>0$ for any $\mathbf{b} \neq 0$. Therefore the only term in
\[\Theta_{\mathbf{b}}(\tau)= \sum_{\mathbf{m} \in L+\mathbf{b}}
 e^{\pi i \tau k Q(\mathbf{m})}\]
which could not tend to 0  as $\tau \to i\infty$ is the one with $\mathbf{m}=-\mathbf{b}$. 
This term exists only if $\mathbf{b} \in L$, and in this case the limit is 1. The theorem 
follows.
\end{proof}

\begin{corollary}
\label{cor:mult3} 
$\Theta_{\mathbf{a}}(\tau) \in M_{n/2}(\Gamma(N,s), \vartheta)$
for a multiplier system $\vartheta$ such that
\[ \vartheta(A)= \left(\frac{D}{d}\right)\left(\epsilon^{-1}_d 
\left(\frac{2c}{d}\right)\right)^n, \]
when $c \neq 0$ and $d$ is odd.
\end{corollary}
\begin{proof}
For the elements of $\Gamma(N,s)$, $a \equiv \pm 1 \; \mathrm(mod\; s)$. Thus the conditions 
of Lemma \ref{lem:mult2} are satisfied. Moreover, since $b \equiv 0 \; \mathrm(mod\; s)$, 
$ab Q(\mathbf{a})$ is an integer. This implies that the term 
\[ e^{2 \pi i ab Q(\mathbf{a})} \]
in Theorem \ref{thm:modthm} is equal to 1.
\end{proof}

\begin{remark} 
\label{rmk:multsyseven}	
Suppose that the rank $n$ of the lattice $L$ is even.
Then the multiplier system in Corollary \ref{cor:mult3} simplifies as
\[ \vartheta(A)= \left( \frac{(-1)^{n/2} D}{d} \right), \]
because	$\epsilon_d^{-2}=-1$ and $\left( \frac{2c}{d}\right)^2=1$.
\end{remark}

\subsection{Theta functions of root systems}

Let $\Delta$  be a root system of finite type, and let $L$ be its root lattice. 

If $\Delta$ is an irreducible root system of finite type and $B=(|)$, the standard invariant
form, then the level $N$ of the root lattice $L$ defined in in \eqref{eq:Ndef} is equal to
the number $m$ listed in Table \ref{table:congorder} \cite[page 261]{kac1994infinite}. In the
standard basis  $L\cong  \ZZ^n$
\[ B(\mathbf{a},\mathbf{b})=\mathbf{a}\cdot C_{\Delta} \cdot \mathbf{b}^{\top}, 
\quad Q(\mathbf{a})=\frac{1}{2}\mathbf{a}\cdot C_{\Delta} \cdot \mathbf{a}^{\top}, \]
and $D=\det(B)=|C_{\Delta}|$.


We now set instead $B=k(|)$, where $k=|G_{\Delta}|$. 
In this case the level $N$ of $L$ is $km$,
\[ B(\mathbf{a},\mathbf{b})=\mathbf{a}\cdot kC_{\Delta} \cdot \mathbf{b}^{\top}, 
\quad Q(\mathbf{a})=\frac{1}{2}\mathbf{a}\cdot kC_{\Delta} \cdot \mathbf{a}^{\top}, \]
and $D=\det(B)=k^n|C_{\Delta}|$.

\begin{lemma} Let $\mathbf{a} \in L^{\ast}$. Then
\[
\Theta_{\mathbf{a}}(\tau)= {(q^{k/2})^{\mathbf{a}^{t} \cdot C_\Delta \cdot \mathbf{a}}}
\cdot \sum_{ \mathbf{m}=(m_1,\dots,m_n) \in  \ZZ^n } {q^{ k({\mathbf{m}^ {\top}} \cdot 
\mathbf{a})
}} {(q^{k/2})^{{\mathbf{m}^ {\top}} \cdot C_\Delta \cdot \mathbf{m}}}
 \Big|_{q=\mathrm{e}^{2 \pi
i \tau}}.\]
	\label{lem:thetadef2}
\end{lemma}
\begin{proof}
	Since $L \cong  \ZZ^n $, one can rewrite \eqref{eq:thetadef}
	as
\[ \sum_{\mathbf{m} \in  \ZZ^n} (q^{k/2})^{(\mathbf{m+\mathbf{a}} | 
\mathbf{m+\mathbf{a}})} =
\sum_{\mathbf{m} \in  \ZZ^n} (q^{k/2})^{(\mathbf{m}|
\mathbf{m})+2(\mathbf{m}|\mathbf{a})+(\mathbf{a}|\mathbf{a})} 
.
\]
The pairing between $\mathbf{a} \in L^{\ast}$ and $\mathbf{m} \in L$ is just
\[(\mathbf{m}|\mathbf{a})=\sum_{i=1}^n a_i m_i = \mathbf{m}^ {\top} \cdot \mathbf{a}.\]
\end{proof}

We now fix a particular shift vector $\mathbf{a}$. Let us denote the standard basis of $L$ 
by $\{\alpha_1, \dots, \alpha_n\}$ and the corresponding dual basis of $L^{\ast}$ by 
$\{\omega_1, \dots, \omega_n\}$.
Let
\begin{equation} 
\label{eq:thetadef}
\theta=( \mathrm{dim} \rho_1 ,\dots, \mathrm{dim} \rho_n) = \sum_{i=1}^n 
(\mathrm{dim} \rho_i) \alpha_i \in L.\end{equation}
Our $\mathbf{a} \in L^{\ast}$ will be the dual of $\theta$ with respect to $k(|)$. 
Explicitely, this means that
\begin{equation}
\mathbf{a}= \frac{1}{k}\sum_{i=1}^n (\mathrm{dim} \rho_i) 
\omega_i=\sum_{i=1}^n a_i \alpha_i, 
\label{eq:zetasub}
\end{equation}
where $(a_1,\dots,a_n)=(kC_\Delta)^{-1} \cdot \theta$.
Finally, we introduce the notation
\[ \Theta_{\Delta}(\tau)=\Theta_{\mathbf{a}}(\tau). \]

The next statement follows immediately from Corollary \ref{cor:mult3}.
\begin{corollary}
\label{cor:thetazetamult} 
The function $\Theta_{\Delta}(\tau)$ is a modular form of weight $n/2$ for 
$\Gamma(km,k)$ with a multiplier system $\vartheta$ such that
\[ \vartheta(A)=\left(\frac{k^n|C_{\Delta}|}{d}\right)
\left(\epsilon^{-1}_d \left(\frac{2c}{d}\right)\right)^n, \]
when $c\neq 0$ and $d$ is odd.
\end{corollary}

\begin{proof}[{Proof of Proposition \ref{prop:locmod}}]
Let
\begin{equation}\mathbf{\zeta}= k\mathbf{a}=\sum_{i=1}^n (\mathrm{dim} \rho_i) 
\omega_i=\sum_{i=1}^n b_i \alpha_i,
\label{eq:zetasub}
\end{equation}
where $(b_1,\dots,b_n)=(C_\Delta)^{-1} \cdot \theta$.
Then
\[ Q(\mathbf{a})=\frac{k}{2}\left(\mathbf{a}^ {\top} \cdot C_\Delta 
\cdot \mathbf{a}\right)=  \frac{\zeta^ {\top} \cdot C_\Delta \cdot \zeta}{2k}. \]
Substituting this into the equation in Lemma \ref{lem:thetadef2} yields
\begin{equation*} 
\Theta_{ \Delta }(\tau )= q^{\frac{\zeta^ {\top} \cdot C_\Delta \cdot \zeta}{2k}}
\sum_{ \mathbf{m}=(m_1,\dots,m_n) \in  \ZZ^n } q^{\mathbf{m}^ {\top} \cdot 
( \mathrm{dim} \rho_1 ,\dots, \mathrm{dim} \rho_n) } 
(q^{k/2})^{\mathbf{m}^ {\top} \cdot C_\Delta \cdot \mathbf{m}}
. 
\end{equation*}
Up to the factor $q^{ \frac{\zeta^ {\top} \cdot C_\Delta \cdot \zeta}{2k}}$ this is exactly 
the numerator of $Z_{[ \CC^2/G_\Delta]}(q_0,\dots,q_n)$ appearing in Theorem 
\ref{thm:genfunct} when we substitute $q_i=q^{\mathrm{dim} \rho_i}$, $0 \leq i \leq n$. In 
the denominator of $Z_{[ \CC^2/G_\Delta]}(q_0,\dots,q_n)$, after the same substitution, a 
product of $n+1$ terms of 
\[\prod_{m=1}^{\infty}(1-q^{km})\] 
appears. As a consequence,
\[ q^{\frac{k(n+1)}{24} - \frac{\zeta^ {\top} \cdot C_\Delta \cdot \zeta}{2k}}
(Z_{[ \CC^2/G_\Delta]}(q))^{-1}  = 
\frac{(\eta(k\tau))^{n+1}}{\Theta_{\zeta/k ,k}(\tau )}. \]
The Dedeking eta function $\eta(\tau)$ is a modular form of weight $1/2$ for $\Gamma(1)$. 
Hence, $\eta(k\tau)$ is a modular form of weight $1/2$ for $\Gamma(k)$.
By Lemma \ref{lem:betalength} below,
\[ \frac{k(n+1)}{24} - \frac{\zeta^ {\top} \cdot C_\Delta \cdot \zeta}{2k}=\frac{1}{24}. \]
Hence,
\[ Z_{\Delta}(\tau)=q^{-\frac{1}{24}}Z_{[ \CC^2/G_\Delta]}(q)
 \]
is the quotient of two holomorphic modular forms. It transforms as a modular form for 
$\Gamma(km,k)\cap \Gamma(k)=\Gamma(km,k)=\Gamma(N,k)$ with weight 
\[ \frac{n+1}{2}-\frac{n}{2}=\frac{1}{2}. \]
\end{proof}
\begin{lemma} Let $\Delta$ be a simply laced root system, and let $\zeta$ be 
defined as in \eqref{eq:zetasub}. Then
\[ \frac{(\zeta|\zeta)}{2k}=\frac{\zeta^ {\top} \cdot C_\Delta 
\cdot \zeta}{2k}=\frac{(n+1)k-1}{24}. \]
\label{lem:betalength}
\end{lemma}
\begin{remark}Lemma \ref{lem:betalength} expresses the \emph{modular anomaly} of the 
numerator of $Z_{[ \CC^2/G_\Delta]}(q)$ (see \cite[12.7.5]{kac1994infinite}). It is proved 
in Appendix \ref{sec:app1} below. We have not found it in this generality in the literature, 
but in type $A$ it turns out to be another form of the ``strange formula'' of 
Freudenthal--de Vries \cite{freudenthal1969linear}:
\[ \frac{(\rho|\rho)}{2h}=\frac{\rho^ {\top} \cdot C_\Delta 
\cdot \rho}{2h}=\frac{\mathrm{dim} \mathfrak{g}_{\Delta}}{24}, \]
where $\rho$ is the sum of the positive roots of $\Delta$, $h$ is the (dual) Coxeter number, 
and $\mathfrak{g}_{\Delta}$ is the corresponding Lie algebra.
See Appendix \ref{sec:app11} for the details. We expect that the identity of Lemma 
\ref{lem:betalength} holds in the non-simply laced cases as well.
\end{remark}

Let $\Delta_1$ (resp. $\Delta_2$) be a root system of rank $n_1$ (resp. $n_2$). Denote by 
$G_{\Delta_1}$ $(resp. G_{\Delta_2})$ the corresponding finite group, whose order is 
$k_1=|G_{\Delta_1}|$ (resp. $k_2=|G_{\Delta_2}|$). Let 
$\theta_1$ (resp. $\theta_2$) be as in \eqref{eq:thetadef}.
Let
$\mathbf{a}_1$ (resp.  $\mathbf{a}_2$) be the vector dual to $\theta_1$ (resp. $\theta_2$) 
with respect to the form
$k_1(|)_1$ (resp. $k_2(|)_2$). We define
\[ \Theta_{\Delta_1 \oplus \Delta_2}(\tau)=\Theta_{\mathbf{a}_1 \oplus \mathbf{a}_2}(\tau),\]
where the right side is the theta function of the lattice $L_1 \oplus L_2$ equipped with the 
form $k_1(|)_1 \oplus k_2(|)_2$. The next statement is a straightforward calculation.

\begin{lemma}
\[ \Theta_{\Delta_1 \oplus \Delta_2}(\tau)=\Theta_{\Delta_1}(\tau)
\cdot \Theta_{\Delta_2}(\tau).\]
\end{lemma}

\begin{corollary} Let $\Delta$ be an irreducible, finite type root system.
\begin{enumerate}
\item If the rank of $\Delta$ is even, then 
$\Theta_{\Delta}(\tau) \in M_{n/2}(\Gamma(N,k), \vartheta)$, where $N=km$, and 
\[ \vartheta(A)= \left( \frac{(-1)^{n/2} k^n |C_{\Delta}|}{d} \right). \]
\item If the rank of $\Delta$ is odd and $\Delta$ is not of type A, then 
$\Theta_{\Delta\oplus A_1}(\tau) \in M_{(n+1)/2}(\Gamma(N,k), \vartheta)$, 
where $N=km$, and the multiplier system is
\[ \vartheta(A)= \left( \frac{(-1)^{(n+1)/2} 4k^n|C_{\Delta}|}{d} \right). \]
\end{enumerate}
\end{corollary}
\begin{proof} Part (1) follows from Remark \ref{rmk:multsyseven} and Corollary \ref{cor:thetazetamult}.
	
If $\Delta$ is not of type $A$, then $2|k$ and $8|km$. By Corollary \ref{cor:thetazetamult}, 
$\Theta_{\Delta}(\tau) $ is a modular form for $\Gamma(km,k)$, and  $\Theta_{A_1}(\tau)$ is 
a modular form for $\Gamma(8,2)$. Hence, their product is a modular form for $\Gamma(km,k)$. 
The formula of the multiplier system follows from Part (1) and from that $2|C_{A_1}|=4$.
\end{proof}

\begin{remark}
In Section \ref{sec:locetaprod} below we perform computer calculations. In the odd rank 
cases we found it better to work with $\Theta_{\Delta}(\tau)\cdot 
\Theta_{A_1}(\frac{k}{2}\tau)$. With the same reasoning as above, 
$\Theta_{A_1}(\frac{k}{2}\tau)$ is a modular form for $\Gamma(4k,k)$. Since in all non-type 
A, odd rank cases $4|m$, we have that $\Theta_{\Delta}(\tau)\cdot 
\Theta_{A_1}(\frac{k}{2}\tau)$ is a modular form for $\Gamma(km,k)$. Moreover, as the 
determinant of $kC_{\Delta} \oplus k C_{A_1}$ is $2k^{n+1}|C_{\Delta}|$, the multiplier 
system is
	\[ \vartheta(A)= \left( \frac{(-1)^{(n+1)/2} 2k^{n+1}|C_{\Delta}|}{d} \right). \]
\end{remark}

\section{Modular forms II: Eta products}
\label{sec:modformIIetaprod}

\subsection{Eta products}
An eta products is a finite product
\begin{equation} f(\tau)=\prod_m \eta(m \tau )^{a_m} 
\label{eq:etaproddef}
\end{equation}
where $m$ runs through a finite set of positive integers and the exponents
$a_m$ may take values from $\ZZ$. The least common multiple of all $m$ such that $a_m\neq 0$ will be denoted by $N$; it is called the minimum level of $f(\tau)$.

For a general eta quotient $f(\tau)$ as in \eqref{eq:etaproddef}, let $k=\sum_m{a_m}$. The expression $f(\tau)(\mathrm{d}\tau)^{k/2}$ transforms as a $k/2$-differential due to the transformation law of the Dedekind eta function. Since $\eta(\tau)$ is nonzero on $\mathbb{H}$, (quotients of) eta products never has finite poles. The only issue for an eta product to be a (possibly half-integral weight) modular form is whether the numerator vanishes to at least the same order as the denominator at each cusp. 

\begin{theorem}[{\cite[Theorem 3]{gordon1993multiplicative}}] 
	\label{thm:etaprodmodeven}	
	Let $f$ be an eta product as in \eqref{eq:etaproddef} such that $n=\sum_{m \mid N} a_m$ is even. Let $s=\prod_{m \mid N} m^{a_m}$,  $\frac{1}{24}\sum_{m \mid N} ma_m=c/e$ and $\frac{1}{24}\sum_{m \mid N} \frac{N}{m}a_m=c_0/e_0$, both in lowest terms. Then $f(\tau)$ is a modular form of weight $n/2$ for $\Gamma_0(Ne_0)\cap\Gamma^0(e)$ with the multiplier system defined by the Dirichlet character (mod $Ne_0$)
	\[ \gamma(A)=\left( \frac{(-1)^{n/2} s}{a} \right) \]
	for $a>0$, $gcd(a,6)=1$.
	%Then:
	%\begin{enumerate}
	%	\item $f(\tau)$ is a modular form of weight $n/2$ for $\Gamma_0(Ne_0)\cap\Gamma^0(e)$ with the multiplier system defined by the Dirichlet character (mod $Ne_0$)
	%	\[ \gamma(A)=\left( \frac{(-1)^{n/2} s}{a} \right) \]
	%	for $a>0$, $gcd(a,6)=1$.
	%	\item $F(\tau)=f(e\tau)$ is a modular form of weight $n/2$ for $\Gamma_0(Ne_0e)$ with a multiplier system defined by the Dirichlet character (mod $Ne_0e$) $\gamma(A)$ as in (1).
	%\end{enumerate}
\end{theorem}
\begin{remark}
	\label{rem:multsyst}
	\begin{enumerate}
		\item The fact that the $\gamma(A)$ values for $a>0$, $gcd(a,6)=1$ are enough to define a multiplier system follows from \cite[Lemma 3]{newman1959construction}, and the multiplier system was calculated originally in \cite[Theorem 1]{newman1959construction}.
		\item\label{it:multsyst2} Since $N|c$ and $ad-bc=1$, we have that $ad \equiv 1  \;(\textrm{mod }m)$ for all $m|N$. This means that
		\[ \left(\frac{a}{m}\right)=\left(\frac{d}{m}\right),\]
		or equivalently, that 
		\[ \left(\frac{m}{a}\right)=\left(\frac{m}{d}\right).\]
		Hence, the multiplier system in Theorem \ref{thm:etaprodmodeven} can also be written as 
		\[ \gamma(A)=\left( \frac{(-1)^{n/2} s}{d} \right). \]
	\end{enumerate}
\end{remark}

% It was not stated there explicitely but it also assumed in the background of the proof that $gcd(a,m)=1$ for all $m$ with nonzero exponent. But again, due to \cite[Lemma 3]{newman1959construction} this is enough.

The content of Theorem \ref{thm:etaprodmodeven} is explained in \cite[Section 1]{gordon1989multiplicative}. 
In the case when $\sum_{m \mid N} \frac{N}{m}a_m \equiv 0 \;(\textrm{mod }24)$, $f(\tau)$ has an integral order at 0. If this condition is not satisfied for $N$, it can be guaranteed by replacing $N$ with $Ne_0$. In effect this \emph{widens} the cusp of $\Gamma_0(N)$ at 0 by a factor of $e_0$. Similarly,  $\sum_{m \mid N} ma_m \equiv 0 \;(\textrm{mod }24)$ if and only if $f(\tau)$ has an integral order at the cusp at $i\infty$. If this is not the case, widening the cusp $\Gamma_0(Ne_0)$ at $i\infty$ can be achieved by passing to the subgroup $\Gamma_0^0(Ne_0,e)=\Gamma_0(Ne_0) \cap \Gamma^0(e)$. The numbers $e_0$ and $e$ are called the ramification numbers of $f(\tau)$ at $0$ and $i\infty$ respectively. We will say that $f(\tau)$ is unramified if $e=e_0=1$.




%\subsection{The local partitions functions as eta products}

Let $\Delta$ be a simply laced root system. We introduce the notations
\[ Z_{\Delta}(\tau)=q^{-\frac{1}{24}}Z_{[\CC^2/G_\Delta]}(q)  \]
and 
\[  \Theta_{\Delta}(\tau)=\Theta_{ \zeta/k ,k}(\tau ). \]
In particular,
\[ Z_{\Delta}(\tau)=\frac{\Theta_{\Delta}(\tau)}{\eta(k\tau)^n}, \]
where $n$ is the rank of $\Delta$, and $k$ is the order of the corresponding finite group.
We will show that in the cases when $\Delta=A_n$, $n\geq 1$, $D_4, D_6, D_7$ or $E_6$ the functions $\Theta_{\Delta}(\tau)$, and hence $Z_{\Delta}(\tau)$, can be expressed as eta products. Conjecturally the same statement holds for all ADE types.


%\begin{example} In the case of $\Delta=A_1$, $\Theta_{A_1}(\tau)=\frac{\eta^2(2\tau)}{\eta(\tau)}$ which is a noncuspidal holomorphic modular form of weight $1/2$ and level 2 \cite[Example 3.12 (1)]{kohler2011eta}.\end{example}
We will denote the eta products on the right hand sides of Conjecture \ref{conj:zloceta} (\ref{it:zloceta1}) by $\eta_{\Delta}(\tau)$. Then Conjecture \ref{conj:zloceta} boils down to showing that $\Theta_{\Delta}(\tau)=\eta_{\Delta}(\tau)$. 



%A weight $k$ weakly holomorphic modular form is a function on $\mathbb{H}$ that obeys the weight $k$
%modular transformation law for $\Gamma_0(N)$, is holomorphic on $\mathbb{H}$, but may possess poles at the
%cusps. 

%\subsection{A procedure for proving Conjecture \ref{conj:zloceta}}
\subsection{Sturm bounds and the proof of Theorem~???}
To compare the eta products of Conjecture \ref{conj:zloceta} with the theta functions of Section \ref{sec:loccontr} we want to show that they are modular forms and also obtain their multiplier systems.

To prove Conjecture \ref{conj:zloceta} we need to show that $\Theta_{\Delta}(\tau)=\eta_{\Delta}(\tau)$. Theorem \ref{thm:etaprodmodeven} provides the multiplier system of $\eta_{\Delta}(\tau)$ only for root systems of even rank. It is possible to obtain an analog of Theorem \ref{thm:etaprodmodeven} for root systems of odd rank as well. Since these calculations would be too circuitous, we instead reduce to the case of root systems of even rank by taking a direct sum with $A_1$, for which the identity $\Theta_{A_1}(\tau)=\eta_{A_1}(\tau)$ is known by Example \ref{ex:anlocaleta}. More precisely, for computational reasons in the odd rank cases we will show that
\[\Theta_{\Delta}(\tau)\cdot \Theta_{A_1}\left(\frac{k}{2}\tau\right)=\eta_{\Delta}(\tau)\cdot \eta_{A_1}\left(\frac{k}{2}\tau\right).\]	
%For $n$ odd we will in fact show that 
%\[ \Theta_{\Delta}(\tau)\theta(\tau)=\eta_{\Delta}(\tau)\frac{\eta^5(2\tau)}{\eta^2(\tau)\eta^2(4\tau)}.\]
\begin{lemma}
	\label{lem:ee0s}
	If $\Delta$ is an irreducible simply laced root system of rank $n$ with $|G_{\Delta}|=k$, then for $\eta_{\Delta}(\tau)$ the numbers appearing in Theorem \ref{thm:etaprodmodeven}	 are as follows:
	\[ e=\frac{24}{gcd(24, k(n+1)-1)}; \quad e_0=1; \quad s=k^n|C_{\Delta}|; \quad N=k. \]
\end{lemma}
\begin{proof} Direct calculation shows that in each case $\sum_{m \mid N} ma_m=k(n+1)-1$, and $\sum_{m \mid N} \frac{N}{m}a_m=0$. The third identity is also straightforward.
\end{proof}
\begin{lemma}
	\label{lem:ee0sa1}
	If $\Delta$ is an irreducible simply laced root system of odd rank $n$ with $|G_{\Delta}|=k$, then for $\eta_{\Delta}(\tau) \cdot \eta_{A_1}(\frac{k}{2}\tau)$ the numbers appearing in Theorem \ref{thm:etaprodmodeven} are as follows:
	\[ e=\frac{24}{gcd(24, k(n+\frac{5}{2})-1)}; \quad e_0=1; \quad s=2k^{n+1}|C_{\Delta}|; \quad N=k. \]
\end{lemma}
\begin{proof} For $\eta_{A_1}(\frac{k}{2}\tau)=\eta^2(k\tau)\eta^{-1}(\frac{k}{2}\tau)$, $\sum_{m \mid N} ma_m=\frac{3k}{2}$ and $\prod_{m \mid N} m^{a_m}=2k$. \end{proof}
\begin{corollary}  Let $\Delta$ be a simply laced root system. 
	\begin{enumerate}
		\item If $n$ is even, let $e$ be as Lemma \ref{lem:ee0s}. Then the function $\eta_{\Delta}(\tau)$ is a modular form of weight $\frac{n}{2}$ for $\Gamma_0(k)\cap\Gamma^0(e)$ with the multiplier system defined by
		\[ \gamma(A)=\left( \frac{(-1)^{n/2}k^n|C_{\Delta}|}{d} \right). \]
		%\[ \gamma(A)=\left( \frac{k^n|C_{\Delta}|}{d} \right) \left( \overline{\epsilon}_d \left(\frac{2c}{d}\right) \right)^n. \]
		\item If $n$ is odd, let $e$ be as Lemma \ref{lem:ee0sa1}. Then the function $\eta_{\Delta}(\tau) \cdot \eta_{A_1}(\frac{k}{2}\tau)$ if a modular form of weight $\frac{n+1}{2}$ for $\Gamma_0(k)\cap\Gamma^0(e)$ with the multiplier system defined by
		\[ \gamma(A)=\left( \frac{(-1)^{(n+1)/2}2k^{n+1}|C_{\Delta}|}{d} \right). \]
	\end{enumerate}
	%The function $\eta_{\Delta}(e\tau)$ is a modular form of weight $\frac{n}{2}$ for $\Gamma_0(ke)$ with the multiplier system defined by
	%\[ \gamma(A)=\left( \frac{k^n|C_{\Delta}|}{d} \right) \left( \overline{\epsilon}_d \left(\frac{2c}{d}\right) \right)^n. \]
	\label{cor:etamultsyscalc}
\end{corollary}
\begin{proof} Follows from Theorem \ref{thm:etaprodmodeven}.
\end{proof}


\begin{lemma} 
	\label{lem:grpint}
	Let $\Delta$ be an irreducible simply laced root system. Let $k=|G_{\Delta}|$, and let $e$ be as in Corollary \ref{cor:etamultsyscalc}. 
	\begin{enumerate}
		\item If $\Delta$ is either of type D and even rank or of type E, then $e|k$. As a consequence, $\Gamma(km,m) \cap \Gamma_0(k) \cap \Gamma^0(e)=\Gamma(km,m)$.
		\item If $\Delta$ is of type D and odd rank, then $e|2k$. As a consequence, $\Gamma(km,m) \cap \Gamma_0(k) \cap \Gamma^0(e)=\Gamma(km,m/2)$.
	\end{enumerate}
\end{lemma}
\begin{proof}
	We will show that $\Gamma(km,m)$ (resp. $\Gamma(km,m/2)$) is contained in $\Gamma_0(k) \cap \Gamma^0(e)$. For this we only need that $e|k$ (resp. $e|2k$). In the type $E$ case this is automatic, since $e|24$ always and $24|k$ in all three cases.
	
	Let $\Delta$ be of type $D$ whose rank $n$ is even. Then $k(n+1)-1=(4n-8)(n+1)-1=4n^2-4n-9$, which is always an odd number. The divisors of 24 are 2 and 3. So the only possibilities for $e$ are 8 and 24 depending on whether $4n^2-4n-9$ is divisible by 3 or not. Now $4n^2-4n-9=4n(n-1)-9$, so it is not divisible by 3 if and only if $n \equiv -1 (\; \mathrm{mod}\; 3)$. Hence $e=24$ if and only if $n=6l+2$ for some integer $l$. But this means that $k=4n-8=4(6l+2)-8=24l$, so $24|k$. In the cases when $n=6l$ (resp. $n=6l+4$) the order $k=24l-8$ (resp. $k=24l+8$). So in both cases $8|k$. 
	
	Suppose now that $\Delta$ is of type $D$ whose rank $n$ is odd. Then $k(n+\frac{5}{2})-1)=(4n-8)(n+\frac{5}{2})-1=4n^2+2n-21$, which is again always an odd number. Similarly as above, it is not divisible by 3 if and only if $n \equiv -1 (\; \mathrm{mod}\; 3)$.
	If this is the case, then $e=24$ and $n=6l+5$ for some integer $l$. Then, $k=4n-8=4(6l+5)-8=24l-12$, so $e|2k$.
	The other case is when $e=8$ . Then either $n=6l+1$ for some integer $l$ and hence $k=4n-8=4(6l+1)-8=24l-4$, or $n=6l+3$ and hence $k=4n-8=4(6l+3)-8=24l-12$. In both cases $e|2k$.
\end{proof}

\begin{corollary} Let $\Delta$ be an irreducible simply laced root system of rank $n$. Let $k=|G_{\Delta}|$, and let $e$ be as in Corollary \ref{cor:etamultsyscalc}. 
	Let \[\Gamma=\begin{cases}
	\Gamma(km,m), \textrm{ if } \Delta \textrm{ is of type D and } n \textrm{ is even, or } \Delta \textrm{ is of type E} \\
	\Gamma(km,m/2), \textrm{ if } \Delta \textrm{ is of type D and } n \textrm{ is odd}.
	\end{cases}\]
	Then both
	\[
	\begin{cases}
	\eta_{\Delta}(\tau) \textrm{ and }\Theta_{\Delta}(\tau), \textrm{ if } n \textrm{ is even}, \\
	\eta_{\Delta}(\tau) \cdot \eta_{A_1}(\frac{k}{2}\tau) \textrm{ and }\Theta_{\Delta}(h\tau) \cdot \Theta_{A_1}(\frac{k}{2}\tau), \textrm{ if } n \textrm{ is odd},
	\end{cases}
	\]
	are modular forms for $\Gamma$ of the same weight and they have the same multiplier system.
	\label{cor:etathetamod}
\end{corollary}
\begin{proof}
	Follows from Corollary \ref{cor:thetazetamult}, Corollary \ref{cor:etamultsyscalc} and Lemma \ref{lem:grpint}.
\end{proof}



%Write 
%\[ \frac{1}{24} \sum_m \frac{N}{m}a_m = \frac{c_0}{e_0} \]
%and 
%\[ \frac{1}{24} \sum_m ma_m = \frac{c}{e}. \]
%Then 


%In each cases of Conjecture \ref{conj:zloceta}, $Z_{\Delta}^{-1}(\tau)$ transforms as a weight $1/2$-modular form, and they are automatically holomorphic on $\mathbb{H}$.
The next result gives a limit up to which the vanishing of the Fourier coefficients of a modular form guarantees the vanishing of the modular form. It is generally known as the Sturm bound. %We have not found this formulation of the Lemma \ref{lem:sturmbound} in the literature, only an analogous one where $\Gamma(N)$ is replaced with $\Gamma_0(N)$ \cite[Lemma 3.1]{kumar2014note}.
\begin{theorem}[{\cite[Theorem 1]{sturm1987congruence}}]
	\label{thm:sturmbound}
	Let $\Gamma$ be a congruence subgroup of $\mathrm{SL}_2(\ZZ)$, $n$ be a positive even integer, and $\vartheta$ be a multiplier system for $\Gamma$. Let $f=\sum_{m=0}^{\infty} a(m)q^m \in M_{n/2}(\Gamma,\vartheta)$. If $a(m)=0$ for all $m \leq \frac{n}{24} [\mathrm{SL}_2(\ZZ) \, : \, \Gamma ]$, then $f=0$. As a consequence, if the Fourier coefficients of two modular forms in $M_{n/2}(\Gamma,\vartheta)$ agree at least up to degree $\frac{n}{24} [\mathrm{SL}_2(\ZZ) \, : \, \Gamma ]$, then the two modular forms are equal.
\end{theorem}
\begin{corollary}
	For every fixed $\Delta$ Conjecture \ref{conj:zloceta} can be checked numerically.
\end{corollary}
\begin{proof}
	%One need to check two things about the holomorphic modular form $\Theta_{\Delta}(\tau)$ and the corresponding eta product. First, whether $\Theta_{\Delta}(\tau)$ tranforms accoding to the same Dirichlet character as the eta product. This is true by comparing Corollary \ref{cor:thetazetamult} with Corollary \ref{cor:etaprodmultalt}. Second,
	Because of Theorem \ref{thm:sturmbound}
	one only has to check whether the $q$-expansions of the two functions from Corollay \ref{cor:etathetamod} agree at least up to order $\frac{n}{24} [\mathrm{SL}_2(\ZZ) \, : \, \Gamma(N,m) ]$.
\end{proof}


%\subsection{Proof of Conjecture \ref{conj:zloceta} in certain cases}
\begin{proposition}
	\label{prop:zlocetacheck}
	Conjecture \ref{conj:zloceta} is true in the cases when $\Delta=A_n$, $n\geq 1$, $D_4, D_5, D_6$ or $E_6$.
\end{proposition}

\begin{proof}[{Proof of Proposition \ref{prop:zlocetacheck}}]
	The case $\Delta=A_n$ is explained in Example \ref{ex:anlocaleta} above. 
	
	
	
	In all other cases, taking into account the index formula from Lemma \ref{lem:indgNmsl2}, the groups $\Gamma(km,m)$ (resp. $\Gamma(km,m/2)$) and the corresponding Sturm bounds are calculated in Table \ref{table:sturm}. We performed a computer check in each case and found that the Fourier coefficients agree at least up to the appropriate bound.
	
	\begin{table}
	\begin{center}
		\begin{tabular}{ |c|c|c|c|c|c| }
			\hline
			$\Delta$ & $k$ & $m$ & $\Gamma$ & $[\mathrm{SL}_2(\ZZ) \, : \, \Gamma]$ & Sturm bound \\
			\hline
			$D_4$ & 8 & 2 & $\Gamma(16,2)$ & 768 & 128 \\
			%		$D_5$ & 12 & 24 & 288 & 2034 & 2034 & 960 & 24 \\
			%		$D_5\oplus A_1$ & 24 & 12 & 288 & 1152 & 1152 & 576 & 24 \\
			$D_5\oplus A_1$ & 12 & 8 & $\Gamma(96,4)$ & 36864 & 9216 \\
			$D_6$ & 16 & 4 & $\Gamma(64,4)$ & 12288 & 3072 \\
			$E_6$ & 24 & 3 & $\Gamma(72,3)$ & 20736 & 5184 \\
			%		& $n \equiv 2\;\mathrm{mod}\;4$ & 4 \\
			%		& $n \equiv 0\;\mathrm{mod}\;4$ & 2 \\
			%		$E_6$ & & 3 \\
			%		$E_7$ & & 4 \\
			%		$E_8$ & & 1 \\
			\hline
		\end{tabular}
		\vspace{0.2in}
		\caption{Sturm bounds}
\end{center}
		\label{table:sturm}
	\end{table}
\end{proof}



%\subsection{Miscellany}

%\pagebreak


\appendix
\section{Some proofs}
\subsection{Proof of Lemma \ref{lem:indgNmsl2}}
\label{sec:app0}

The following indices are known \cite[Section 1.2]{diamond2005first}:
\begin{gather}
\label{eq:GGind} [\mathrm{Sl}_2(\ZZ) \, : \, \Gamma(N)] = N^3 \prod_{p \mid N}\left(1-\frac{1}{p^2} \right), \\
\label{eq:G0G1ind} [\Gamma_0(N) \, : \, \Gamma_1(N) ]= \phi(N), \\
\label{eq:GGind} [\Gamma_1(N) \, : \, \Gamma(N) ]= N.
\end{gather}
The index \eqref{eq:G0G1ind} is because of the following. Due to $c \equiv 0 \;( \mathrm{mod}\;N)$ we must have $ad\equiv 1 \;( \mathrm{mod}\;N)$. This equation has $\phi(N)$ solutions, but we only allow one of these in $\Gamma_1(N)$: the one with $a\equiv d\equiv 1 \;(\mathrm{mod}\;N)$. The $\phi(N)$ residue classes modulo $N$ are distributed uniformly into the $\phi(N/m)$ relative prime residue classes modulo $N/m$. Hence, the congruence $a \equiv 1 \;( \mathrm{mod}\;N/m)$ has $\phi(N)/\phi(N/m)$ residue classes as solutions. If $N/m=2$, this is all the solutions of $a \equiv \pm 1 \;( \mathrm{mod}\;N/m)$. If $N/m>2$, then the number of solutions of $a \equiv \pm 1 \;( \mathrm{mod}\;N/m)$ is $2\phi(N)/\phi(N/m)$. As a consequence, the congruence subgroup
\begin{equation}  \left\{ \begin{pmatrix} a & b \\ c & d \end{pmatrix} \in \mathrm{SL}_2(\ZZ): a \equiv \pm 1 \;(\mathrm{mod}\;N/m), c \equiv 0 \;(\mathrm{mod}\;N) \right\} 
\label{eq:auxcong}
\end{equation}
in $\Gamma_0(N)$ has index $\phi(N)/\phi(N/m)=\phi(N)$ if $N/m=2$, and $2\phi(N)/\phi(N/m)$ if $N/m>2$.

Second, the index \eqref{eq:GGind} comes from requireing $b \equiv 1 \;( \mathrm{mod}\;N)$. Similarly the index of $\Gamma(N,m)$ inside the group defined in \eqref{eq:auxcong} is $N/m$. Combining all these we obtain that 
\[ [\Gamma_0(N) \, : \, \Gamma(N,m) ]= \begin{cases}
2N \phi(N), \textrm{ if } N/m =2 \\
\frac{2N \phi(N)}{m \phi(N/m)}, \textrm{ if } N/m >2.
\end{cases} \]

The index of $\Gamma_0(N)$ inside $\mathrm{SL}_2(\ZZ)$ turns out to be
\[  [\mathrm{SL}_2(\ZZ) \, : \, \Gamma_0(N) ]=\frac{N^2}{\phi(N)}\prod_{p \mid N}\left(1-\frac{1}{p^2} \right). \]
Hence,
\begin{multline*} [\mathrm{SL}_2(\ZZ) \, : \, \Gamma(N,m) ] = [\mathrm{SL}_2(\ZZ) \, : \, \Gamma_0(N) ] \cdot [\Gamma_0(N) \, : \, \Gamma(N,m) ] \\
=\begin{cases} 
2N^2\prod_{p \mid N}\left(1-\frac{1}{p^2} \right),\textrm{ if } N/m =2 \\
2N^2\prod_{p \mid N}\left(1-\frac{1}{p^2} \right) \prod_{p \mid N/m }\left(\frac{p}{p-1} \right),\textrm{ if } N/m >2,
\end{cases}
\end{multline*}
where we have used the following expression for the $\phi$ function:
\[ \phi(N)=N \prod_{p \mid  N}\left(1-\frac{1}{p} \right). \]

\subsection{Proof of Lemma \ref{lem:betalength}}
\label{sec:app1}
We follow the notations of \cite{bourbaki2002lie}. %Bourbaki Chapter 6, Section 4.
\begin{proof}[Type $A_n$, $n \geq 1$]
	\label{sec:app11}
	In this case $\mathrm{dim} \rho_i = 1$, $0 \leq i \leq n$. This implies that 
	\begin{itemize}
		\item $k=n+1=h$, the (dual) Coxeter number of the root system,
		\item and $\zeta = C_{\Delta}^{-1} \cdot (1,\dots,1)= \rho$, the sum of the positive roots.
		% \item and as a consequence, $\beta=\rho/h$.
	\end{itemize}
	The ``strange formula'' of Freudenthal-de Vries \cite{freudenthal1969linear} says that for any simple Lie algebra:
	\[ \frac{(\rho|\rho)}{2h}=\frac{\rho^{\top} \cdot C_\Delta \cdot \rho}{2h}=\frac{\mathrm{dim} \mathfrak{g}_{\Delta}}{24}. \]
	It is known that $\mathrm{dim} \mathfrak{g}_{A_n}= n(n+2)$. Hence,
	\[ \frac{(\zeta|\zeta)}{2k}=\frac{\rho^{\top} \cdot C_\Delta \cdot \rho}{2h}=\frac{n(n+2)}{24}=\frac{(n+1)^2-1}{24} \]
	as claimed.
\end{proof}

\begin{proof}[Type $D_n$, $n \geq 4$]
	In this and the remaining cases we will do direct calculation. 
	
	Let $V=\mathbb{R}^n$ and let $\varepsilon_1, \dots, \varepsilon_n$ be the canonical basis of $V$. Put
	\[
	\begin{array}{rcl}
	\alpha_1 & = &\varepsilon_1- \varepsilon_2 \\
	\alpha_2 & = &\varepsilon_2- \varepsilon_3 \\
	& \vdots &  \\
	\alpha_{n-1} & = &\varepsilon_{n-1} - \varepsilon_n \\
	\alpha_{n} & =  & \varepsilon_{n-1} + \varepsilon_n.
	\end{array}
	\]
	Then $\alpha_1,\dots,\alpha_n$ is the set of simple positive roots for $\Delta$ of type $D_n$. The fundamental weights are
	\[ 
	\begin{aligned}
	\omega_i & = \varepsilon_1 + \varepsilon_2 + \dots + \varepsilon_i\\
	& = \alpha_1 +2 \alpha_2 + \dots + (i-1)\alpha_{i-1} +i(\alpha_i+\alpha_{i+1}\dots+\alpha_{n-2})+\frac{1}{2}i(\alpha_{n-1}+\alpha_{n}) 
	\end{aligned}
	\]
	for $i < n-1$, and
	\[
	\begin{aligned}
	\omega_{n-1} & = \frac{1}{2}(\varepsilon_1 + \varepsilon_2 + \dots + \varepsilon_{n-2} +\varepsilon_{n-1}-\varepsilon_{n}) \\
	& = \frac{1}{2}(\alpha_1 + 2 \alpha_2 + \dots + (n-2)\alpha_{n-2}+\frac{1}{2}n\alpha_{n-1}+\frac{1}{2}(n-2)\alpha_n), \\
	\omega_{n} & = \frac{1}{2}(\varepsilon_1 + \varepsilon_2 + \dots + \varepsilon_{n-2} +\varepsilon_{n-1}+\varepsilon_{n}) \\
	& = \frac{1}{2}(\alpha_1 + 2 \alpha_2 + \dots + (n-2)\alpha_{n-2}+\frac{1}{2}(n-2)\alpha_{n-1}+\frac{1}{2}n\alpha_n). \\
	\end{aligned}
	\]
	Moreover,
	\[ \zeta = C_{\Delta}^{-1} \cdot (1,2,\dots,2,1,1) = \omega_1 + 2\omega_{2} + \dots + 2 \omega_{n-2}+ \omega_{n-1} + \omega_{n-2}. \]
	A quick computation shows that in terms of the roots $\alpha_1,\dots,\alpha_n$
	\[ \zeta= \sum_{i=1}^{n-2} 2\left(in - \frac{(i+1)(i+1)}{2}\right)\alpha_i+ \frac{n^2-2n}{2}(\alpha_{n-1}+\alpha_{n}). \]
	Hence,
	\[ 
	\begin{aligned}
	(\zeta|\zeta) & = (1,2,\dots,2,1,1)^{\top} \cdot C_{\Delta}^{-1} \cdot (1,2,\dots,2,1,1)  \\
	& = -(2n-4)+4 \left((n-1)\sum_{i=1}^{n-2}i -\frac{1}{2}\sum_{i=1}^{n-2}i^2 - \frac{n-2}{2}\right)+n^2-2n \\
	& = \frac{4(n-1)(n-2)(n-1)}{2}-\frac{4(n-2)(n-1)(2n-3)}{12}+n^2-6n+8 \\
	& = \frac{4 n^3 - 12n^2  -n+ 8}{3}=\frac{(n-2)(4n^2-4n-9)}{3}.
	\end{aligned}
	\]
	This implies that
	\[ \frac{(\zeta|\zeta)}{2k}=\frac{(n-2)(4n^2-4n-9)}{24(n-2)}=\frac{4n^2-4n-9}{24}. \]
	This is equal to 
	\[ \frac{(n+1)k-1}{24}=\frac{(n+1)(4n-8)-1}{24}\]
	as claimed.
\end{proof}

\begin{proof}[Type $E_6$]
	
	
	\[
	\frac{1}{2\cdot 24}
	\begin{pmatrix}
	1 \\ 2 \\ 3 \\ 2 \\ 2 \\ 1
	\end{pmatrix}^{\top}
	\cdot
	\begin{pmatrix}
	2 & -1 & 0 & 0 & 0 & 0 \\
	-1 & 2 & -1 & 0 & 0 & 0\\
	0 & -1 & 2 & -1 & -1 & 0 \\
	0 & 0 & -1 & 2 & 0 & 0 \\
	0 & 0 & -1 & 0 & 2 & -1 \\
	0 & 0 & 0 & 0 & -1 & 2
	\end{pmatrix}^{-1}
	\cdot
	\begin{pmatrix}
	1 \\ 2 \\ 3 \\ 2 \\ 2 \\ 1
	\end{pmatrix}
	=\frac{7\cdot 24-1}{24}.
	\]
\end{proof}

\begin{proof}[Type $E_7$]
	\[ 
	\frac{1}{2\cdot 48}
	\begin{pmatrix}
	1 \\ 2 \\ 3 \\ 4 \\2  \\ 3 \\ 2
	\end{pmatrix}^{\top}
	\cdot
	\begin{pmatrix}
	2 & -1 & 0 & 0 & 0 & 0 & 0 \\
	-1 & 2 & -1 & 0 & 0 & 0 & 0\\
	0 & -1 & 2 & -1 & 0 & 0 & 0 \\
	0 & 0 & -1 & 2 & -1 & -1 & 0 \\
	0 & 0 & 0 & -1 & 2 & 0 & 0 \\
	0 & 0 & 0 & -1 & 0 & 2 & -1 \\
	0 & 0 & 0 & 0  & 0 & -1 & 2
	\end{pmatrix}^{-1}
	\cdot
	\begin{pmatrix}
	1 \\ 2 \\ 3 \\ 4 \\2  \\ 3 \\ 2
	\end{pmatrix}
	=\frac{8\cdot 48-1}{24}.
	\]
\end{proof}

\begin{proof}[Type $E_8$]
	\[ 
	\frac{1}{2\cdot 120}
	\begin{pmatrix}
	2 \\ 3 \\ 4 \\5  \\ 6 \\ 3 \\ 4 \\ 2
	\end{pmatrix}^{\top}
	\cdot
	\begin{pmatrix}
	2 & -1 & 0 & 0 & 0 & 0 & 0 & 0\\
	-1 & 2 & -1 & 0 & 0 & 0 & 0 & 0\\
	0 & -1 & 2 & -1 & 0 & 0 & 0 & 0 \\
	0 & 0 & -1 & 2 & -1 & 0 & 0 & 0 \\
	0 & 0 & 0 & -1 & 2 & -1 & -1 & 0 \\
	0 & 0 & 0 & 0 & -1 & 2 & 0 & 0 \\
	0 & 0 & 0 & 0  & -1 & 0 & 2 & -1 \\
	0 & 0 & 0 & 0 & 0  & 0 & -1 & 2
	\end{pmatrix}^{-1} 
	\cdot
	\begin{pmatrix}
	2 \\ 3 \\ 4 \\5  \\ 6 \\ 3 \\ 4 \\ 2
	\end{pmatrix}
	=\frac{9\cdot 120-1}{24}.
	\]
\end{proof}

\section{Formulations of the conjecture}

This section is temporary, just to record this somewhere.
\smallskip

\subsection{Number theoretic formulation}

The natural number solutions $(e,f,v,M)$ to the equation
\[
e+f+v=M+2,\quad e,f,v \text{ divide }M
\]
have an ADE classification. For a given $ADE$ root system $\Delta $ and
corresponding solution $ (e,f,v,M)$, our conjecture is then equivalent to
\[
Z_{\Delta } = \frac{\eta^{2}(2\tau )}{\eta (\tau )} \cdot \left\{\frac{\eta
(2M\tau ) }{\eta (2e\tau )\eta (2f\tau )\eta (2v\tau )} \right\} .
\]

\subsection{Group theoretic formulation}

Solutions to the number theoretic formulation have a group theory
interpretation: Let $H\subset SO(3)$ be a finite subgroup. Then $H$
has a presentation of the form
\[
H=\{\left\langle \alpha ,\beta ,\gamma \right\rangle :\quad \alpha \beta \gamma =\alpha^{p}=\beta^{q}=\gamma^{r}=1 \}
\]
where 
\[
\frac{1}{p} + \frac{1}{q} + \frac{1}{r} =1+\frac{2}{M}, \quad p,q,r|
M 
\]
and $M=|H|$.

The above equation is equivalent to the previous formulation with 
\[
p\cdot e=q\cdot f= r\cdot v = M.
\]
The classification of solutions is given by the following table. 

\begin{center}
\begin{tabular}{|c|c|c|c|c|}
\hline
Type &	$H$ &	$M$&	$(p,q,r)$ &	$(e,f,v)$\\ \hline \hline 
$A_{n}$     & cyclic & $n+1$ & $(1,n+1,n+1)$& $(n+1,1,1)$	\\ \hline
$D_{n}$     & dihedral & $2n-2$ & $(2,2,n-2)$& $(n-1,n-1,2)$	\\ \hline
$E_{6}$     & tetrahedral & $12$ & $(2,3,3)$& $(6,4,4)$	\\ \hline
$E_{7}$     & octahedral & $24$ & $(2,3,4)$& $(12,8,6)$	\\ \hline
$E_{8}$     & icosohedral & $60$ & $(2,3,5)$& $(30,20,12)$	\\ \hline
\end{tabular}
\end{center}


Note that in $D_{n}$ and $E_{n}$ cases, the numbers $(e,f,v)$ can be
interpreted as the number of edges, faces, and vertices of the regular
polyhedron with the corresponding symmetry group. For $E_{n}$ these
are the platonic solids; for $D_{n}$ these are the degenerate regular
polyhedrons obtained from gluing two regular $(n-1)$-gons to each
other.

The groups $H\subset SO(3)$ are closely related to the groups
$G\subset SU(2)$ also classified by $ADE$ root systems. For the
$A_{n}$ case, both are isomorphic to $\ZZ_{n+1}$. For the $D_{n}$ and
$E_{n}$ cases, the group $G$ is the double cover of the group $H$
induced by the degree 2 map $SU(2)\to SO(3)$. 






\section{Table of eta products}
\renewcommand{\arraystretch}{1.5}



The following table provides the list of the modular forms
$Z_{X,G}^{-1}$, expressed as eta products, for each of the 82 possible
symplectic actions of a group $G$ on a $K3$ surface $X$. Our numbering
matches Xiao's \cite{xiao1996galois} whose table we refer to for a
description of each group.
\begin{longtable}{|l|l|l|l|l|}
  \hline
Xiao $\# $ & $|G|$ & Singularities of $X/G$&  The modular form $Z_{X,G}^{-1}$ & Weight \\ 
  \hline
0 & 1 &  & $ \eta \left( \tau \right)   ^{24}$ & 12 \\ 
  1 & 2 & $8 A_{1}$ & $ \eta \left( 2\tau \right)   ^{8}  \eta \left( \tau \right)   ^{8}$ & 8 \\ 
  2 & 3 & $6 A_{2}$ & $ \eta \left( 3\tau \right)   ^{6}  \eta \left( \tau \right)   ^{6}$ & 6 \\ 
  3 & 4 & $12 A_{1}$ & $ \eta \left( 2\tau \right)   ^{12}$ & 6 \\ 
  4 & 4 & $2 A_{1} + 4 A_{3}$ & $ \eta \left( 4\tau \right)   ^{4}  \eta \left( 2\tau \right)   ^{2}  \eta \left( \tau \right)   ^{4}$ & 5 \\ 
  5 & 5 & $4 A_{4}$ & $ \eta \left( 5\tau \right)   ^{4}  \eta \left( \tau \right)   ^{4}$ & 4 \\ 
  6 & 6 & $8 A_{1} + 3 A_{2}$ & ${\frac {  \eta \left( 3\tau \right)   ^{8}  \eta \left( 2\tau \right)   ^{3}}{\eta \left( 6\tau \right) }}$ & 5 \\ 
  7 & 6 & $2 A_{1} + 2 A_{2} + 2 A_{5}$ & $ \eta \left( 6\tau \right)   ^{2}  \eta \left( 3\tau \right)   ^{2}  \eta \left( 2\tau \right)   ^{2} \mbox{}  \eta \left( \tau \right)   ^{2}$ & 4 \\ 
  8 & 7 & $3 A_{6}$ & $ \eta \left( 7\tau \right)   ^{3}  \eta \left( \tau \right)   ^{3}$ & 3 \\ 
  9 & 8 & $14 A_{1}$ & ${\frac {  \eta \left( 4\tau \right)   ^{14}}{  \eta \left( 8\tau \right)   ^{4}}}$ & 5 \\ 
  10 & 8 & $9 A_{1} + 2 A_{3}$ & ${\frac {  \eta \left( 4\tau \right)   ^{9}  \eta \left( 2\tau \right)   ^{2}}{  \eta \left( 8\tau \right)   ^{2}}}$ & 9/2 \\ 
  11 & 8 & $4 A_{1} + 4 A_{3}$ & $ \eta \left( 4\tau \right)   ^{4}  \eta \left( 2\tau \right)   ^{4}$ & 4 \\ 
  12 & 8 & $3 A_{3} + 2 D_{4}$ & ${\frac {  \eta \left( \tau \right)   ^{2}  \eta \left( 4\tau \right)   ^{6}}{\eta \left( 2\tau \right) }}$ & 7/2 \\ 
  13 & 8 & $ A_{1} + 4 D_{4}$ & ${\frac {  \eta \left( 4\tau \right)   ^{13}  \eta \left( \tau \right)   ^{4}}{  \eta \left( 8\tau \right)   ^{2} \mbox{}  \eta \left( 2\tau \right)   ^{8}}}$ & 7/2 \\ 
  14 & 8 & $ A_{1} +  A_{3} + 2 A_{7}$ & $ \eta \left( 8\tau \right)   ^{2}\eta \left( 4\tau \right) \eta \left( 2\tau \right)   \eta \left( \tau \right)   ^{2}$ & 3 \\ 
  15 & 9 & $8 A_{2}$ & $ \eta \left( 3\tau \right)   ^{8}$ & 4 \\ 
  16 & 10 & $8 A_{1} + 2 A_{4}$ & ${\frac {  \eta \left( 5\tau \right)   ^{8}  \eta \left( 2\tau \right)   ^{2}}{  \eta \left( 10\tau \right)   ^{2}}}$ & 4 \\ 
  17 & 12 & $4 A_{1} + 6 A_{2}$ & ${\frac {  \eta \left( 6\tau \right)   ^{4}  \eta \left( 4\tau \right)   ^{6}}{  \eta \left( 12\tau \right)   ^{2}}}$ & 4 \\ 
  18 & 12 & $9 A_{1} +  A_{2} +  A_{5}$ & ${\frac {  \eta \left( 6\tau \right)   ^{9}\eta \left( 4\tau \right) \eta \left( 2\tau \right) }{  \eta \left( 12\tau \right)   ^{3}}}$ & 4 \\ 
  19 & 12 & $3 A_{1} + 3 A_{5}$ & $ \eta \left( 6\tau \right)   ^{3}  \eta \left( 2\tau \right)   ^{3}$ & 3 \\ 
  20 & 12 & $ A_{2} + 2 A_{3} + 2 D_{5}$ & ${\frac {  \eta \left( 4\tau \right)   ^{3}  \eta \left( 3\tau \right)   ^{2}  \eta \left( \tau \right)   ^{2} \mbox{}  \eta \left( 6\tau \right)   ^{4}}{\eta \left( 12\tau \right)   \eta \left( 2\tau \right)   ^{4}}}$ & 3 \\ 
  21 & 16 & $15 A_{1}$ & ${\frac {  \eta \left( 8\tau \right)   ^{15}}{  \eta \left( 16\tau \right)   ^{6}}}$ & 9/2 \\ 
  22 & 16 & $10 A_{1} + 2 A_{3}$ & ${\frac {  \eta \left( 8\tau \right)   ^{10}  \eta \left( 4\tau \right)   ^{2}}{  \eta \left( 16\tau \right)   ^{4}}}$ & 4 \\ 
  23 & 16 & $5 A_{1} + 4 A_{3}$ & ${\frac {  \eta \left( 8\tau \right)   ^{5}  \eta \left( 4\tau \right)   ^{4}}{  \eta \left( 16\tau \right)   ^{2}}}$ & 7/2 \\ 
  24 & 16 & $6 A_{1} +  A_{3} + 2 D_{4}$ & ${\frac {  \eta \left( 8\tau \right)   ^{12}  \eta \left( 2\tau \right)   ^{2}}{  \eta \left( 16\tau \right)   ^{4} \mbox{}  \eta \left( 4\tau \right)   ^{3}}}$ & 7/2 \\ 
  25 & 16 & $6 A_{3}$ & $ \eta \left( 4\tau \right)   ^{6}$ & 3 \\ 
  26 & 16 & $4 A_{1} +  A_{3} +  A_{7} +  D_{4}$ & ${\frac {  \eta \left( 8\tau \right)   ^{7}  \eta \left( 2\tau \right)   ^{2}}{  \eta \left( 16\tau \right)   ^{2} \mbox{}\eta \left( 4\tau \right) }}$ & 3 \\ 
  27 & 16 & $2 A_{1} + 4 D_{4}$ & ${\frac {  \eta \left( 8\tau \right)   ^{14}  \eta \left( 2\tau \right)   ^{4}}{  \eta \left( 4\tau \right)   ^{8} \mbox{}  \eta \left( 16\tau \right)   ^{4}}}$ & 3 \\ 
  28 & 16 & $2 A_{1} +  A_{3} + 2 A_{7}$ & $ \eta \left( 8\tau \right)   ^{2}\eta \left( 4\tau \right)   \eta \left( 2\tau \right)   ^{2}$ & 5/2 \\ 
  29 & 16 & $ A_{3} +  D_{4} + 2 D_{6}$ & ${\frac {\eta \left( 4\tau \right)   \eta \left( 8\tau \right)   ^{7}  \eta \left( \tau \right)   ^{2}}{  \eta \left( 16\tau \right)   ^{2} \mbox{}  \eta \left( 2\tau \right)   ^{3}}}$ & 5/2 \\ 
  30 & 18 & $8 A_{1} + 4 A_{2}$ & ${\frac {  \eta \left( 9\tau \right)   ^{8}  \eta \left( 6\tau \right)   ^{4}}{  \eta \left( 18\tau \right)   ^{4}}}$ & 4 \\ 
  31 & 18 & $2 A_{1} + 3 A_{2} + 2 A_{5}$ & ${\frac {  \eta \left( 9\tau \right)   ^{2}  \eta \left( 6\tau \right)   ^{3}  \eta \left( 3\tau \right)   ^{2} \mbox{}}{\eta \left( 18\tau \right) }}$ & 3 \\ 
  32 & 20 & $2 A_{1} + 4 A_{3} +  A_{4}$ & ${\frac {  \eta \left( 10\tau \right)   ^{2}  \eta \left( 5\tau \right)   ^{4}\eta \left( 4\tau \right) }{\eta \left( 20 \mbox{}\tau \right) }}$ & 3 \\ 
  33 & 21 & $6 A_{2} +  A_{6}$ & ${\frac {  \eta \left( 7\tau \right)   ^{6}\eta \left( 3\tau \right) }{\eta \left( 21\tau \right) }}$ & 3 \\ 
  34 & 24 & $5 A_{1} + 3 A_{2} + 2 A_{3}$ & ${\frac {  \eta \left( 12\tau \right)   ^{5}  \eta \left( 8\tau \right)   ^{3}  \eta \left( 6\tau \right)   ^{2} \mbox{}}{  \eta \left( 24\tau \right)   ^{3}}}$ & 7/2 \\ 
  35 & 24 & $4 A_{1} + 2 A_{2} + 2 A_{5}$ & ${\frac {  \eta \left( 12\tau \right)   ^{4}  \eta \left( 8\tau \right)   ^{2}  \eta \left( 4\tau \right)   ^{2} \mbox{}}{  \eta \left( 24\tau \right)   ^{2}}}$ & 3 \\ 
  36 & 24 & $5 A_{1} +  A_{3} +  A_{5} +  D_{5}$ & ${\frac {  \eta \left( 12\tau \right)   ^{7}\eta \left( 6\tau \right) \eta \left( 2\tau \right) \eta \left( 8\tau \right) }{  \eta \left( 24\tau \right)   ^{3} \mbox{}\eta \left( 4\tau \right) }}$ & 3 \\ 
  37 & 24 & $2 A_{2} +  A_{5} +  D_{4} +  E_{6}$ & ${\frac {  \eta \left( 8\tau \right)   ^{4}\eta \left( 4\tau \right) \eta \left( 3\tau \right)   \eta \left( 12\tau \right)   ^{4} \mbox{}\eta \left( \tau \right) }{  \eta \left( 6\tau \right)   ^{2}  \eta \left( 24\tau \right)   ^{2}  \eta \left( 2\tau \right)   ^{2}}}$ & 5/2 \\ 
  38 & 24 & $2 A_{2} +  A_{3} + 2 E_{6}$ & ${\frac {  \eta \left( 8\tau \right)   ^{6}\eta \left( 6\tau \right)   \eta \left( \tau \right)   ^{2}  \eta \left( 12\tau \right)   ^{2} \mbox{}}{  \eta \left( 2\tau \right)   ^{4}  \eta \left( 24\tau \right)   ^{2}}}$ & 5/2 \\ 
  39 & 32 & $8 A_{1} + 3 A_{3}$ & ${\frac {  \eta \left( 16\tau \right)   ^{8}  \eta \left( 8\tau \right)   ^{3}}{  \eta \left( 32\tau \right)   ^{4}}}$ & 7/2 \\ 
  40 & 32 & $9 A_{1} + 2 D_{4}$ & ${\frac {  \eta \left( 16\tau \right)   ^{15}  \eta \left( 4\tau \right)   ^{2}}{  \eta \left( 32\tau \right)   ^{6} \mbox{}  \eta \left( 8\tau \right)   ^{4}}}$ & 7/2 \\ 
  41 & 32 & $3 A_{1} + 5 A_{3}$ & ${\frac {  \eta \left( 16\tau \right)   ^{3}  \eta \left( 8\tau \right)   ^{5}}{  \eta \left( 32\tau \right)   ^{2}}}$ & 3 \\ 
  42 & 32 & $4 A_{1} + 2 A_{3} + 2 D_{4}$ & ${\frac {  \eta \left( 16\tau \right)   ^{10}  \eta \left( 4\tau \right)   ^{2}}{  \eta \left( 32\tau \right)   ^{4} \mbox{}  \eta \left( 8\tau \right)   ^{2}}}$ & 3 \\ 
  43 & 32 & $5 A_{1} + 2 A_{7}$ & ${\frac {  \eta \left( 16\tau \right)   ^{5}  \eta \left( 4\tau \right)   ^{2}}{  \eta \left( 32\tau \right)   ^{2}}}$ & 5/2 \\ 
  44 & 32 & $2 A_{1} + 2 A_{3} +  A_{7} +  D_{4}$ & ${\frac {  \eta \left( 16\tau \right)   ^{5}  \eta \left( 4\tau \right)   ^{2}}{  \eta \left( 32\tau \right)   ^{2}}}$ & 5/2 \\ 
  45 & 32 & $3 A_{1} +  D_{4} + 2 D_{6}$ & ${\frac {  \eta \left( 16\tau \right)   ^{10}  \eta \left( 2\tau \right)   ^{2}}{  \eta \left( 32\tau \right)   ^{4} \mbox{}  \eta \left( 4\tau \right)   ^{3}}}$ & 5/2 \\ 
  46 & 36 & $2 A_{1} + 2 A_{2} + 4 A_{3}$ & ${\frac {  \eta \left( 18\tau \right)   ^{2}  \eta \left( 12\tau \right)   ^{2}  \eta \left( 9\tau \right)   ^{4} \mbox{}}{  \eta \left( 36\tau \right)   ^{2}}}$ & 3 \\ 
  47 & 36 & $ A_{1} + 6 A_{2} +  A_{5}$ & ${\frac {\eta \left( 18\tau \right)   \eta \left( 12\tau \right)   ^{6}\eta \left( 6\tau \right) }{  \eta \left( 36 \mbox{}\tau \right)   ^{2}}}$ & 3 \\ 
  48 & 36 & $6 A_{1} +  A_{2} + 2 A_{5}$ & ${\frac {  \eta \left( 18\tau \right)   ^{6}\eta \left( 12\tau \right)   \eta \left( 6\tau \right)   ^{2}}{  \eta \left( 36 \mbox{}\tau \right)   ^{3}}}$ & 3 \\ 
  49 & 48 & $5 A_{1} + 6 A_{2}$ & ${\frac {  \eta \left( 24\tau \right)   ^{5}  \eta \left( 16\tau \right)   ^{6}}{  \eta \left( 48\tau \right)   ^{4}}}$ & 7/2 \\ 
  50 & 48 & $6 A_{2} + 2 A_{3}$ & ${\frac {  \eta \left( 16\tau \right)   ^{6}  \eta \left( 12\tau \right)   ^{2}}{  \eta \left( 48\tau \right)   ^{2}}}$ & 3 \\ 
  51 & 48 & $5 A_{1} +  A_{2} + 2 A_{3} +  A_{5}$ & ${\frac {  \eta \left( 24\tau \right)   ^{5}\eta \left( 16\tau \right)   \eta \left( 12\tau \right)   ^{2} \mbox{}\eta \left( 8\tau \right) }{  \eta \left( 48\tau \right)   ^{3}}}$ & 3 \\ 
  52 & 48 & $4 A_{1} + 3 A_{5}$ & ${\frac {  \eta \left( 24\tau \right)   ^{4}  \eta \left( 8\tau \right)   ^{3}}{  \eta \left( 48\tau \right)   ^{2}}}$ & 5/2 \\ 
  53 & 48 & $ A_{1} +  A_{2} + 2 A_{3} + 2 D_{5}$ & ${\frac {  \eta \left( 24\tau \right)   ^{5}  \eta \left( 16\tau \right)   ^{3}  \eta \left( 12\tau \right)   ^{2} \mbox{}  \eta \left( 4\tau \right)   ^{2}}{  \eta \left( 48\tau \right)   ^{3}  \eta \left( 8\tau \right)   ^{4}}}$ & 5/2 \\ 
  54 & 48 & $4 A_{1} +  A_{2} +  A_{7} +  E_{6}$ & ${\frac {  \eta \left( 24\tau \right)   ^{5}  \eta \left( 16\tau \right)   ^{3}\eta \left( 6\tau \right)  \mbox{}\eta \left( 2\tau \right) }{  \eta \left( 48\tau \right)   ^{3}  \eta \left( 4\tau \right)   ^{2}}}$ & 5/2 \\ 
  55 & 60 & $4 A_{1} + 3 A_{2} + 2 A_{4}$ & ${\frac {  \eta \left( 30\tau \right)   ^{4}  \eta \left( 20\tau \right)   ^{3}  \eta \left( 12\tau \right)   ^{2} \mbox{}}{  \eta \left( 60\tau \right)   ^{3}}}$ & 3 \\ 
  56 & 64 & $5 A_{1} + 3 A_{3} +  D_{4}$ & ${\frac {  \eta \left( 32\tau \right)   ^{8}\eta \left( 16\tau \right) \eta \left( 8\tau \right) }{  \eta \left( 64 \mbox{}\tau \right)   ^{4}}}$ & 3 \\ 
  57 & 64 & $6 A_{1} + 3 D_{4}$ & ${\frac {  \eta \left( 32\tau \right)   ^{15}  \eta \left( 8\tau \right)   ^{3}}{  \eta \left( 64\tau \right)   ^{6} \mbox{}  \eta \left( 16\tau \right)   ^{6}}}$ & 3 \\ 
  58 & 64 & $3 A_{1} + 3 A_{3} +  A_{7}$ & ${\frac {  \eta \left( 32\tau \right)   ^{3}  \eta \left( 16\tau \right)   ^{3}\eta \left( 8\tau \right)  \mbox{}}{  \eta \left( 64\tau \right)   ^{2}}}$ & 5/2 \\ 
  59 & 64 & $5 A_{3} +  D_{4}$ & ${\frac {  \eta \left( 32\tau \right)   ^{3}  \eta \left( 16\tau \right)   ^{3}\eta \left( 8\tau \right)  \mbox{}}{  \eta \left( 64\tau \right)   ^{2}}}$ & 5/2 \\ 
  60 & 64 & $4 A_{1} +  A_{3} + 2 D_{6}$ & ${\frac {  \eta \left( 32\tau \right)   ^{8}  \eta \left( 16\tau \right)   ^{3}  \eta \left( 4\tau \right)   ^{2} \mbox{}}{  \eta \left( 64\tau \right)   ^{4}  \eta \left( 8\tau \right)   ^{4}}}$ & 5/2 \\ 
  61 & 72 & $4 A_{1} + 3 A_{2} +  A_{3} +  D_{5}$ & ${\frac {  \eta \left( 36\tau \right)   ^{6}  \eta \left( 24\tau \right)   ^{4}\eta \left( 18\tau \right)  \mbox{}\eta \left( 6\tau \right) }{  \eta \left( 72\tau \right)   ^{4}  \eta \left( 12\tau \right)   ^{2}}}$ & 3 \\ 
  62 & 72 & $3 A_{1} + 2 A_{3} + 2 A_{5}$ & ${\frac {  \eta \left( 36\tau \right)   ^{3}  \eta \left( 18\tau \right)   ^{2}  \eta \left( 12\tau \right)   ^{2} \mbox{}}{  \eta \left( 72\tau \right)   ^{2}}}$ & 5/2 \\ 
  63 & 72 & $ A_{2} + 3 A_{3} + 2 D_{4}$ & ${\frac {\eta \left( 24\tau \right)   \eta \left( 9\tau \right)   ^{2}  \eta \left( 36\tau \right)   ^{6}}{  \eta \left( 72\tau \right)   ^{3} \mbox{}\eta \left( 18\tau \right) }}$ & 5/2 \\ 
  64 & 80 & $3 A_{1} + 4 A_{4}$ & ${\frac {  \eta \left( 40\tau \right)   ^{3}  \eta \left( 16\tau \right)   ^{4}}{  \eta \left( 80\tau \right)   ^{2}}}$ & 5/2 \\ 
  65 & 96 & $3 A_{1} + 3 A_{2} + 3 A_{3}$ & ${\frac {  \eta \left( 48\tau \right)   ^{3}  \eta \left( 32\tau \right)   ^{3}  \eta \left( 24\tau \right)   ^{3} \mbox{}}{  \eta \left( 96\tau \right)   ^{3}}}$ & 3 \\ 
  66 & 96 & $2 A_{1} + 2 A_{2} +  A_{3} + 2 A_{5}$ & ${\frac {  \eta \left( 48\tau \right)   ^{2}  \eta \left( 32\tau \right)   ^{2}\eta \left( 24\tau \right)  \mbox{}  \eta \left( 16\tau \right)   ^{2}}{  \eta \left( 96\tau \right)   ^{2}}}$ & 5/2 \\ 
  67 & 96 & $2 A_{1} + 3 A_{2} +  A_{7} +  D_{4}$ & ${\frac {  \eta \left( 48\tau \right)   ^{5}  \eta \left( 32\tau \right)   ^{3}  \eta \left( 12\tau \right)   ^{2} \mbox{}}{  \eta \left( 96\tau \right)   ^{3}  \eta \left( 24\tau \right)   ^{2}}}$ & 5/2 \\ 
  68 & 96 & $3 A_{1} + 2 A_{3} +  A_{5} +  D_{5}$ & ${\frac {  \eta \left( 48\tau \right)   ^{5}  \eta \left( 24\tau \right)   ^{2}\eta \left( 8\tau \right)  \mbox{}\eta \left( 32\tau \right) }{  \eta \left( 96\tau \right)   ^{3}\eta \left( 16\tau \right) }}$ & 5/2 \\ 
  69 & 96 & $3 A_{1} + 2 A_{2} + 2 E_{6}$ & ${\frac {  \eta \left( 48\tau \right)   ^{5}  \eta \left( 32\tau \right)   ^{6}  \eta \left( 4\tau \right)   ^{2} \mbox{}}{  \eta \left( 96\tau \right)   ^{4}  \eta \left( 8\tau \right)   ^{4}}}$ & 5/2 \\ 
  70 & 120 & $2 A_{1} +  A_{2} + 2 A_{3} +  A_{4} +  A_{5}$ & ${\frac {  \eta \left( 60\tau \right)   ^{2}\eta \left( 40\tau \right)   \eta \left( 30\tau \right)   ^{2} \mbox{}\eta \left( 24\tau \right) \eta \left( 20\tau \right) }{  \eta \left( 120\tau \right)   ^{2}}}$ & 5/2 \\ 
  71 & 128 & $3 A_{1} + 2 A_{3} +  D_{4} +  D_{6}$ & ${\frac {  \eta \left( 64\tau \right)   ^{8}\eta \left( 32\tau \right) \eta \left( 8\tau \right) }{  \eta \left( 128\tau \right)   ^{4} \mbox{}\eta \left( 16\tau \right) }}$ & 5/2 \\ 
  72 & 144 & $ A_{1} + 4 A_{2} + 2 A_{5}$ & ${\frac {\eta \left( 72\tau \right)   \eta \left( 48\tau \right)   ^{4}  \eta \left( 24\tau \right)   ^{2}}{  \eta \left( 144 \mbox{}\tau \right)   ^{2}}}$ & 5/2 \\ 
  73 & 160 & $2 A_{1} + 3 A_{3} + 2 A_{4}$ & ${\frac {  \eta \left( 80\tau \right)   ^{2}  \eta \left( 40\tau \right)   ^{3}  \eta \left( 32\tau \right)   ^{2} \mbox{}}{  \eta \left( 160\tau \right)   ^{2}}}$ & 5/2 \\ 
  74 & 168 & $ A_{1} + 3 A_{2} + 2 A_{3} +  A_{6}$ & ${\frac {\eta \left( 84\tau \right)   \eta \left( 56\tau \right)   ^{3}  \eta \left( 42\tau \right)   ^{2} \mbox{}\eta \left( 24\tau \right) }{  \eta \left( 168\tau \right)   ^{2}}}$ & 5/2 \\ 
  75 & 192 & $2 A_{1} + 6 A_{2} +  D_{4}$ & ${\frac {  \eta \left( 96\tau \right)   ^{5}  \eta \left( 64\tau \right)   ^{6}\eta \left( 24\tau \right)  \mbox{}}{  \eta \left( 192\tau \right)   ^{4}  \eta \left( 48\tau \right)   ^{2}}}$ & 3 \\ 
  76 & 192 & $2 A_{1} +  A_{2} + 2 A_{3} +  A_{5} +  D_{4}$ & ${\frac {  \eta \left( 96\tau \right)   ^{5}\eta \left( 64\tau \right) \eta \left( 32\tau \right)  \mbox{}\eta \left( 24\tau \right) }{  \eta \left( 192\tau \right)   ^{3}}}$ & 5/2 \\ 
  77 & 192 & $2 A_{1} +  A_{2} + 3 A_{3} +  E_{6}$ & ${\frac {  \eta \left( 96\tau \right)   ^{3}  \eta \left( 64\tau \right)   ^{3}  \eta \left( 48\tau \right)   ^{3} \mbox{}\eta \left( 8\tau \right) }{  \eta \left( 192\tau \right)   ^{3}  \eta \left( 16\tau \right)   ^{2}}}$ & 5/2 \\ 
  78 & 288 & $2 A_{1} + 2 A_{2} +  A_{3} + 2 D_{5}$ & ${\frac {  \eta \left( 144\tau \right)   ^{6}  \eta \left( 96\tau \right)   ^{4} \mbox{}\eta \left( 72\tau \right)   \eta \left( 24\tau \right)   ^{2}}{  \eta \left( 288\tau \right)   ^{4}  \eta \left( 48\tau \right)   ^{4}}}$ & 5/2 \\ 
  79 & 360 & $ A_{1} + 2 A_{2} + 2 A_{3} + 2 A_{4}$ & ${\frac {\eta \left( 180\tau \right)   \eta \left( 120\tau \right)   ^{2}  \eta \left( 90\tau \right)   ^{2} \mbox{}  \eta \left( 72\tau \right)   ^{2}}{  \eta \left( 360\tau \right)   ^{2}}}$ & 5/2 \\ 
  80 & 384 & $ A_{1} + 3 A_{2} + 2 A_{3} +  D_{6}$ & ${\frac {  \eta \left( 192\tau \right)   ^{3}  \eta \left( 128\tau \right)   ^{3} \mbox{}  \eta \left( 96\tau \right)   ^{3}\eta \left( 24\tau \right) }{  \eta \left( 384\tau \right)   ^{3}  \eta \left( 48\tau \right)   ^{2}}}$ & 5/2 \\ 
  81 & 960 & $ A_{1} + 3 A_{2} + 2 A_{4} +  D_{4}$ & ${\frac {  \eta \left( 480\tau \right)   ^{4}  \eta \left( 320\tau \right)   ^{3} \mbox{}  \eta \left( 192\tau \right)   ^{2}\eta \left( 120\tau \right) }{  \eta \left( 960\tau \right)   ^{3}  \eta \left( 240\tau \right)   ^{2}}}$ & 5/2 \\ 
   \hline
\caption{Table of the modular forms $Z_{X,G}^{-1}$ for all symplectic
$G$ actions.} \label{table: list of eta products}
\end{longtable}


\bibliography{k3hilb}
\bibliographystyle{plain}

\end{document}

